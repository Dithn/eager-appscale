The growth of the World Wide Web (WWW), web services, and cloud computing have
significantly influenced the way developers implement software applications.
Instead of implementing all the functionality from the scratch, developers
increasingly offload as much application functionality as possible to remote,
web-accessible application programming interfaces (web APIs) hosted ``in the
cloud''. As a result, web APIs are rapidly proliferating.
At the time of this writing, 
ProgrammableWeb~\cite{pweb}, a popular web API index, lists more than $11,000$
web APIs and a nearly 100\% annual growth rate. 

This proliferation of web APIs demands new techniques that
control and govern the evolution of APIs as a first-class software
resource. A lack of API governance can lead to 
security breaches, denial of service (DoS)
attacks, poor code reuse and violation of service-level agreements (SLAs). 
Unfortunately, most existing cloud platforms
within which web APIs are hosted provide only minimal governance support.

Toward this end, we propose EAGER ({\bf E}nforced {\bf A}PI {\bf G}overnance
{\bf E}ngine for {\bf R}EST), a model and an architecture that augments existing
cloud platforms in order to facilitate API governance as a 
cloud-native feature. EAGER enforces proper versioning of APIs and supports dependency 
management and comprehensive policy enforcement at API deployment-time. 

Deployment-time enforcement (heretofore unexplored) is attractive for several
reasons.  First, if run-time only API governance is implemented, 
policy violations will go undetected until the offending APIs are used.  
By enforcing governance at deployment-time,
EAGER implements ``fail fast'' in which violations are detected
immediately. Further, the overall
system is prevented from entering a non-compliant state which aids in the
certification of regulatory compliance. 

EAGER further enhances software maintainability by guaranteeing that 
developers reuse existing APIs when possible to create new software artifacts. 
Concurrently, it
tracks changes made by developers to deployed web APIs to prevent
any backwards-incompatible API changes from being put into production.

EAGER includes a language for specifying 
API governance policies.  It incorporates a developer-friendly 
Python programming language syntax for 
specifying complex policy statements in a simple and 
intuitive manner. Moreover, we ensure that specifying the required policies 
is the only additional activity that API providers should perform in
order to benefit from EAGER. All other API governance related verification and 
enforcement work is carried out by the cloud platform automatically.

To evaluate the proposed 
architecture, we implement EAGER as an extension to AppScale~\cite{appscale13}, 
an open source
cloud platform that emulates Google App Engine. We show that the EAGER 
architecture can be easily implemented in extant clouds with
minimal changes to the underlying platform technology. 

In the sections that follow, we present the design and implementation of
EAGER. We then empirically evaluate EAGER using a wide range of APIs and
experiments.  Finally, we discuss some related work, and conclude.
