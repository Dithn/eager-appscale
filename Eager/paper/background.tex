%Service Oriented Architecture (SOA)~\cite{Haines:2010:SAM:1787234.1787269} is a popular architectural style for
%mplementing complex systems in a modular fashion. A SOA system
%s implemented as a composition of a number of independent software components
%eferred to as ``services.'' Each service has a well-defined interface which
%xports some functionality that can be used by other services and/or users.
%his service abstraction helps group related data and logic together thereby
%aking long-term software maintenance more effective and cost efficient. 

The popularization of network computing and the World Wide Web (WWW) 
has led to the development and adoption of web services~\cite{6094008} as
the technology of choice for implementing modern service-oriented
(SOA~\cite{Haines:2010:SAM:1787234.1787269}) architectures.
The
interface portion of a web service, which abstracts and modularizes
its service implementation
details while making the service network-accessible, is commonly referred to
as a {\em web API}. As far as the users and applications that consume a 
web service
are concerned, the web API is the only point of contact and source of
functionality for the underlying service implementation.

Software engineering best practices separate the service implementation
and API implementation, both during development and maintenance.
The service implementation and API are integrated via 
a ``web service stack'' that implements functionality common to all web
services (message routing, request authentication, etc.)
Because the API is visible to external parties ({\em i.e.} clients of the
services), any changes to the API
impacts users and applications not under the immediate administrative control
of the software maintenance team.  For this reason API features 
usually undergo long
period of ``deprecation'' so that independent clients of the services can have
ample time to ``get ready'' for an API change.  At the same time,
technological innovations often prompt service reimplementation and/or 
upgrade to
achieve greater cost efficiencies, performance levels, etc.
Thus APIs typically have a more
slowly evolving and long lasting lifecycle than the the service
implementations
to which the interface. 

Cloud computing is based on the idea of exposing some digital asset or a
capability ({\em e.g.} compute power, database, etc.) 
as a highly scalable web service.  Mobile
devices, due to their limited hardware resources often offload much of their
processing and storage needs to remote services running in a ``cloud''
connected the Internet.  Web APIs
play a crucial role in both these paradigms. 

As a result, modern computing clouds, especially clouds implementing some form
of Platform as a Service (PaaS)~\cite{4548165}, have accelerated the
proliferation of
web APIs and their use.  Most PaaS
clouds~\cite{appscale13,cloudfoundry,openshift} include
features designed to
ease the development and hosting web APIs for scalable use over the Internet. 

In particular, API governance promotes code reuse among developers
since the API must be treated as tracked and controlled software entity.
It also ensures that software users benefit from change control since the APIs
they use
change in as controlled and nondisruptive manner.  From a maintenance
perspective, API governance 
makes it possible to enforce best practice coding procedures, 
naming conventions, and deployment procedures uniformly.
%\item enforce controlled and secured use of APIs
%\item provide API services with differentiated SLAs
%\end{itemize}
%%NOT HERE -- needs to move to the related work section
%The SOA community has done a lot of work in the area of service governance that can be adapted into
%API governance. Indeed, most of the commonly used service governance tactics such as versioning and policy enforcement
%can be used to assert some level of governance on web APIs as well.
API governance is also critical to API lifecycle
management --  the management of deployed APIs in response to new feature
requests, bug fixes, and organizational priorities. 
API ``churn'' that results from lifecycle management
is a common phenomenon and a growing
problem for web-based applications~\cite{portingeffort}.
Without proper governance systems to manage the constant evolution of APIs,
API providers run the risk of making their APIs unreliable while potentially
breaking downstream applications that depend on the APIs.

%\subsection{API Governance Models in the Cloud}

Unfortunately, most web frameworks used to develop and host web APIs do not 
provide API governance facilities. This missing functionality is
especially glaring
for cloud platforms that are focused on rapid
deployment of APIs at scale.   Commercial pressures frequently prioritize
deployment speed and scale over longer-term maintenance considerations only to
generate unanticipated future costs.

As a partial countermeasure, developers of cloud-based web services are 
frequently given
additional tasks associated with 
implementing custom {\em ad hoc} governance solutions using either locally
developed mechanisms or loosely integrated
third party API management services. 
These add-on governance
approaches often fall short in terms of their consistency and enforcement
capabilities since
by definition they have to operate outside the
cloud (either external to it or as a cloud-hosted application). 
As such, they do not have the end-to-end 
access to all the metadata and cloud-internal control mechanisms
that are necessary to implement strong governance at scale. 

%As an example, consider the case where a developer deploys a new version of an API. In this
%situation, the actual code changes take place in the cloud. An 
%external governance system cannot see the code changes and therefore cannot facilitate any deployment-time governance checks.
%It may be possible to perform the API deployment through the external governance system, using it
%as a proxy. But this approach leaves room for the developer to accidentally
%or even intentionally bypass the governance system and roll out the changes directly to the cloud. Hence, the external governance systems
%cannot ``enforce'' the required level of control.

%Finally, the service providers have to pay extra to use third party API governance solutions, and they require
%additional man power to be configured, maintained and monitored. This may offset some of the marginal cost benefits obtained
%by using a cloud platform as a deployment target for APIs.
