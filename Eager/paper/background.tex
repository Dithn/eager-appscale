In this section we will briefly review some of the concepts and technologies that underpin our vision for providing enforced governance to
web APIs deployed in the cloud. Our research both relies on and advances these already widely adopted and well-researched areas in
computer science.

\subsection{SOA, Web Services and Web APIs}
Service Oriented Architecture (SOA) is a popular architectural style for implementing complex systems in a modular fashion. In this approach, 
a system 
is implemented as a composition of a number of independent software components referred to as ``services''. Each service has a well-defined
interface which exports some functionality that can be used by other services and/or users. This service abstraction helps group related data
and logic together thereby making it easier to implement and maintain the system in the long run. SOA also naturally lends itself to implementing
systems using a wide range of programming languages, tools and platforms.

With the popularization of network computing and the World Wide Web (WWW) the web services technology has emerged as a prime choice 
for implementing systems based on SOA. A web service is a
software component with a well-defined interface that facilitates interoperable machine-to-machine interactions over a network. As it becomes 
evident from
this definition, web services can easily be adapted to play the role of  ``services'' in SOA. The interface portion of a web service, which
abstracts out its implementation details while making the service network-accessible, is commonly referred to as 
a web API. As far as the users and applications that consume a web service is considered, the web API is the only visible and
interesting part of a web service.

When considering the web service development process, the implementation portion of a web service is developed and maintained separately from
its API portion. The ``web service stack'' connects a web API to an implementation. Because the API is visible to external parties, any 
changes to the API immediately impacts many users and applications. Therefore APIs have a much slowly evolving and long lasting lifecycle than 
their implementation counterparts. 

\subsection{Cloud Computing and Web APIs}
The importance and the interest towards web APIs have recently skyrocketed due to the proliferation of cloud computing and mobile computing.
Cloud computing is based on the idea of exposing some digital asset or a capability (e.g. compute power) as a highly scalable web service. A 
web API determines
what functionalities are exported by the cloud, and how the users/applications would consume them. Mobile devices, due to their limited hardware
resources must offload most of their processing and storage needs to remote services on the Internet. Again, web APIs help with this offloading. 
Indeed, most of the popular social media and productivity apps
that are in use today rely on one or more web APIs hosted by their respective vendors and service providers (e.g. Twitter app, Facebook app and 
various geo/mapping apps).

On the other hand, modern computing clouds, especially PaaS clouds, have become fertile breeding grounds for many web services and hence 
web APIs. Most PaaS clouds make it simple to develop web services and host web APIs for scalable use over the Internet. Mobile 
Backend-as-a-Service (MBaaS) clouds take this one step further where they specialize in exposing web APIs for facilitating mobile
app development. As a result we see a significant boost in the number of web APIs hosted in the cloud.

\subsection{API Governance in the Cloud}
As more and more web APIs are developed and deployed in the cloud, the need for sophisticated API governance systems also continues to
increase. API governance is required to:
\begin{itemize}
\item promote and ensure code reuse among service and application developers
\item detect programming and deployment errors and take corrective action
\item change and update the APIs over time in a controlled and undisruptive manner
\item enforce proper coding habits, naming conventions and deployment practices
\item enforce controlled, secured and monitored usage of APIs
\item provide API services with competitive reliability, availability guarantees and SLAs
\end{itemize}

The traditional SOA community has done a lot of work in the area of service governance that can be easily extrapolated and adapted into
API governance. Indeed, most of the commonly used service governance tactics such as versioning, policy enforcement and and lifecycle
management are fully applicable to managing web APIs as well.

API governance can be particularly helpful in dealing with the API lifecycle management, i.e. changing the deployed APIs in response to new
feature requests, bug fixes and various other technical and commercial reasons. API churn is a very common phenomenon in the WWW. Without
proper governance systems to manage the constant evolution of APIs, API providers run the risk of making their API deployments unreliable
while potentially breaking many downstream applications that depend on the APIs that are being changed.

\subsection{Limitations of Existing API Governance Models in the Cloud}
Unfortunately, most web development frameworks used to develop and host web APIs do not provide easy means of API governance. This is
especially true in the case of cloud platforms (e.g. PaaS, MBaaS), which at the moment are focused entirely on facilitating easy development
and deployment of APIs, but not enforced governance. As a consequence, developers that use cloud platforms as deployment targets for web APIs, 
need to implement custom governance solutions or purchase third party API management services to fulfill the requirements listed above. But 
then again, these add-on governance
systems often fall short of supporting the required level of control and governance in the cloud, as they by definition have to operate outside of the
cloud. These external services do not have complete access to all the metadata and resources in the cloud to perform strong deployment-time and
run-time governance checks. 

As an example, consider the case where a developer attempts to deploy a new version of an existing API. In this
situation, the actual code and configuration changes take place in the cloud, which is the runtime or the hosting environment for the API. An 
external governance system cannot see the code changes that take place and therefore cannot facilitate any deployment-time governance checks.
However, it may be possible to perform the API deployment through the external governance system, using the external governance system
as a proxy or a gatekeeper. But this approach leaves room for the developer to accidentally
or even intentionally bypass the governance system and roll out the changes directly to the cloud. In other words, the external governance systems
cannot ``enforce'' the required level of governance, simply because they are external to the actual environment that hosts the API code.

Finally, the service providers have to pay extra to use third party API governance solutions, and they require
additional man power to be configured, maintained and monitored. This may unfortunately offset some of the marginal cost benefits obtained
by using a cloud platform as a deployment target for APIs.