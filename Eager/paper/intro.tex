The growth of the World Wide Web (WWW), web services, and cloud computing have
significantly influenced the way developers implement software applications.
Instead of implementing all the functionality from the scratch, developers
increasingly offload ({\em i.e.} as a ``mash-up'') as much application functionality 
as possible 
to remote,
web-accessible services, each of which exports its own application programming
interface (API).  Thus, a modern application often combines local program logic
with calls to APIs that interface to cloud-hosted services.
This approach significantly reduces both the programming and
the maintenance workload associated with the application.  In theory, because
the APIs interface to software that is curated by cloud providers, the
application leverages greater
reliability, scalability, performance, 
and availability in the implementations it calls upon through these APIs than
it would if those implementations were local to the application
(e.g. as locally available software libraries).
Moreover, by accessing common services, developers avoid ``re-inventing the
wheel'' each time they need a commonly available service and the scale at
which clouds operate ensures that these services have access to enough request
capacity.

As a result, web-accessible APIs and the software implementations to which
they provide access are rapidly proliferating.
At the time of this writing, 
ProgrammableWeb~\cite{pweb}, a popular web API index, lists more than $11,000$
publicly available
web APIs and a nearly 100\% annual growth rate~\cite{pweb_growth}.
These APIs increasingly employ the REST (Representational State Transfer) architectural style and 
many of them target commerce-oriented applications (e.g.
advertising, shopping, travel, etc.).
However, several non-commercial entities have also recently
published web 
APIs, e.g. IEEE~\cite{ieeeapis}, UC Berkeley~\cite{ucbapis}, and the US White
House~\cite{whitehouseapis}. 

This proliferation of web APIs demands new techniques and systems that
automate the maintenance and evolution of APIs as a first-class software
resource.  API management in the form of run-time mechanisms to implement
access control and performance-based service level agreements (SLAs) is not
new, and many good commercial offerings exist today~\cite{3scale,apigee,layer7}.   
However, support for \textit{API governance} -- consistent, generalized, policy
implementation across multiple separate APIs in an administrative domain --
is a new area of research made poignant by the emergence of cloud computing.

A lack of API governance can lead to 
security breaches, denial of service (DoS)
attacks, poor code reuse, violation of service-level agreements (SLAs), 
naming and branding issues, and abuse of digital 
assets by the API consumers. Unfortunately, most existing cloud platforms
within which web APIs are hosted provide only minimal governance support, {\em
e.g.}
registration and authorization.  These mechanisms
(available from various commercial vendors such as
3Scale~\cite{3scale}, Apigee~\cite{apigee},
Layer7~\cite{layer7})
are important to policy implementation since governance policies often need to
express access control specifications.  
However, developers are still responsible for implementing governance policies
that {\em combine} features such as API versioning, 
dependency management, and SLA enforcement as part of their respective
applications.  Moreover, each application must
implement its own governance -- there is no system for ensuring that the
cloud administrators set and enforce policy to which developers must conform.
%policies implemented by different developers are consistent both in their
%specification and their implementation.

In contrast API management solutions (which are typically implemented as
stand-alone services that are not integrated with the cloud) do attempt to
address some governance concerns beyond access control.  However, because they
are not integrated within the cloud provisioning environment, their function is
advisory and documentarian.  That is, they 
do not possess the ability to implement full enforcement and, instead, alert
operators to potential issues without preventing non-compliant behavior.
In addition, many of these systems operate outside the 
cloud that actually hosts the APIs thereby generating an additional cost that
may preclude their use.
As such, they can fail independently of the cloud, thereby affecting 
the scalability and availability of the software that they control.
Finally, because they are not integrated with the cloud itself it is difficult
for them to
implement governance at deployment-time -- the phase of the software lifecycle
during which an API change or a new API is being put into service.
Because of the scale at which clouds operate, deployment-time governance is
critical since it permits policy violations to be remediated before the
changes are put into production ({\em i.e.} before run-time).

Thus, our thesis is that governance must be implemented as a built-in, native
cloud service to
overcome these shortcomings.
That is, instead of an API management
approach that layers governance features on top of the cloud, 
we propose to provide API governance as a fundamental component of the cloud
platform.  Cloud-native governance abstractions
\begin{itemize}
\item enable both deployment-time and run-time enforcement of governance
policies as part of the cloud hosting functionality,
\item avoid inconsistencies and failure modes caused  
by integration and configuration of governance services that are not
end-to-end integrated within the cloud fabric itself, 
\item leverage already-present cloud functionality such as fault tolerance,
high availability, elasticity, and end-to-end security implementation to
facilitate governance, and
\item unify a vast diversity of API
governance features across all stages of the API lifecycle
(development, deployment, evolution, deprecation, retirement). 
\end{itemize}
As a native functionality, such an
approach also simplifies and automates API governance implementation for
the administrators or ``DevOps'' teams responsible for application
software deployment and
maintenance thereby allowing the separation of administrative and
cloud management concerns from development concerns.

To explore the efficacy of cloud-integrated API governance we have developed
an experimental cloud platform that supports governance policy specification
and enforcement for the applications it hosts. 
EAGER -- {\bf E}nforced {\bf A}PI {\bf G}overnance
{\bf E}ngine for {\bf R}EST -- is a model and an architecture that is designed
to be integrated within existing
cloud platforms in order to facilitate API governance as a 
cloud-native feature. EAGER enforces proper versioning of APIs and supports dependency 
management and comprehensive policy enforcement at API deployment-time. 

Using EAGER, we investigate the trade-offs between deployment-time policy
enforcement and run-time policy enforcement.
Deployment-time enforcement is attractive for several
reasons.  First, if run-time only API governance is implemented, 
policy violations will go undetected until the offending APIs are used,
possibly in a deep stack or call path in an application.  
As a result, it may be difficult or time consuming to pinpoint the specific
API and policy that are being violated (especially in a heavily loaded web
scalable web service).
In these settings, multiple deployments and rollbacks may occur before a policy
violation is triggered making it difficult or impossible to determine the root
cause of the violation.  By enforcing governance as much as possible
at deployment-time
EAGER implements ``fail fast'' in which violations are detected
immediately making diagnosis and remediation less complex.  
Further, from a maintenance perspective,  the overall
system is prevented from entering a non-compliant state, which aids in the
certification of regulatory compliance.  In addition, run-time governance
typically implies that each API call will be intercepted by a policy-checking engine
that uses admission control and an enforcement mechanism creating scalability
concerns.  Because deployment
events occur before the application is executing,
traffic need not be intercepted and checked ``in flight''
improving the
scaling properties of governed systems.  However, not all governance policies can be
implemented strictly at deployment-time.  As such, EAGER includes run-time
deployment facilities as well.  The goal of the research is to identify how to
implement enforced API governance most efficiently by combining deployment-time  
governance where possible and run-time governance where necessary.

%Thus while
%EAGER also implements run-time
%enforcement in a way similar to API management
%solutions~\cite{wso2am,apigee,layer7,3scale}, in this paper, we focus
%on its deployment-time governance features as a way of helping to meet the
%emerging challenges faced by those
%implementing scalable web service venues. 

EAGER implements policies governing the APIs that are 
deployed within a single administrative domain.  Focusing governance on
the APIs simplifies both policy specification and the consistent
and automatic implementation of policy enforcement.  At the same time, it promotes
software maintainability by separating the API lifecycle management from that
of the service implementations and the client users.  That is, APIs are often
longer lived than the individual clients that use them or the implementations
of the services that they represent.  At the same time they represent the
``gateway'' between software functionality consumption 
(API clients and users) and service
production (web service implementation).  Policy definition and enforcement at
the API level permits the service and client implementations to change
independently without the loss of governance control.

Enforced API governance further enhances software maintainability by guaranteeing that 
developers reuse existing APIs when possible to create new software artifacts
(to prevent API redundancy and unverified API use). Concurrently, it
tracks changes made by developers to deployed web APIs to prevent
any backwards-incompatible API changes from being put into production.

EAGER includes a language for specifying 
API governance policies.  The EAGER language is distinct from 
existing policy languages like WS-Policy~\cite{WSPolicy,soagovstandard}
in that it avoids the complexities of XML, 
and it incorporates a developer-friendly Python programming language syntax for 
specifying complex policy statements in a simple and 
intuitive manner. Moreover, we ensure that specifying the required policies 
is the only additional activity that API providers should perform in
order to benefit from EAGER. All other API governance related verification and 
enforcement work is carried out by the cloud platform automatically.

To evaluate the feasibility and performance of the proposed 
architecture, we prototype the EAGER concepts in an implementation
that extends AppScale~\cite{appscale13}, 
an open source
cloud platform that emulates Google App Engine. We describe
the implementation and integration as an investigation of
the generality of the approach.  By focusing on deployment actions and
run-time message checking, we believe that the integration methodology
will translate to other extant cloud platforms.

%show that the EAGER 
%architecture can be easily implemented in extant clouds with
%minimal changes to the underlying platform technology. 

We further show that 
EAGER API governance and policy enforcement impose a negligible 
overhead on the application deployment process, and the overhead
is linear in the number of APIs in the applications 
being validated.  
Finally, we show that EAGER is able to
scale to tens of thousands of deployed web APIs and hundreds of user 
defined governance policies.

In the sections that follow, we present some background on API governance
and overview the design and implementation of
EAGER. We then empirically evaluate EAGER using a wide range of APIs and
experiments.  Finally, we discuss related work, and conclude.
