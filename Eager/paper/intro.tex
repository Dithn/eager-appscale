The growth of the World Wide Web (WWW), web services, and cloud computing have significantly influenced the
way developers design and implement software applications. Instead of implementing all the required functionality from
the scratch or using standalone libraries, developers increasingly offload the
developement of as much functionality as possible to remote,
web-accessible application programming interfaces (web APIs) hosted ``in the
cloud'', thereby significantly reducing the 
programming workload. By doing so, developers also benefit indirectly from the efforts put in 
by the API and cloud providers to 
increase the reliability, scalability and availability of their respective service offerings. 

To facilitate wide spread use of their technologies by other programmers, development teams 
around the world expose their applications via web APIs.
As a result, today, we are experiencing a proliferation of web APIs. According to the 
statistics published by the popular web API index, ProgrammableWeb, the number of publicly available
web APIs has shown nearly 100\% annual growth rate, from 1000 APIs in 2008 to 11000 APIs in 2013. 
Interestingly, it took almost 3 years for ProgrammableWeb to accumulate that
first 1000 APIs. 

The growth and use of web APIs is not limited to the IT industry. 
Industries across sectors use the WWW as a medium for conducting their 
business and increasingly make their digital assets available via web APIs. 
As of March 2014, ProgrammableWeb lists 256 APIs for advertising, 
355 APIs for shopping, 227 APIs for travel and 224 APIs for education.
Recently non-commercial entities have started publishing web 
APIs, e.g. IEEE~\cite{ieeeapis}, UC Berkeley~\cite{ucbapis}, and the US White
House~\cite{whitehouseapis}.  Moreover, the 2012 US government memorandum 
on ``Building a 21st Century Digital Government'' directs all federal agencies to 
focus their IT strategy around creating public and open, cloud-hosted web APIs.

This proliferation of web APIs requires new techniques and systems for 
effectively managing and governing these artifacts.
The lack of governance can lead to 
security breaches, denial of service (DoS)
attacks, poor code reuse, violation of service-level agreements (SLAs), 
naming and branding issues, and abuse of digital 
assets by the API consumers. Unfortunately, most existing cloud platforms
within which web APIs are hosted provide only minimal governance, e.g.
API development authorization, API deployment checks and registration.
As a result, developers are responsible for API versioning, access control,
dependency management, and policy inforcement. Recent advance in 
API management (e.g. 3Scale~\cite{3scale}, Apigee~\cite{apigee},
Layer7~\cite{layer7}) have emerged to fill this gap. 
These third party solutions offer access control
via API keys (typically based on OAuth), access rate limiting, and 
several other useful API governance features. But such third party 
solutions have several limitations:
\begin{enumerate}
\item Third party API management solutions operate as services external to the target cloud that actually hosts the APIs. Therefore 
the API providers have to manage and potentially pay for an additional service, which increases the management overhead and overall cost.
\item Since the API management solutions are external to the target cloud, the ability to control and govern web APIs in an enforced manner is
lost. In other words, it is not possible to perform crucial design-time and deployment-time validations and governance checks for the APIs
that are being developed and rolled out into the cloud. 
\item The API management solutions fail independently of the target cloud, thereby affecting the reliability, scalability and availability of the 
cloud-hosted APIs. This implies that the API providers cannot take advantage of the potential scalability and availability benefits of using a cloud
platform as a deployment target.
\end{enumerate}

We postulate that a cloud platform can facilitate effective and 
enforced governance in a developer-friendly and cost effective manner by
making API governance and the related tooling an integrated component of the cloud itself. 
That is, instead of using a third party API management
solution that simply layers governance features on top of the cloud, 
we propose to provide API governance as a fundamental component of the cloud
platform.  By doing so, we are able (i) to preclude the need for management,
integration, and configuration of third party services, (ii) to take advantage
of the existing functionality in cloud platforms that are also required for 
API governance (fault tolerance, elasticity, security mechanisms), and (iii)
to unify a vast diversity of API
governance features and functionality across all stages of API lifetime
(development, deployment, evolution, deprecation, retirement). Such an
approach greatly simplifies and automates API governance which we believe
is necessary for wide spread use.

Toward this end, we propose EAGER (Enforced API Governance Engine for REST), 
a model and architecture that augment existing
cloud platforms that facilitates API governance as a 
cloud-native feature. EAGER enforces proper versioning of APIs and supports dependency 
management, access control, and comprehensive policy enforcement at API deployment time. 
Deployment-time enforcement is an initial step in a complete governance
solution and ensures that APIs put into production meet all expectations of
the developer and other stakeholders, and that they are ready for use
by potentially thousands of consumers over the lifetime of the API.
Our dependency management and policy enforcement also guarantees that 
developers reuse existing APIs when possible to create new software artifacts
(to prevent API redundancy and unverified API use).
EAGER also tracks changes made by developers to deployed web APIs and prevents
any backwards incompatible API changes from being put into production, 
thereby preventing downstream applications from breaking.

We augment our approach with a unifying language for specifying 
API governance policies. Our language is distinct from 
existing policy languages like WS-Policy~\cite{WSPolicy,soagovstandard} and
WS-Agreement~\cite{WSAgreement} in that it avoids the verbosity of XML, 
and it incorporates a developer-friendly Python programming language syntax for 
specifying even the most complex policy statements in a simple and 
intuitive manner. Moreover, we ensure that specifying the required policies 
is the only additional activity that API providers should perform in
order to benefit from EAGER. All other API governance related verification and 
enforcement work is carried out by the cloud platform automatically.

To evaluate the feasibility and performance of the proposed 
architecture, we implement EAGER as an extension to AppScale~\cite{appscale13}, 
an open source
cloud platform that emulates Google App Engine. We show that the EAGER 
architecture can be easily implemented in extant clouds with
minimal changes to the underlying platform technology. We further show that 
EAGER API governance and policy enforcement impose a negligible 
overhead on the application deployment process, and the overhead
is linear in the number of APIs present in the application being validated.  
Finally, we show that EAGER is able to
scale to tens of thousands of deployed web APIs and hundreds of user 
defined governance policies.

In the sections that follow, we present some background on API governance
and overview the design and implementation of
EAGER. We then empirically evaluate EAGER using a wide range of APIs and
experiments.  Finally, we discuss future and related work, and conclude.


