Web services and cloud computing have revolutionized the way software is
developed, deployed, and consumed.  As a consequence, there has been a
proliferation of web services, which developers make accessible to users via
web application programming interfaces (web APIs) and cloud-based deployment
technologies.  Because this model significantly simplifies and expedites
deployment of web APIs, it also poses new software maintenance and evolution
challenges.  In particular, it is difficult to track, control, and compel
reuse of web APIs, inadequately tested programs can be
deployed into production, and API changes can be introduced that
break API-user code or that breach security or organizational procedures.

To address these challenges, we investigate a new approach to API governance
-- combined policy, implementation, and deployment control of APIs for
software and data deployed as web services.  Our approach, called EAGER,
provides a software architecture that can be easily integrated into cloud
platforms as cloud-native features, and that provides system wide, d
eployment-time enforcement of API governance policies.
Specifically, EAGER can check for and
prevent backward incompatible API changes from
being deployed into production,
enforces service reuse, and facilitates enforcement of other best practices
in software maintenance via policies.  We implement EAGER by
extending an open source platform-as-a-service cloud and
demonstrate its feasibility, efficiency,
scalability, and effectiveness for enforcing cloud-based API governance.

\ignore{
Web services and cloud computing have
revolutionized the way software is developed and maintained. Increasingly,
cloud-based applications take the form of network-accessible services that export
web application programming interfaces (web APIs).  Application
clients and users compose the functionality they require by making calls
to these published interfaces. 

As cloud deployments scale, the effective use of this flexible programming
architecture
poses new software maintenance and
evolution challenges.  Especially when the web services themselves
make use of APIs exported by other services, the problem of governance -- policy
design and implementation that ensures predictable behavior of the overall
system -- becomes critical.
Without reliable and consistent API governance,
services, applications, and users, must manage explicitly 
software maintenance issues such as poorly developed
code going into production unchecked, API changes breaking existing 
applications, and API dependency management.

In this paper, we describe EAGER -- a model and 
software architecture that extends existing cloud platforms
to provide enforced API governance at a fundamental level. Using policies
specified
by the cloud maintenance team, EAGER can check for and
prevent backward incompatible API changes from being deployed into the cloud. EAGER
also enforces code reuse and a variety of other software maintenance best practices
via its comprehensive policy enforcement support. We detail EAGER's 
functionality using an open source
Platform-as-a-Service (PaaS) cloud as an example service venue
and demonstrate its feasibility, efficiency, scalability and
effectiveness as an API governance system for clouds.
}
