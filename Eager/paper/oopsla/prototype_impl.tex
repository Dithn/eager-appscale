We implemented a prototype of Cerebro that works with Google App Engine and AppScale cloud platforms.
The prototype analyzes web APIs implemented in Java, and makes SLA predictions for them.
%In this section we present the details of our prototype.

\subsection{Static Analyzer}
We implemented the static analyzer component of Cerebro using Soot~\cite{Vallee-Rai:2010:SJB:1925805.1925818}, a powerful
static analysis and code manipulation framework for Java. It supports performing a wide range of analyses on both Java source
code and byte code. Soot converts the input Java code into one of four intermediate representations (Baf, Jimple,
 Shimple and Grimp), and constructs the CFG for the code. It provides a number of common static
 analyses as built-in utilities, and also provides a set of programming APIs to enable developing custom
 analyses.
 
Our prototype static analyzer takes the compiled byte code of web applications as input. It uses the
Soot framework to transform byte code into Jimple, and from that it constructs the CFGs for 
all the Java methods in the exposed web API classes. Currently, our
prototype treats following classes as exposed web APIs:

\begin{itemize}
\item classes that extend the \textit{javax.servlet.HttpServlet} class (i.e. Java servlet implementations)
\item classes that bare the JAXRS \textit{@Path} annotation (i.e. JAXRS API implementations)
\item any other classes explicitly specified by the developer in a special configuration file
\end{itemize}
 
Our prototype then proceeds to analyze each of the public methods in the web API classes
using the algorithm described earlier. Each public method is considered a separate operation
in the web API, and hence given a separate execution time prediction.
All cloud SDK operations supported by Google App Engine
are grouped under the Java package name \textit{com.google.appengine.apis}. Whenever we detect
a function call node in a CFG, we check whether the call target belongs to the above package.
This way we can efficiently identify the cloud SDK invocations in the code.

We also use Soot's built-in loop extraction analysis
to detect loops in the code. In cases where loops contain cloud SDK calls, we attempt to estimate
the loop bounds using a loop bound prediction algorithm based on abstract interpretation~\cite{bygde2010static}. This
approach yields successful results with regard to data-independent loops. In case of data-dependent
loops (which is the more common case), we simply prompt the API developer to specify the loop iteration
count. We also support a non-interactive mode, where the developer can provide the maximum number of
records in the datastore (iterations) as a startup parameter to Cerebro.%, in which case the static analyzer will simply use the
%provided input value to annotate the cloud SDK calls in data-dependent loops.

\subsection{Watchtower}
We implemented Watchtower as a Java web application that can be deployed on Google App Engine and
AppScale. It is capable of benchmarking a wide range of cloud SDK operations
-- primarily the ones related to datastore and memcache. All benchmarking results are timestamped and 
saved in the datastore. Our implementation of Watchtower also benchmarks the iterative datastore read operation
with result set sizes 10, 100 and 1000. 

This implementation makes use of the job scheduling capabilities provided by Google App
Engine and AppScale. When deployed, it starts a cronjob in the cloud platform to invoke the benchmarking code
once every 60 seconds. Our implementation also exposes a set of REST APIs, which
are used by other components of Cerebro to retrieve the collected benchmarking data as time series.

\subsection{QBETS and SLA Predictor}
We use an implementation of QBETS written in C. We wrap this C program in a little bit of Go code, so some 
of its features are exposed as a web API. This web API is used by the static analyzer to pass in the extracted
cloud SDK invocation sequences, and other input parameters. We also have code for contacting Watchtower 
to retrieve the time series data required to make predictions, aggregating the time series, and then 
starting the QBETS analysis. 

\subsection{Using the Prototype}
To start using our prototype, one must first deploy the Watchtower application on the target cloud platform. 
Ideally Watchtower should be allowed to run for a several hours,
so that it can collect enough time series data for the QBETS analysis. Then,
an API developer who has implemented a web API for the cloud can build his/her code into a deployable
artifact (typically a .war file), and submit it to the static analyzer of Cerebro, along with the desired percentile
and upper confidence values (if not provided our prototype defaults to $p=95$ and $c=0.05$). This kicks off
the process for predicting SLAs for web API operations in the input web application.
%The static analyzer will extract
%the cloud SDK invocations from the web API code, and pass them along to the SLA predictor. The predictor
%will retrieve the time series data for cloud SDK operations from Watchtower, run QBETS and return the
%generated prediction values.