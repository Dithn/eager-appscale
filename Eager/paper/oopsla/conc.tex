\begin{itemize}
\item We tested Cerebro for prediction accuracy using a wide range of web applications. In these tests Cerebro was used to
predict the 95th percentile of the API execution time. In all cases, Cerebro succeeded in achieving a percentage
accuracy level close to or higher than 95\%. That is, the actual API execution times measured from the 
applications were below the predicted 95th percentile values 95\% or more of the time.
\item For most of the test APIs, Cerebro managed to make tight predictions (i.e the predictions are not too far
from the actual values). In situations where the web APIs have highly variant performance characteristics with many high outliers,
we saw that Cerebro trades off tightness for accuracy, by making conservative predictions.
\item We observed that Cerebro takes some time to learn from the gathered time series data before it can make highly reliable
predictions. In our tests this learning period took somewhere from 150 to 200 minutes. The percentage accuracy undergoes
many large fluctuations during this period, but once Cerebro's prediction algorithm has converged, it produces highly
accurate and reliable results consistently. Only very minor fluctuations in percentage accuracy can be observed after
convergence.
\item We introduced a theoretical model for determining when to treat a predicted SLA as invalid, and used that model
to compute the prediction validity periods for Cerebro. We noticed that on average the prediction validity period
can fall somewhere between 24 and 72 hours. Further, we observed that 95\% of the time the validity period is at least 1.41 hours
on App Engine, and 1.95 hours on AppScale.
\item When comparing the App Engine results to AppScale results we saw that Cerebro produces tighter and
more long lasting SLA predictions on AppScale. We attributed this behavior to the fact that AppScale is a small and controlled
environment that is much more stable over time. %In contrast Google App Engine is much larger, shared among many users
%and provides no control over the underlying hardware resources.
\end{itemize}