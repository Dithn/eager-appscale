Cloud computing has proved to be a successful model for hosting web-facing
applications that are invoked by their users as services.  While clouds
currently offer Service Level Agreements (SLAs) containing guarantees of
availability they currently do not provide the ability to make performance
guarantees.

In this work we present Cerebro -- a system for establishing statistical
guarantees of application response time in cloud settings.  Cerebro combines
off-line static analysis of application control structure with on-line cloud
performance monitoring and statistical forecasting to predict bounds on the
response time of web-facing application programming interfaces (APIs).
Because Cerebro does not require application instrumentation or 
per-application cloud benchmarking, it does not impose runtime overhead
and hence is suitable for use at cloud scales.  Also, because the bounds are
statistical, they are appopriate for use as the basis of SLAs between
cloud-hosted applications and their users.

We investigate the correctness of Cerebro predictions, the tightness of their
bounds, and the duration over which the bounds persist in both Google App
Engine and AppScale (public and private cloud platforms respectively).  
We also detail the time necessary for the forecasting 
methodology to ``learn'' response-time bounds in these cloud settings. 


