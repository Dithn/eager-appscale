Our research is strongly rooted in a number of mature research areas in computer 
science and mathematics. These areas include static program analysis, cloud computing, time series analysis
and SOA governance.

The problem of predicting execution time SLAs of web APIs has a strong likeness to worst-case execution
time (WCET) analysis~\cite{Wilhelm:2008:WEP:1347375.1347389}. The objective of WCET analysis is to determine the maximum execution time of a software component in a given hardware platform. It is 
typically discussed in the context of real-time systems, where the developers should be able to document
and enforce precise hard real-time constraints on the execution time of programs. In order to save time, 
manpower and hardware resources, WCET analysis solutions are generally designed favoring static
analysis methods over software testing. We share similar concerns with regard to cloud platforms,
and strive to eliminate software testing in the favor of static analysis in our work. 

Ermedahl et al designed and implemented SWEET~\cite{ermedahl2007loop}, a WCET analysis tool that make use of program slicing~\cite{Sandberg:2006:FWF:1134650.1134666},
 abstract interpretation~\cite{Cousot:1977:AIU:512950.512973} and invariant analysis~\cite{Muchnick:1998:ACD:286076} to determine the loop bounds and worst-case execution time 
 of a program. Program slicing helps to reduce the amount of code and program states that need to be 
 analyzed by SWEET. This is somewhat similar to our idea of extracting just the cloud SDK invocations form 
 a given web API code. SWEET uses abstract interpretation in interval and congruence domains to identify
 the set of values that can be assigned to key control variables of a program. These sets are then
 used to calculate exact loop bounds for most data-independent loops in the code. Invariant analysis  is
 used to detect variables that do not change during the course of a loop iteration, and remove them from
 the analysis thus further simplifying the loop bound estimation. Lokuceijewski et al proposes 
 a similar WCET analysis using program slicing and abstract interpretation~\cite{Lokuciejewski:2009:FPS:1545006.1545064}. They additionally use a technique
called polytope models to further speed up the analysis.
 
The corpus of research that covers the use of static analysis methods to estimate the execution 
time of software applications is massive. Gulwani, Jain and Koskinen used two techniques named control-flow
 refinement and progress invariants to estimate the bounds for procedures with nested and multi-path loops~\cite{Gulwani:2009:CRP:1542476.1542518}.
 Gulwani, Mehra and Chilimbi proposed SPEED~\cite{Gulwani:2009:SPE:1480881.1480898}, a system that computes symbolic bounds for programs. This
 system makes use of user-defined quantitive functions to predict the bounds for loops iterating over
data structures like lists, trees and vectors. Our idea of using user-defined values to bound
 data-dependent loops (e.g. iterative datastore reads) is partly inspired from this concept.
 Bygde~\cite{bygde2010static} proposed a set of algorithms for predicting data-independent loops using abstract interpretation
 and element counting (a technique that was partly used in \cite{ermedahl2007loop}). We incorporated some
 of these algorithms into our prototype of Cerebro, due to their simplicity. Frost et al developed TetaJ, a WCET
 tool for Java byte code programs~\cite{Frost:2011:WAJ:2043910.2043916}.
 
Cerebro has a strong resemblance to some of the execution time analysis systems discussed above.
However, there are also several key differences. For instance, Cerebro is focused on solving the
execution time prediction problem for cloud applications. As seen from our survey results, applications 
developed for PaaS clouds have a set of unique properties, that can be used to greatly simply the
static analysis on them. Also, Cerebro is designed to only work with web API codes. This makes designing 
a solution much more simpler compared to having to design a system that analyzes arbitrary program codes.
In order to handle the constantly changing and evolving nature of cloud platforms, we have designed
Cerebro using a hybrid approach of static analysis and runtime monitoring of cloud SDK operations. This is
a significant depart from the above systems, which are entirely based on static analysis methods. Finally,
we use a time series analysis method~\cite{Nurmi:2007:QQB:1791551.1791556} to predict API execution time upper bounds with specific confidence
levels, which makes it possible to use the results as precise SLAs.

SLA management on service-oriented systems and cloud systems has also been throughly researched
over the years. However, a lot of the existing work has focused on issues like SLA monitoring~\cite{Michlmayr:2009:CQM:1657755.1657756,Tripathy:2011:MMS:1980822.1980832,Raimondi:2008:EOM:1453101.1453125,Bertolino:2007:SUS:1294904.1294914}, SLA negotiation~\cite{Mahbub:2011:PSN:2061042.2062022,Yaqub:2014:ONS:2680847.2681496,6546098}, and SLA modeling~\cite{Chau:2008:ASM:1463788.1463802,Stamou:2013:SGM:2516588.2516592,Skene:2004:PSL:998675.999422}. 
Some work has also looked at incorporating a given SLA to the design of a system, and
then monitoring it at the runtime to ensure SLA compliant behavior~\cite{He:2013:TSC:2532443.2532449}. 
Our research takes a
different approach from such works, whereby it attempts to predict the performance SLAs that can be 
supported by a given web API implementation. To the best of our knowledge, Cerebro is the first system to predict performance SLAs for web APIs developed for PaaS clouds.

A work that closely resembles ours have been proposed by Ardagna, Damiani and Sagbo~\cite{6649675}. They develop
a system for early estimation of service performance based on simulations. Given a STS
model (Symbolic Transitions System) of a service, their system is able to generate a simulation script, which
can be used to assess the performance of the service. STS models are a type of finite state automata. 
Further, they use probabilistic distributions with fixed parameters
to represent the delays incurred by various operations in the service. Cerebro is easier to use than this
system because we do not require API developers to construct any models of the web APIs. They only
need to provide the source code of the API implementations. Also, instead of using probabilistic distributions
with fixed parameters, Cerebro uses actual historical performance metrics of cloud SDK operations. This enables
Cerebro to generate more accurate results, that reflect the dynamic nature of the cloud platform better.