In this section we describe the experimental results we have obtained using Cerebro and the associated tools. We conduct
tests using five sample applications. 

\begin{description}
\item[StudentInfo] A JAXRS application for managing students of a class. Provides APIs for
adding/removing students, and listing the student information.
\item[ServerHealth] An application that monitors a given web URL for server uptime. Provides an
API for obtaining the uptime statistics computed by the application.
\item[SocialMapper] A simple social networking application. Provides APIs for adding users,
comments and other resources.
\item[StockTrader] A stock trading application. Provides APIs for adding users, registering
companies, buying and selling stocks among users.
\item[Rooms] A hotel booking application. Provides APIs for registering hotels and looking up
available rooms.
\end{description}

We instrument each of the above applications to measure the time taken by their web APIs to execute the
enclosed code. This is done by adding some extra Java code to each of the web APIs exposed by the sample applications.
We ensure that the instrumentation does not alter the original web API code by anyway (i.e. the original algorithms, control flow
and data flow are not impacted by the instrumentation). Then for each application we also
introduce a mechanism to output the measured execution times, so an external client can query and collect the execution
times of the web APIs.

We carry out each of our tests in two separate environments -- Google App Engine public cloud, and
the AppScale private cloud. The AppScale private cloud used for testing was powered by four m3.2xlarge virtual machines 
running on a private Eucalyptus cluster.

In the first set of experiments, we benchmark each application for a period of 18 to 20 hours.
During this time, we run an HTTP client on a separate machine that invokes the instrumented web APIs once every minute, 
and collects the server-side execution times (as measured by the instrumentation code), and records them for later processing.
At the same time, we also run the Watchtower application in the same target cloud environment as the test application to
benchmark the individual cloud SDK operations. Therefore at the end of a benchmarking run, we have two sets of data at hand:

\begin{itemize}
\item Web API execution times collected by the benchmarking HTTP client
\item Cloud SDK benchmarking data gathered by Watchtower
\end{itemize}

We now run Cerebro and make predictions about the web API execution time using the cloud SDK benchmarking data
collected by Watchtower. We configure Cerebro to predict an upper bound for the 95th percentile of the web API
execution time, with an upper confidence of 0.01. Cerebro generates a sequence of execution time predictions -- one 
per minute. Then we compare these predictions against the actual web API execution times collected during the same
time period. More specifically, we check whether each prediction is correct (i.e. prediction is equal to higher than the
actual value), and compute a running tally of percentage accuracy.
