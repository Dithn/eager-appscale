Cloud computing is an attractive model for deploying web services in
a highly scalable manner. Users access such
cloud-hosted services via their web-facing application programming
interfaces (APIs). Prior work has shown that it is possible to use a combined
approach of static analysis and cloud platform monitoring to predict the response
time upper bounds of such web APIs. This technique can be employed to
automatically generate service level agreements (SLAs) concerning the
performance of cloud-hosted web APIs.

In this work, we explore the validity period of auto-generated SLAs in
cloud settings. We discuss a simple model by which API consumers
can establish a response time SLA with the cloud platform, and renegotiate it when/if 
the SLA becomes invalid due to the dynamic nature of the cloud.
Using empirical methods and simulations on a real world
public cloud platform, we show that it is possible to auto-generate
highly durable response time SLAs for cloud-hosted web APIs, thereby
keeping the number of SLA invalidations and renegotiations very low, over
long periods.
