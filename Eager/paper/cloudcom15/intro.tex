Over the years web services have become an essential technology for implementing
complex distributed applications in a simple and modular fashion. Many web applications, cloud services and mobile
applications today, depend on one or more remote web services. More specifically, such
applications heavily rely on the web application programming interfaces (web APIs) through
which the remote web services are exposed to the Internet.

Developing applications using existing web services (and APIs) has several benefits. From an
architectural point of view, web APIs promote modularity, loose coupling and separation of concerns. From
an implementation point of view, they expedite development sprints, reduce code duplication and simplify
long term maintenance. These benefits have propelled web APIs to extreme popularity,
as evidenced by the explosive growth of web APIs on the Internet. 

However, reusing existing services also poses several challenges. In particular, web services
impact the correctness, performance, and availability of the
applications that depend on them. Also, web services change over time while their APIs remain 
stable, unbeknownst to the developers that use them (an artifact of loose coupling).
Unfortunately, up until now there has been a shortage of tools to help developers 
reason about web API dependencies throughout an application's 
lifecycle (i.e. development, deployment, and runtime).  Without such tools, 
programmers had to resort to extensive testing and profiling 
to understand the performance impact of web APIs on their applications.

To address these issues, we have developed Cerebro, a new approach that
predicts the response time of web APIs hosted in Platform-as-a-Service (PaaS) clouds.
Cerebro enables the PaaS administrator to determine what response time service level 
agreement (SLA) can be fulfilled by each web API operation exported by the applications
hosted in the PaaS. 

Applications developed for PaaS clouds rely on a collection of
scalable services offered by the underlying cloud platform. These services are exposed to
the application through a cloud-specific software development kit (cloud SDK). Cerebro
employs static analysis to extract the sequence of cloud SDK operations invoked by a
given PaaS-hosted application, and combines that information with monitoring
data pertaining to those cloud SDK operations to predict an upper bound for the response
time of the application. This approach is fast, does not require any testing or profiling
of the application, and can be used to predict performance SLAs for a web API before
it is even deployed in the target PaaS cloud.

Our previous work has shown that the response time SLAs predicted by Cerebro are
\begin{itemize}
\item Accurate: Predicted upper bounds exceed the actual response time of the APIs
with a specific, configurable success rate.
\item Tight: Predicted upper bounds do not exceed the actual response time of the APIs by
a large margin (i.e. the predicted values are sufficiently close to the actual values).
\item Durable: Predicted SLAs remain valid for a sufficiently long period, before they
become obsolete due to the changes that occur in the cloud platform.
\end{itemize}

In this work we further investigate the durability of the SLAs predicted by Cerebro, 
as it applies to the developers who create applications using the cloud-hosted web APIs.
We shall refer to such application developers as API consumers. We
assume a simple model, where Cerebro gives each API consumer an initial response
time SLA, when he/she first subscribes to an API. A monitoring agent built into
Cerebro then keeps
track of the performance changes that occur in the cloud. When it detects that the cloud
can no longer support the previously negotiated SLA (i.e. the SLA has been
violated or become obsolete), it prompts the API consumer, and renegotiates for a new
SLA.

We use a combination of empirical testing and simulations to analyze how the SLA values
predicted by Cerebro change over time, and how they impact API consumers. We
conduct all our experiments in the Google App Engine public cloud environment 
using a set of open source Java web applications. Our results indicate that on average, the
SLAs predicted by Cerebro do not require renegotiation for at least 12 days. We also find that
over a period of 112 days, the maximum number of times any API consumer have to
renegotiate a given SLA is only 6. 

Sometimes when Cerebro prompts an API consumer to
renegotiate an SLA, the predicted new SLA value is very close to the obsolete old SLA value. 
Such renegotiations are not useful, and only serve to increase the renegotiation overhead
for API consumers. We
propose a threshold-based mechanism to keep such instances to a minimum, and show that
with this optimization in place, we can further elongate the SLA validity periods, and reduce
the maximum number of per API consumer renegotiations to 5. 