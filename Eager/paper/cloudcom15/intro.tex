Web services today are an essential technology for implementing
complex distributed applications.  To promote modularity and reuse, 
and to leverage the testing, deployment and scaling support, and maintenance 
provided by others, developers increasingly integrate remote, Internet-accessible web services 
via web application programming interfaces (web APIs)
into their web, cloud, and mobile applications.  
The benefits to programmer productivity that such development practices
permit have resulted in a vast number and diversity of available web APIs~\cite{pweb}.

Unfortunately, reusing existing services has its drawbacks. In particular, 
web APIs can impact the performance, behavior, and reliability of the applications
that integrate them.  Moreover, web service functionality and performance can change over time 
(even when the API remains the same) without notice.
Moreover, there is a shortage of tools that help developers 
reason about the impact of web API dependencies throughout an application's 
lifecycle (i.e. development, deployment, and runtime).  Without such tools, 
programmers must resort to extensive and continuous testing and profiling 
to understand the impact of their use of web APIs on the performance and behavior
of their applications.

To address this need, we have developed Cerebro~\cite{cerebro-soccsub15}.
Cerebro is a cloud platform-as-a-service (PaaS) technology that
predicts the response time of web APIs hosted by the PaaS.
Cerebro enables the PaaS administrator to determine response time service level 
agreements (SLAs) that applications hosted in the PaaS can guarantee.  Cerebro is able to 
do so since PaaS applications spend a majority of their time accessing PaaS services
(e.g. for persistent storage, data mangement, caching, etc.). Specifically, 
Cerebro uses static analysis to extract the series of calls to PaaS services
in an operation and uses them to estimate an upper bound on the execution time of 
an operation.  Cerebro does so via statistical distributions of the performance of PaaS services
in the platform that it collects via lightweight, application-independent monitoring.
In this prior work, we find that Cerebro is fast, does not require modification,
testing, or profiling of the applications themselves, and can be used to predict 
performance SLAs for web APIs at \textit{deployment time}.
We also show that Cerebro response time SLAs 
(i) are accurate -- they predict upper bounds that exceed the actual response time of the APIs
with a specific, configurable success rate, and (ii)
they are tight -- they predict upper bounds that do not exceed the actual 
response time of the APIs by a large 
margin (i.e. the predicted values are sufficiently close to the actual values). 

In this paper, we focus on the \textit{duration} that Cerebro predictions hold, i.e. the period
of time that predicted response time SLAs remain valid.  Predictions are invalidated
when conditions in the the platform change in ways that violate the SLA (beyond random
chance).  Such changes can result from congestion (multitenency), resource availability, 
and modifications to the implementations of the underlying PaaS services. Invalidations 
require that SLAs be renegotiated and thus impose overhead on developers that use the 
web APIs exported by applications hosted by the PaaS.

We refer to such developers as AIP consumers.  Cerebro gives each API consumer an initial 
response time SLA when they register for an API key to use an application exported by the PaaS.
As mentioned above, a Cerebro monitoring agent tracks of performance of the PaaS services 
continuously.  When Cerebro detects that a change has occurred in the PaaS that
violates a previously negotiated SLA, Cerebro contacts the API consumer of hosted application
for which the SLA is violated and renegotiates a new SLA.

To analyze how SLA values predicted by Cerebro change over time and how API consumers are 
impacted by such changes, we perform extensive testing and emprical evaluation
of Google's public PaaS cloud App Engine and a set of open source Java web applications.  
We also use simulation to explore different options for SLA renegotiation.
Our results indicate that on average, the minimum duration for which Cerebro SLAs 
remain valid for this PaaS is 12 days.  We also find that
over a period of 112 days, the maximum number of times that any API consumer must renegotiate
their SLA is six.  We also find that in some cases Cerebro prompts an API consumer to
renegotiate an SLA when the predicted new SLA value is very close to the invalidated SLA value. 
Such renegotiations are not useful in practice and only serve to increase the renegotiation overhead
for API consumers. We thus also present a threshold-based mechanism that both reduces 
the number of required renegotiations and extends SLA validity duration significantly.
