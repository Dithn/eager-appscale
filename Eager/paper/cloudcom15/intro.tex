Over the years web services have become a foundational technology that enables implementing
complex applications in a simple and modular fashion. Many web applications, cloud services and mobile
applications in popular use today, depend on one or more remote web services. More specifically, such
applications heavily rely on the web application programming interfaces (web APIs) through
which the remote web services are exposed to the Internet.

Developing applications using existing web services (and APIs) has several benefits. From an
architect's point of view, web APIs promote modularity, loose coupling and separation of concerns. From
a programmer's point of view, they expedite development sprints, reduce code duplication and simplify
long term maintenance. These benefits have propelled web APIs to high popularity,
as evidenced by the explosive growth of web APIs on the Internet. 

However, reusing existing services also poses several challenges. In particular, web services
impact the correctness, performance, and availability of the
applications that depend on them. Also, web services change over time while their APIs remain 
stable, unbeknownst to the developers that use them (an artifact of loose coupling).
Unfortunately, up until now there has been a shortage of tools to help developers 
reason about these dependencies throughout an application's 
lifecycle (i.e. development, deployment, and runtime).  Without such tools, 
programmers had to resort to extensive testing and profiling 
to understand the performance impact of web APIs on their applications.