Web APIs impact the correctness and performance of the applications that depend on them.
However, most web APIs do not provide SLAs with regard to such properties, making it
difficult to reason about the performance of the applications that use web APIs. Cerebro attempts
to solve this problem for web APIs developed for PaaS clouds, by automatically predicting
response-time SLAs for them. 
%Our prior work has shown that Cerebro predicted SLAs are
%both accurate and tight.

In this work we analyze the validity period of the SLAs predicted by Cerebro. We present a simple
SLA negotiation model, and a conservative statistical approach for detecting when a predicted SLA has become
invalid. We evaluate our methods on the Google App Engine public cloud using empirical testing
and simulations. We find that on App Engine, Cerebro predictions are valid for at least 12 days
on average. We also find that API consumers do not have to renegotiate SLAs often, and the maximum
number of times an API consumer must renegotiate an SLA over a period of 112 days is six. We
also present and evaluate a threshold-based mechanism to eliminate the SLA renegotiations where
the old and new SLA values are very close to each other. This optimization further increases
the average SLA validity period, and reduces the number of SLA renegotiations per API consumer.
Overall, this work shows that automatic definition of response-time SLAs for web APIs is practically
viable in real world cloud settings, and API consumer timeframes. 
