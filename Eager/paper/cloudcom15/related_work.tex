SLA management on service-oriented systems and cloud systems has been 
throughly researched over the years. However, a lot of the existing work has focused on issues 
such as SLA monitoring~\cite{Michlmayr:2009:CQM:1657755.1657756,Tripathy:2011:MMS:1980822.1980832,Raimondi:2008:EOM:1453101.1453125,Bertolino:2007:SUS:1294904.1294914} and SLA modeling~\cite{Chau:2008:ASM:1463788.1463802,Stamou:2013:SGM:2516588.2516592,Skene:2004:PSL:998675.999422}. 

In our work, we predict the SLAs that can be supported on a given web API by using a combination of
static analysis, cloud platform monitoring and time series analysis. Cerebro continuously monitors the cloud platform,
and periodically re-evaluates the response time SLAs of web APIs -- an idea that has a lot in common with existing SLA
monitoring techniques. Upon detecting an SLA violation, we prompt the API consumer, and negotiate a new
SLA. Cerebro can use existing SLA negotiation methods~\cite{Mahbub:2011:PSN:2061042.2062022,Yaqub:2014:ONS:2680847.2681496,6546098} to carry out the actual communication between the cloud platform and the API consumer. 

There has also been prior work in the area of predicting 
SLA violations~\cite{Leitner10,6976585,Duan:2006:PIP:1142473.1142582}. 
These systems take an existing SLA and historical performance data of a service, and predict when the 
service might violate the given SLA in the future. 
Cerebro's notion of prediction validity period has some commonalities with this concept. In fact, Cerebro
can make use of such a method to determine the frequency at which it should re-evaluate the predicted
SLAs.