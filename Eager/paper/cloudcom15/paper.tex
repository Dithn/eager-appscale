\documentclass[compsoc]{IEEEtran}
\hyphenation{op-tical net-works semi-conduc-tor}

\begin{document}
%
% paper title
% can use linebreaks \\ within to get better formatting as desired
\title{Bare Demo of IEEEtran.cls for Conferences}

% author names and affiliations
% use a multiple column layout for up to three different
% affiliations
\author{
\IEEEauthorblockN{Hiranya Jayathilaka, Chandra Krintz, Rich Wolski}\\
\IEEEauthorblockA{Department of Computer Science
UC Santa Barbara\\
Santa Barbara, CA, USA\\
Email: \{hiranya,ckrintz,rich\}@cs.ucsb.edu}
}

% make the title area
\maketitle


\begin{abstract}
%\boldmath
The abstract goes here.
\end{abstract}

\section{Introduction}
This demo file is intended to serve as a ``starter file''
for IEEE conference papers produced under \LaTeX\ using
IEEEtran.cls version 1.7 and later.
% You must have at least 2 lines in the paragraph with the drop letter
% (should never be an issue)
I wish you the best of success.


\section{Conclusion}
The conclusion goes here.

\section*{Acknowledgment}
The authors would like to thank...


\begin{thebibliography}{1}

\bibitem{IEEEhowto:kopka}
H.~Kopka and P.~W. Daly, \emph{A Guide to \LaTeX}, 3rd~ed.\hskip 1em plus
  0.5em minus 0.4em\relax Harlow, England: Addison-Wesley, 1999.

\end{thebibliography}

\end{document}