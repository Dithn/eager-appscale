\documentclass[10pt]{article}
\usepackage{graphicx}
\DeclareGraphicsExtensions{.pdf,.png}
\graphicspath{{./figures/}}
\usepackage{url}
\usepackage{epsfig} 
\usepackage{amssymb}
\usepackage{amsmath}
\usepackage{amsfonts}
\usepackage{floatflt}
\setcounter{secnumdepth}{5}
\setcounter{tocdepth}{5}


\setlength{\oddsidemargin}{0in}   % 10pt is 1.875 for margins in latex
\setlength{\evensidemargin}{0in}
\setlength{\textwidth}{6.5in}
\setlength{\textheight}{9in}
\setlength{\topmargin}{-0.5in}

\newcommand{\ignore}[1]{}

\thispagestyle{empty}
\begin{document}
\date{}

\title{Governance of Web APIs in Modern Cloud Computing Environments}

\author{Hiranya Jayathilaka\\
Department of Computer Science\\
Univ. of California, Santa Barbara, CA\\
Email: hiranya@cs.ucsb.edu
}
\maketitle

\section{Introduction}
Cloud computing is fostering a model of 
application development that combines authored code with functionality
provided by extant, curated web services.  Consumer-targeting applications 
(particularly those designed for mobile platforms and the web) interact with highly
scalable and reliable ``back-end'' web services that are constantly
maintained in well-connected, secure data centers.  In addition, enterprise
Information Technology (IT) strategies are focusing on deploying both hardware
and software infrastructure to host their {\em digital assets} as
web services for controlled
access by their employees via internal corporate applications or
by their customers via software and data ``as-a-service''.

This rapidly proliferating model of application development and IT operation
is designed to scale, both in the number of applications that access the
IT-managed services, and in the number of web services that must be
hosted and curated.  Each web service exports one or more Application Programming
Interfaces (APIs) that must be accessible by users, user applications, and/or
other services.
Because applications encode their internal logic in terms of
remote ``calls'' to curated services,
these APIs define functional boundaries that must be incorporated into the
application architecture.  

APIs also define the boundaries between IT's management responsibility and the
customer base it serves.  
Applications and users ``see'' the APIs and not the
complexity of the services that implement them (which may, in fact, be held as
trade secrets).  Thus, the APIs have become the critical value-carrying 
digital asset in many sectors of the new digital economy.

From an IT management perspective, the vast collection of APIs must be
maintained as critical infrastructure components or else ``client'' applications
and other internal web services will fail to meet their service-level agreements (SLAs)
or become unavailable altogether, due to the changes and faults in APIs.  
That is, API software components have become separate digital artifacts that 
must be managed by IT to ensure operational functionality.
Modern software management strategies, such as DevOps,
require teams comprised of both developers and sysadmins to develop, 
deploy, and manage web services
that IT must curate and oversee.  In the case of Google or Amazon, these
teams number in the hundreds and the APIs that their services export in the
thousands.  Further, these services often make API calls to each other 
creating dependencies between APIs that must be carefully managed.
Moreover, the API code modules have a software life cycle that is
independent and longer than the life cycle of the services themselves. 
As technological improvements make better service implementations possible, IT
management must upgrade these services while keeping the APIs unchanged.
Similarly, as new API versions emerge to increase functionality, backward
compatibility must be maintained and/or deprecated in a way that takes into
account the strategic objectives of the organization.  As an organization
scales under this new model, the costs associated with
incorrectly managing APIs scale as well.

Today's cloud computing platforms severely lag behind in supporting
API governance. Typical cloud environments impose certain restrictions
on application and web service developers in order to guarantee scalability
and high availability (e.g. Google App Engine prevents all accesses to the file system). 
However, they do not prevent the developers from violating
software development and maintenance policies and best practices,
which is something a proper enforced governance facility should be capable of. 
Consequently, developers often violate naming and versioning conventions
when it comes to naming their APIs and other digital assets, take dependencies
on incorrect or deprecated APIs, and very often end up re-implementing program logic
from the scratch instead of using extant APIs that provide the required functionality.
This lack of API governance also leads to many security vulnerabilities (e.g. DoS attacks
by malicious or poorly coded clients), violations of IP rights and licensing terms, and in
some cases even financial penalties. Web APIs also play a key role in the areas of SLA
management and enforcement. Typically, any public API should be exposed with a well-defined
SLA detailing its performance and quality of service characteristics. Without proper means
of API governance, developers that host services in cloud settings have no way of enforcing
competitive SLAs on their APIs, or even getting a thorough idea of what kind of SLAs
their service deployments can uphold. Today, developers have to perform a lot of offline
and online testing to learn the performance characteristics of their APIs which is tedious and
time-consuming. Furthermore developers often have to be content with simple SLA monitoring 
rather than enforcing them.

With this research, {\bf we propose to develop EAGER -- {\em E}nforced {\em
A}PI {\em G}overnance {\em E}ngine for {\em R}EST -- 
an automated platform for
implementing enforced
API governance}.  In a cloud-based IT context, APIs must be {\em governed} by
policies that are defined by the organization to ensure predictable and
economic operation of the web services hosted by the cloud.  
Governance policies specify the conditions that must be met before 
an API is exposed to users and while it is in service.
The policies themselves cover IT management functions such as change
control, versioning and dependency management, auditing, and access control.
In a dynamic and scalable cloud setting, such IT governance functions must be 
implemented
in a reliable, auditable, and automated way to preserve the scaling benefits
that cloud computing makes possible.  EAGER
is a scalable, distributed system that supports the specification 
and analysis of
governance policies for APIs and their application to cloud-hosted
web services statically, during deployment, and at runtime.
Moreover, EAGER targets RESTful services since REST is 
the predominant service architecture used for web 
and cloud APIs and REST governance solutions are incomplete and ad hoc. 
%From an architectural
%perspective, web services contain separate API and service implementation code
%that is ``stitched'' together by a web service container or ``stack''.
%Thus the API software components become a separate digital asset that 
%must be managed by IT to ensure operational functionality.  
Our plan is to
\begin{itemize}
\item develop a {\bf policy specification language for REST-based web
services} (based on a subset of Python) that
enables governance enforcement through static analysis, during service
deployment, and at runtime while the service is active and available,
\item develop EAGER as a distributed {\bf platform for enforcing IT 
management policies} governing
REST-based web services when operating at scale in cloud settings, and
\item focus on the {\bf API software components} as the boundary between 
IT-managed
service implementations and client users, software applications, and other
services.
\end{itemize}

EAGER will enforce policies by combining
static verification and analysis with automated
service deployment, and runtime
sandboxing and admission control.  However, to make our approach practically
feasible (as a way of ensuring its broader impact) {\em we will focus on
the API components of cloud hosted web services only}.  That is, we view the
governance problem from the perspective of IT management and focus on the APIs
that expose hosted (but otherwise opaque) REST-based web services to clients
(applications and users) served by the IT organization.
%CJK
Doing so enables us to unify and simplify policy specification across the 
governance process, to automate policy checking and enforcement beyond what 
is available
and possible today, and to employ novel static and dynamic verification and
analysis techniques to improve the robustness of real systems for
organizations implementing the next generation of network-accessible digital
assets.


\subsection{Outcomes and Assessment}

The outcome of this research plan,
we believe, will be new advances in services computing, IT management, cloud
platform computing, and API policy specification,  verification, and  enforcement.
If successful, the research will be transformative as it
enables scalable management of the services that form the
bulk of cloud workloads.
Moreover, our use of this research framework for
education will give students hands-on experience with the
state-of-the-art in distributed system technologies and will enable us to
introduce computing through web services to students across many disciplines
(e.g. statistics, computer science, engineering, business, and others).
Thus we plan to assess the outcomes of our research, in part, by the degree
to which students glean new insights and skills that are applicable to cloud
computing.  By training a diversity of  students in this way, using EAGER, we 
also assure broader impact of our work in an educational context.

In addition,
as cloud systems researchers, we use open source, on-premises clouds (AppScale
and Eucalyptus) to serve research and educational artifacts (data sets, class
assignments and web pages, code repositories, etc.) to our students,
colleagues, and the wider research community.  Another measure of success will
be the degree to which EAGER reduces the management burden that results from
the scale and speed that these systems require.

Finally, we plan to make EAGER available as a persistent software artifact to
the wider research community and to monitor its uptake.  Because of its design
and its use of open source cloud technologies, EAGER will be readily available
to researchers and educators who wish to develop new artifacts and curricula
for cloud computing.  We will study EAGER's uptake as measured by accesses to
its source code and solicit feedback from interested users.


\end{document}
