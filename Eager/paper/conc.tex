The paradigm of software engineering is rapidly shifting towards SOA. Instead of developing everything from the scratch, or using standalone
libraries, programmers prefer to delegate most of the application functionality to remote web APIs. To facilitate this new development model,
many service providers are keen on hosting web APIs in cloud environments, which are highly scalable, fault-tolerant deployment targets available under
well-defined SLAs and pay-as-you-go cost models. As the number of cloud-hosted APIs continues to increase, the need for enforced API governance
has become an absolute necessity to ensure the quality, reliability and maintainability of web APIs. Existing approaches rely on third party
API management solutions that operate outside the clouds, which are expensive, tedious to manage and do not provide the required level of
control and manageability over the APIs deployed in the cloud.

We propose EAGER, a model and a software architecture that facilitates API governance as a cloud-native feature. EAGER supports comprehensive
policy enforcement, dependency management and a variety of other deployment-time API governance features. It promotes many software development
and maintenance best practices including versioning, code reuse and retaining API backwards compatibility. We also provide a language based on
Python that enables creating, debugging and maintaining API policies in a simple and intuitive manner. EAGER can be built right into the cloud platforms
used to host APIs and automates most governance tasks that would otherwise require custom code or developer intervention.

We implement a prototype of EAGER on an open source PaaS cloud to demonstrate its feasibility and effectiveness as a governance engine. Our
test results show that EAGER adds negligibly small overhead to the application deployment process in the cloud, and the overhead grows linearly
with the number of APIs deployed. We also show that EAGER scales well to handle tens of thousands of APIs and hundreds of policies. In this work
we have also laid the foundation for supporting run-time API governance as a cloud-native feature which we will explore in our future research.