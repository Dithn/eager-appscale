As the paradigm of cloud computing grows in popularity, the need for monitoring cloud-hosted 
applications is coming to the forefront. Application developers and cloud administrators
wish to detect performance anomalies in cloud applications, and
perform root cause analysis. However, the high level of abstraction provided by cloud
platforms, coupled with their scale and complexity, makes performance monitoring and
systemwide root cause analysis a daunting problem. This situation is particularly apparent in
PaaS clouds, which hide application runtime details beneath a layer of curated services.
The existing cloud monitoring solutions do not have the necessary penetrative power
to monitor all the different layers of such cloud platforms. Consequently, their monitoring
and root cause analysis capabilities are severely limited.

We propose Roots, an online monitoring framework for applications deployed in a PaaS cloud. 
Roots is designed to function as a curated service
built into the PaaS cloud. It relieves the application developers from having to configure
their own monitoring solutions, or having to instrument the application code in anyway.
Roots captures runtime data from all the different layers involved
in processing application requests. It can correlate events across different layers
to track the propagation of application requests through the cloud platform, and
identify bottlenecks.

Roots monitors applications for SLO compliance. When Roots detects an SLO violation, 
it analyzes workload data and other application runtime data
to perform root cause analysis. Roots is able to determine whether a particular
anomaly was caused by a change in the application workload, or due to a bottleneck
in the cloud platform. To this end we devise a bottleneck identification algorithm, that
uses a combination of linear regression, quantile analysis and change point detection.
We evaluate Roots using a prototype built for the AppScale open source PaaS. 
Our results indicate that Roots is highly effective at detecting performance anomalies
via SLO violations. We also show that our algorithm for bottleneck identification
produces accurate results almost 100\% of the time. These empirical trials further 
indicate that Roots does not add a significant overhead to the applications deployed
on the cloud platform. It is very lightweight, and scales well to handle large application
volumes with a single pod being able to handle tens of thousands of applications.