We propose Roots, a near real time monitoring and diagnostics
framework for web applications deployed in a PaaS cloud. 
Roots is designed to function as a curated service
built into the cloud platform, as opposed to an external monitoring system. 
It relieves the application developers from having to configure
their own monitoring solutions or, indeed,
from having to instrument the application code.
%Roots captures runtime data from the different layers involved
%in processing application requests. It can correlate events across different layers
%to track the propagation of application requests through the cloud platform, and
%identify bottlenecks deep within the kernel services of the PaaS.

%Roots monitors applications for SLO compliance, and detects anomalies via SLO violations.
%When Roots detects an anomaly, 
%it analyzes workload data and other application runtime data
%to perform root cause analysis. Roots is able to determine whether a particular
%anomaly was caused by a change in the application workload, or due to a bottleneck
%in the cloud platform. To this end we also devise a bottleneck identification algorithm, that
%uses a combination of linear regression, quantile analysis and change point detection.

We evaluate Roots using a prototype built for the AppScale open source PaaS. 
Our results indicate that Roots is effective at detecting performance anomalies
in near real time. We also show that our bottleneck identification algorithm
produces accurate results nearly 100\% of the time (cf
Table~\ref{tab:results_summary}), pinpointing the exact PaaS
service or the application component responsible for each anomaly. Our empirical trials further 
reveal that Roots does not add a significant overhead to the applications deployed
on the cloud platform. Finally, we show that Roots is lightweight, 
and scales well to handle large populations of applications. 
%Specifically,
%we see that a single Roots pod is able to monitor tens of thousands of applications.

%In our future work we plan to expand the data gathering capabilities of Roots into
%the low level virtual machines and containers that host various services of the cloud
%platform. We intend to tap into the hypervisors
%and container managers to harvest runtime data regarding the resource usage (CPU, memory, disk etc.) of
%PaaS services and other application components. With that we expect to extend
%the root cause analysis support of Roots so that it can not only pinpoint the
%bottlenecked application components, but also the low level hosts and system
%resources that constitute each bottleneck.

\vspace{0.1in}
\noindent
This work is supported in part by NSF (CCF-1539586, CNS-0905237) and Huawei.
