Roots falls into the category of performance anomaly detection and bottleneck identification 
(PADBI) systems~\cite{Ibidunmoye:2015:PAD:2808687.2791120}. 
They play a crucial role in achieving guaranteed performance and
quality of service by detecting performance issues before they escalate into major outages
or SLO violations~\cite{6045942}. 
However, the paradigm of cloud computing is yet to be
fully penetrated by PADBI systems research. The size, complexity and the dynamic nature of 
cloud platforms make performance monitoring a particularly challenging problem.
The existing technologies like Amazon CloudWatch~\cite{cloudwatch},
New Relic~\cite{newrelic} and DataDog~\cite{datadog} facilitate monitoring cloud applications 
by instrumenting low level cloud resources (e.g. virtual machines), and application code. But such technologies
are either impracticable or insufficient in
PaaS clouds where the low level cloud resources are hidden from users by software abstractions, while
application-level instrumentation is generally tedious and error-prone.

Tracing system activities via request tagging has been employed in several
previous works such as X-Trace~\cite{Fonseca:2007:XPN:1973430.1973450} and 
PinPoint~\cite{Chen:2002:PPD:647883.738238}.
X-Trace records network activities across protocols and layers, but
does not support root cause analysis. PinPoint traces
interactions among J2EE middleware components to localize faults rather than performance
issues. Currently it is limited to single-machine tracing within a JVM.

Aguilera et al developed a performance debugging framework for distributed systems comprised of
blackbox components~\cite{Aguilera:2003:PDD:945445.945454}. 
They infer the inter-call causal paths between application components, and attribute delays to those
components. Roots does something similar, but we only consider one level of
interaction in our work -- from application code to PaaS kernel services. This is sufficient
to determine the PaaS kernel invocation that may have caused a performance anomaly.
The blackbox components considered in our work (i.e. the PaaS kernel services) do not
typically interact with each other. However, each PaaS kernel service is a distributed system
of its own right with many internal components running on different containers and VMs.
In our future work we wish to also factor in the events that occur within PaaS services
and the underlying IaaS components, 
and expand the bottleneck identification capabilities of Roots down to internal
service components, containers and VMs.

Attariyan et al presented X-ray~\cite{Attariyan:2012:XAR:2387880.2387910}, 
a performance summarization system that can be applied
to a wide range of production software. X-ray attributes performance costs to each
basic block executed by a software application, and estimates the likelihood a block
was executed due to a potential root cause. With that X-ray can compute a list of root cause
events ordered by performance costs. However, X-ray relies on instrumenting application
binaries. It also attributes performance
issues to root causes under user's control such as configuration issues.
Therefore for each application the X-ray user must specify the application configurations,
program inputs, 
as well as a way to identify when a new request begins. Roots requires neither binary
instrumentation, nor any additional user input. Also it can detect performance bottlenecks
in the cloud platform that are beyond application developer's control. In the
future we wish to explore the idea of selective instrumentation, in which
we temporarily instrument an application to gather additional runtime data for a period, so
we can perform X-ray like performance summarization on-demand.

Systems noted above (\cite{Chen:2002:PPD:647883.738238,Aguilera:2003:PDD:945445.945454,Attariyan:2012:XAR:2387880.2387910}) 
collect/trace data online, but perform heavy computations
offline. We take the same approach in Roots by running anomaly detection
and root cause analysis as offline processes. However, none of the above
systems have been designed for or tested in a cloud environment. Roots on
the other hand is specifically designed to monitor web applications in
PaaS clouds where the abstractions, scale and user expectations pose unique challenges.

Our work is heavily inspired by the past literature that detail the key features of 
cloud APMs~\cite{DaCunhaRodrigues:2016:MCC:2851613.2851619,Ibidunmoye:2015:PAD:2808687.2791120} . 
We also borrow from
Magalhaes and Silva who uses statistical correlation and linear regression to perform
root cause analysis in web applications~\cite{Magalhaes:2010:DPA:1906485.1906774, Magalhaes:2011:RAP:1982185.1982234}. 
Dean et al implemented PerfCompass~\cite{Dean:2014:PTR:2696535.2696551}, 
an anomaly detection and localization method for IaaS clouds. They instrument operating system kernels
of VMs to perform root cause analysis in IaaS clouds. We take a similar approach where we instrument
the kernel services of the PaaS cloud.
Nguyen et al presented PAL, another anomaly detection mechanism targeting
distributed applications deployed on IaaS clouds~\cite{Nguyen:2011:PPR:2038633.2038634}. 
Similar to Roots, they also use an SLO monitoring approach to detect anomalies.

Anomaly detection is a general problem not restricted to performance analysis. Researchers
have studied anomaly detection from various standpoints and come up with many
algorithms~\cite{Chandola:2009:ADS:1541880.1541882}.
While we use many statistical methods
in our work (change point analysis, relative importance, quantile analysis), Roots is not tied to any of these
techniques. Rather, we provide an extensible framework on top of which new anomaly detectors and anomaly
handlers can be built.