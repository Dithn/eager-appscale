Roots falls into the category of performance anomaly detection and bottleneck identification 
(PADBI) systems~\cite{Ibidunmoye:2015:PAD:2808687.2791120}. 
Such systems play a crucial role in achieving guaranteed performance and
quality of service by detecting performance issues in a timely manner before they escalate into major outages
or SLO violations~\cite{6045942}. 
However, the paradigm of cloud computing is yet to be
fully penetrated by PADBI systems research. The size, complexity and the dynamic nature of 
cloud platforms make performance monitoring a particularly challenging problem.
The existing technologies like Amazon CloudWatch~\cite{cloudwatch},
New Relic~\cite{newrelic} and DataDog~\cite{datadog} facilitate monitoring cloud applications 
by instrumenting low level cloud resources (e.g. virtual machines), and application code. But such technologies
are either impracticable or insufficient in
PaaS clouds where the low level cloud resources are hidden from users by software abstractions, while
application-level instrumentation is generally tedious and error-prone.

Our work is heavily inspired by the past literature that detail the key features of 
cloud APMs~\cite{DaCunhaRodrigues:2016:MCC:2851613.2851619,Ibidunmoye:2015:PAD:2808687.2791120} . 
We also borrow from
Magalhaes and Silva who uses statistical correlation and linear regression to perform
root cause analysis in web applications~\cite{Magalhaes:2010:DPA:1906485.1906774, Magalhaes:2011:RAP:1982185.1982234}. 
Dean et al implemented PerfCompass~\cite{Dean:2014:PTR:2696535.2696551}, 
an anomaly detection and localization method for IaaS clouds. They instrument operating system kernels
of VMs to perform root cause analysis in IaaS clouds. We take a similar approach where we instrument
the kernel services of the PaaS cloud.
Nguyen et al presented PAL, another anomaly detection mechanism targeting
distributed applications deployed on IaaS clouds~\cite{Nguyen:2011:PPR:2038633.2038634}. 
Similar to Roots, they also use an SLO monitoring approach to detect anomalies.

Anomaly detection is a general problem not restricted to performance analysis. Researchers
have studied anomaly detection from various standpoints and come up with many
algorithms~\cite{Chandola:2009:ADS:1541880.1541882}.
While we use many statistical methods
in our work (change point analysis, relative importance, quantile analysis), Roots is not tied to any of these
techniques. Rather, we provide an extensible framework on top of which new anomaly detectors and anomaly
handlers can be built.