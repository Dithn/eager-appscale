%% complexity and convenience make diagnosing PaaS application performance hard
%% -- PaaS is opaque to programmers
%% -- PaaS services must operate at scale (asynchronous, distributed)
%% new PaaS service for performance diagnotics
%% -- can't require application instrumentation to be a PaaS service
%% -- must be fully automated
%% -- must be integrated with the runtime system 
%% this paper: anaomaly detection and root cause identification
%% -- anaomaly: change in performance that results in SLA violation
%% -- root cause: diagnosis of the reason for the anomaly as either a change
%%    in workload, increased latency an application component, or increased
%%    latency in a PaaS service and the identification of which one
%% -- Roots is more general with respect to detectors and handlers but we
%%    investigate anomaly detection and root cause identification defined
%%    above in this paper
%% diagnosis => after the fact
%% -- can use asynchronous, but must be able to time correlate
Over the last decade cloud computing has become a popular approach for deploying
applications at scale~\cite{Antonopoulos:2010:CCP:1855007,Pinheiro:2014:ACC:2618168.2618188}. 
Many organizations, academic institutions, and hobbyists make use of public
and/or private clouds to deploy their applications.
This rapid growth in cloud technology has intensified the need 
for new techniques to
monitor applications deployed in cloud platforms~\cite{DaCunhaRodrigues:2016:MCC:2851613.2851619}. 
Application developers and users wish to monitor application performance, detect 
anomalies, and identify system bottlenecks. To obtain this level of deep operational insight into
cloud-hosted applications, the cloud platforms must be equipped with powerful
data gathering and analysis capabilities that span the entire stack of the cloud. 
However, most cloud technologies available
today either do not provide any application monitoring support, or only provide primitive
monitoring features such as application-level logging. Hence, they are not capable of conducting
performance anomaly detection or bottleneck identification, which requires much more fine-grained
data collection and analytics across the entire cloud platform.

Further compounding this problem, today's cloud platforms are very 
large and complex~\cite{DaCunhaRodrigues:2016:MCC:2851613.2851619,Ibidunmoye:2015:PAD:2808687.2791120}. They are
comprised of many layers, where each layer may consist of a multitude of interacting components.
Therefore when a performance anomaly manifests in a user application, it is rather challenging 
to determine the exact layer or the component of the cloud platform that may be responsible for it. 
Facilitating this level of comprehensive root cause analysis requires
data collection at different layers of the cloud, and mechanisms for correlating events at
different layers and components. Today's cloud platforms do not support such fine-grained
data collection across he board. The plethora of existing third party cloud monitoring solutions
do not have visibility into the low-level activities of the cloud thereby rendering them incapable
of performing systemwide root cause analysis.

Moreover, performance monitoring for cloud applications needs to be highly customizable. Different
applications have different monitoring requirements in terms of data gathering frequency (sampling rate), 
length of the history to consider when performing statistical analysis, and the performance 
SLOs (service level objectives) to maintain over time and check for violations. Cloud monitoring
should be able to facilitate these diverse requirements at the granularity of the applications.
Designing such customizable and extensible performance
monitoring frameworks that are built into the cloud platforms is a novel and challenging undertaking.

To address these needs, we present the design of  a comprehensive application platform 
monitor (APM) called Roots that can be built into a wide variety of cloud Platform-as-a-Service (PaaS) technologies. 
PaaS clouds provide a very high level of abstraction that hides most of the details concerning application
runtime. They provide a set of managed services, which developers compose into applications.
The implementation and deployment details of these services are completely opaque to the developers. 
However, the existing cloud monitoring techniques rely on being able to capture data from all components 
of an application, all the way down to the virtual machines that host them. Such methods are
therefore not viable in a PaaS cloud that hides runtime details beneath a layer of managed services.
To overcome this barrier, we design Roots as another managed service built into the PaaS cloud.
It operates at the same level as the other services offered by the cloud platform. This way Roots can collect data
directly from the service implementations internal to the cloud platform, thus gaining full visibility into all the 
critical runtime events of an application. It also enables Roots to operate fully automatically in the background without
requiring instrumentation of application code. Roots can intercept and record all the important runtime events as the
application code invokes various service implementations of the PaaS cloud.

%The proposed
%APM is not an external system that monitors a cloud platform from the outside (as most APM systems today). 
%Rather, it integrates with
%the PaaS cloud from within thereby extending and augmenting the existing components of the PaaS cloud
%to provide comprehensive full stack monitoring. 
%We believe that this design decision is a key differentiator over existing cloud 
%application monitoring systems because (i) it is
%able to take advantage of the scaling, efficiency, deployment, fault tolerance, security, 
%and control features that the underlying cloud offers, 
%(ii) while providing low overhead end-to-end monitoring of cloud applications.

Previous work has outlined several key requirements that need to be taken into consideration when
designing a cloud monitoring system~\cite{DaCunhaRodrigues:2016:MCC:2851613.2851619,Ibidunmoye:2015:PAD:2808687.2791120}. 
We incorporate many of these features into our design:
\begin{description}
\item[Scalability] Roots is extremely lightweight, and does not cause any noticeable overhead in 
application performance. It puts strict upper bounds on the volume of data kept in memory. 
The persistent data is accessed on demand, and can be removed after their usefulness has expired.
\item[Multitenancy] Roots facilitates configuring monitoring policies at the granularity of individual applications.
Users can employ different statistical analysis methods to process the monitoring data in ways that are 
most suitable for their applications.
\item[Complex application architecture] We design Roots to collect data from the entire cloud stack 
(load balancers, app servers, PaaS kernel services etc.). Roots is able to correlate data gathered
from different parts of the cloud platform, and perform systemwide bottleneck identification.
\item[Dynamic resource management] Cloud platforms are dynamic in terms of their magnitude 
and topology. By augmenting all the key components along the request processing path of the cloud platform,
we make sure that Roots capture all the critical runtime data. When new processes/components
spring to life in the cloud platform, they inherit the same augmentations, and start reporting to Roots automatically.
\item[Autonomy] Roots is designed to detect performance anomalies online, without manual intervention.
When Roots detects a problem, it attempts to automatically identify the root cause by analyzing
available workload and service invocation data. Data collection begins automatically for new
applications as they are deployed.
\end{description}

Roots collects most of the data it requires by instrumenting various internal components 
of the cloud platform. In addition to high-level metrics like request throughput
and request latency, Roots also records the internal PaaS services invoked by user applications,
and the latency of those internal service calls. It uses batch operations and asynchronous 
communication to record events in a manner that does not introduce a
delay to the application request processing activities. 
In addition, Roots employs a collection of lightweight  benchmarking
processes to collect performance data regarding user applications. Both
the benchmarking processes, and the data analysis processes are executed 
out of the request processing flow of the cloud platform. Such processes can be
grouped together, and managed as a single entity called a
``Roots Pod''. Pods keep minimum state
information regarding the applications they monitor and analyze. This enables
a single pod to monitor a large number of applications. Each pod is self-contained,
and therefore scalability and high availability can be achieved by running multiple pods (sharding),
and running multiple replicas of the same pod.

When Roots detects a performance anomaly in an application, it attempts to uncover the
root cause of the anomaly by analyzing the workload data,
and the performance of the internal PaaS cloud services the application depends on. 
This way Roots can determine if the detected anomaly was caused by a change in the
application workload (e.g. a sudden spike in the number of client requests), or an internal
bottleneck in the cloud platform (e.g. a slow database query). To this end we also propose
a statistical bottleneck identification method as part of this work. 
It uses a combination of quantile analysis, change point detection
and linear regression to perform root cause analysis. We demonstrate that this approach is
capable of detecting the root causes of performance anomalies with nearly 100\% accuracy. 
At the same time we emphasize that this is one of the many bottleneck identification methods
that can be implemented on top of Roots.

The following sections detail the architecture of Roots APM, and how it integrates with a typical PaaS
cloud. We describe individual components of the APM, their functions and how they interact with each
other. Where appropriate, we also detail the concrete technologies (tools and products) that we plan to use to implement
various components of the APM, and provide our rationale and intuition behind choosing these technologies.
