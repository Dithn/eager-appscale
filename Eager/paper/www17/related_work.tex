Roots falls into the category of performance anomaly detection and bottleneck identification (PADBI) systems.
A PADBI system is an entity that observes, in real time, the performance behaviors
of a running system or application, while collecting vital measurements at discrete time intervals to create baseline
models of typical system behaviors~\cite{Ibidunmoye:2015:PAD:2808687.2791120}. 
Such systems play a crucial role in achieving guaranteed service reliability, performance and
quality of service by detecting performance issues in a timely manner before they escalate into major outages
or SLO violations~\cite{6045942}. 
PADBI systems are thoroughly researched, and well understood in the context of traditional standalone and
network applications. Many system administrators are familiar with frameworks like Nagios, Open NMS and Zabbix which
can be used to collect data from a wide range of applications and devices. The collected data can be
subjected to statistical analysis and/or machine learning to detect performance anomalies. 

However, the paradigm of cloud computing, being relatively new, is yet to be
fully penetrated by PADBI systems research. The size, complexity and the dynamic nature of 
cloud platforms make performance monitoring a particularly challenging problem.
The existing technologies like Amazon CloudWatch
and New Relic facilitate monitoring cloud applications in IaaS clouds by observing low level
cloud resources (e.g. virtual machines), and instrumenting application code. But such technologies
are neither viable nor sufficient in
PaaS and SaaS clouds where the low level cloud resources are hidden under layers of managed
services and SDKs, and the application code is executed in sandboxed environments that are not
amenable to instrumentation. Roots on the other hand is built into the 
fabric of the PaaS cloud giving it full visibility into all the activities that take place in the entire
software stack.

Our work borrows heavily from the past literature~\cite{DaCunhaRodrigues:2016:MCC:2851613.2851619,Ibidunmoye:2015:PAD:2808687.2791120} 
which detail the key features of cloud APMs. As a result, we pay special attention to requirements like
scalability, autonomy and dynamic resource management in our design.
Ibidunmoye et al highlight the importance of multilevel bottleneck identification as an open research
question~\cite{Ibidunmoye:2015:PAD:2808687.2791120}. This is the ability to
identify bottlenecks from a set of top-level application service components, and further down through the 
virtualization layer to system resource bottlenecks. Our plan for Roots is very much in sync with this
vision. We currently support identifying bottlenecks from a set of services provided by the PaaS kernel.
As a part of our future work, we plan to extend this support towards the virtualization layer and the
physical resources of the cloud platform.

A number of researchers have emphasized the importance of differentiating between workload changes and
application issues when performing bottleneck identification. Cherkasova et al developed an online
performance modeling technique to detect anomalies in traditional transaction processing systems~\cite{4630116}. 
They divide time into contiguous segments, such that within each
segment the application workload (volume and type of transactions) and resource usage (CPU) can be 
fit to a linear regression model.
Segments for which a model cannot be found, are considered anomalies. Then they remove anomalous segments
from the history, and perform model reconciliation to differentiate between workload changes and application problems. 
While this method is powerful, it
requires instrumenting application code to detect different external calls (e.g. database queries) executed by the application. 
Since the model uses different transaction types as parameters, 
some prior knowledge regarding the transactions needs to be fed into the system. 
The algorithm is also very compute intensive, due to continuous segmentation and model fitting.
In contrast, we use a very lightweight SLO monitoring method in Roots to detect performance anomalies, and
only perform heavy computations to perform bottleneck identification. 

Dean et al implemented PerfCompass~\cite{Dean:2014:PTR:2696535.2696551}, 
an anomaly detection and localization method for IaaS clouds. They instrument
the VM operating systems to capture the system calls made by user applications. Anomalies are detected by
looking for unusual increases in system call execution time. They group system calls into execution units
(processes, threads etc), and analyze how many units are affected by any given anomaly (fault impact factor). 
Based on this metric they conclude if the problem was caused by a workload change or an application
level issue. We take a similar approach in Roots, in that we capture the PaaS kernel invocations
made by user applications. We use application response time (latency) as an indicator of anomalies,
and group PaaS kernel invocations into application requests to perform bottleneck identification.

Nguyen et al presented PAL, another anomaly detection and localization mechanism targeting
distributed applications deployed on IaaS clouds~\cite{Nguyen:2011:PPR:2038633.2038634}. 
Similar to Roots, they also use an SLO monitoring approach to detect application
performance anomalies. When an anomaly is detected, they perform change point analysis
on gathered resource usage data (CPU, memory and network) to identify the anomaly onset time.
Having detected one or more anomaly onset events in different components of the
distributed application, they sort the events by time to determine
the propagation pattern of the anomaly.

Magalhaes and Silva have made significant contributions in the area of anomaly detection
and root cause analysis in web applications~\cite{Magalhaes:2010:DPA:1906485.1906774, Magalhaes:2011:RAP:1982185.1982234}. 
They compute the correlation between application
workload and latency. If the level of correlation drops significantly, they consider it to be an
anomaly. A similar correlation analysis between workload and other system parameters 
(e.g. CPU and memory usage) is used to check whether an anomaly was caused by a workload
change. They also use an aspect-oriented programming model in their target applications, which
allows them to easily instrument application code, and gather metrics regarding various 
remote services (e.g. database) invoked by the application. This data is subjected to a 
series of simple linear regressions to perform root cause analysis. This approach assumes
that remote services are independent of each other. However, in a cloud platform where
kernel services are deployed in the same shared infrastructure, this assumption might not
hold true. Therefore we improve on their methodology, and use multiple linear regression
with relative importance to identify cloud platform bottlenecks. Relative importance
is resistant to multicollinearity, and therefore does not require the independence assumption.

Anomaly detection is a general problem not restricted to performance analysis. Researchers
have studied anomaly detection from many different points of view, and as a result many
viable algorithms and solutions have emerged over time~\cite{Chandola:2009:ADS:1541880.1541882}.
Prior work in performance anomaly detection and root cause analysis can be classified as statistical
methods (e.g. \cite{6311395,Malkowski:2007:BDU:1783374.1783389,Magalhaes:2011:RAP:1982185.1982234,Nguyen:2011:PPR:2038633.2038634}) 
and machine learning methods (e.g. \cite{Cohen:2004:CID:1251254.1251270,Yu:2013:SNA:2494621.2494643,bhaduri2011detecting}).
While we use many statistical methods
in our work (change point analysis, relative importance, quantile analysis), Roots is not tied to any of these
techniques. Rather, we provide a framework on top of which new anomaly detectors and anomaly
handlers can be built. It is indeed possible to build an anomaly detector that trains a model
from historic data, and then uses it to detect anomalies in newly encountered data. We
leave evaluating the tradeoffs between resolution power and scalability of such advanced
methods to our future work.