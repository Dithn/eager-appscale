Roots falls into the category of performance anomaly detection and bottleneck identification (PADBI) systems.
A PADBI system is an entity that observes, in real time, the performance behaviors
of a running system or application, while collecting vital measurements at discrete time intervals to create baseline
models of typical system behaviors~\cite{Ibidunmoye:2015:PAD:2808687.2791120}. 
Such systems play a crucial role in achieving guaranteed service reliability, performance and
quality of service by detecting performance issues in a timely manner before they escalate into major outages
or SLO violations. PADBI systems are thoroughly researched, and well understood in the context of traditional standalone and
network applications. Many system administrators are familiar with frameworks like Nagios, Open NMS and Zabbix which
can be used to collect data from a wide range of applications and devices. The collected data can be
subjected to statistical analysis and/or machine learning to detect performance anomalies. 

However, the paradigm of cloud computing, being relatively new, is yet to be
fully penetrated by PADBI systems research. The size, complexity and the dynamic nature of 
cloud platforms make performance monitoring a particularly challenging problem.
The existing technologies like Amazon CloudWatch
and New Relic facilitate monitoring cloud applications in IaaS clouds by observing low level
cloud resources (e.g. virtual machines), and instrumenting application code. But such technologies
are neither viable nor sufficient in
PaaS and SaaS clouds where the low level cloud resources are hidden under layers of managed
services and SDKs, and the application code is executed in sandboxed environments that are not
amenable to instrumentation. The added configuration overhead and the financial cost of such monitoring
solutions may further preclude their use in some cases. Roots on the other hand is built into the 
fabric of the PaaS cloud giving it full visibility into all the activities that take place in the entire
software stack. It does not require additional
code instrumentation or configuration on the part of the application developer. Roots is always on,
and active as long as the underlying cloud platform is. It is uniquely capable of tracing application 
performance anomalies to components deep within the cloud stack, that are not 
usually exposed to external monitoring or probing.

Ibidunmoye et al showed that a PADBI system designed for the cloud should take several key 
properties of cloud applications into consideration~\cite{Ibidunmoye:2015:PAD:2808687.2791120}. 
We account for all these factors in our design of Roots:
\begin{description}
\item[Scalability] Roots is extremely lightweight, and does not cause any noticeable overhead in 
application performance. It puts strict upper  bounds on the volume of data kept in memory. 
The persistent data is accessed on demand, and can be removed after their usefulness has expired.
\item[Multitenancy] Roots facilitates configuring anomaly detectors at the granularity of individual applications.
Users can employ different statistical analysis methods to process the monitoring data in ways that are 
most suitable for their applications.
\item[Complex application architecture] We design Roots to collect data from the entire cloud stack 
(load balancers, app servers, PaaS kernel services etc.). Roots is able to correlate data gathered
from different parts of the cloud platform, and perform systemwide bottleneck identification.
\item[Dynamic resource management] Cloud platforms are dynamic in terms of their magnitude 
and topology. By augmenting all the key components along the request processing path of the cloud platform,
we make sure that Roots capture all the critical runtime data. When new processes/components
spring to life in the cloud platform, they inherit the same augmentations, and start reporting to Roots automatically.
\item[Autonomous operation] Roots is designed to detect performance anomalies online, without manual intervention.
When Roots detects a problem, it attempts to automatically identify the root cause by analyzing
available workload and service invocation data. 
\end{description}
The same survey highlights the importance of multilevel bottleneck identification as an open research
question. This is the ability to
identify bottlenecks from a set of top-level application service components, and further down through the 
virtualization layer to system resource bottlenecks. Our plan for Roots is very much in sync with this
vision. We currently support identifying bottlenecks from a set of services provided by the PaaS kernel.
As a part of our future work, we plan to extend this support towards the virtualization layer and the
physical resources of the cloud platform.

 