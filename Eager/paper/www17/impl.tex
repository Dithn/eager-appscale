We implemented a Roots prototype for the AppScale cloud platform. AppScale is an open
source PaaS cloud, API compatible with the popular Google App Engine (GAE) cloud platform.
Due to the API compatibility, any application developed for the GAE cloud can be deployed
and executed on AppScale with no code changes. Since AppScale is open source software, we
were able to modify parts of its implementation to integrate the Roots APM into it. AppScale also
turned out to be a great test environment since we could deploy it locally, and run a number of
existing App Engine applications on it.

We use ElasticSearch as the data storage component of our prototype. ElasticSearch is ideal 
for storing large volumes of structured and semi-structured data. It supports scalability and 
high availability via sharding and replication.
ElasticSearch continuously organizes and indexes data, making the information available 
for fast retrieval and efficient querying. Additionally it also provides
powerful data filtering and aggregation features, which greatly simplify the implementations of high-level
data analysis algorithms.

We configure AppScale's front-end server (based on Nginx) to tag all incoming application requests
with a unique identifier. This identifier is attached to the request as a custom HTTP header.
All data collecting agents in the cloud extract this identifier, and include it as an attribute
in all the events reported to ElasticSearch. This enables our prototype to aggregate events originating
from the same application.

We implement a number of data collecting agents in AppScale to gather runtime information
from all major components. These agents buffer data locally, and store them in ElasticSearch
in batches. For scraping server logs and storing the extracted entries in ElasticSearch,
we use the Logstash tool. Logstash supports scraping a wide range of standard log formats (e.g. 
Apache HTTPD access logs), and other custom log formats can be supported via a simple configuration.
It also integrates naturally with ElasticSearch.
To capture the PaaS kernel invocation data, we augment AppScale's PaaS SDK implementation,
which is derived from the GAE PaaS SDK. More specifically we implement an agent that records
all PaaS SDK calls, and reports them to ElasticSearch asynchronously. 

Roots pods are implemented as standalone Java server processes. Threads are used to run benchmarkers,
anomaly detectors and handlers concurrently within each pod. Pods communicate with ElasticSearch via
REST calls, and many of the data analysis tasks such as filtering and aggregation are performed
in ElasticSearch itself. By doing so a lot of the heavy computations are offloaded to the 
ElasticSearch cluster, which is specifically designed for high-performance query processing
and analytics. Some of the more sophisticated statistical analysis tasks (e.g. change point detection, 
linear regression) are implemented in R language,
and the Java-based Roots pods integrate with R using the Rserve protocol.


