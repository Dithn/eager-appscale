Our research builds upon advances in the areas of SOA governance and
service management. 
%Many researchers have highlighted the importance of these
%subjects in the past and introduced various models and systems to implement
%them. 
Guan et al introduced FASWSM~\cite{1607141} a web service management
framework for application servers. FASWSM uses an adaptation technique that
wraps web services in a way so they can be managed by the underlying
application server platform. Wu et al introduced DART-Man~\cite{1504267} a web
service management system based on semantic web concepts.  Zhu and Wang
proposed a model that uses Hadoop and HBase to store web service metadata and
process them to implement a variety of management functions~\cite{5959326}.
Our work differentiates 
from these past approaches since EAGER is strongly focused
towards managing APIs in cloud platforms ({\em e.g.} PaaS) and it supports powerful
policy enforcement features.

Lin et al proposed a service management system for clouds that monitors all
service interactions via special ``hooks'' that are connected to the
cloud-hosted services~\cite{5616981}. These hooks monitor and record service
invocations, and also provide an interface so that the individual service
artifacts can be managed remotely. However, this system only supports run-time
service management and provides no support for deployment-time service
management or policy enforcement which are key components of a complete API
governance solution. Kikuchi and Aoki~\cite{6525502} proposed a technique
based on model checking to evaluate the operational vulnerabilities and fault
propagation patterns in cloud services. However, it does not provide any
active monitoring or enforcement functionality that is required to make it a
deployment-time or run-time governance component. Sun et al proposed a
reference architecture for monitoring and managing cloud
services~\cite{5579654}. This too lacks deployment-time governance, policy
validation support and the ability to intercept and act upon API calls which
makes it severely short of being a comprehensive governance solution for
clouds.

%Using policies to govern web service and SOA venues is a well researched idea
%that's already in widespread use. 
Researchers have shown that policies can be
used to perform a wide range of governance tasks for SOA as access
control~\cite{4279630,5713420}, fault diagnosis~\cite{6154236},
customization~\cite{4027138},
composition~\cite{1592403,Erradi:2006:PMS:1515984.1515990} and
management~\cite{Suleiman:2009:IUM:1564601.1564730,6481237,4028029}. We build
on the foundation of these past efforts and use policies to govern
RESTful web APIs deployed in cloud settings. 
However, our work is differentiated by
%of policy languages, each targeted at achieving a specific governance task,
%and instead strive to implement a single 
its unifying policy language that can be
used to specify a wide range of governance requirements. We also use a simple
and developer-familiar syntax based on the Python programming language.
%instead of
%using WS-Policy or any other XML-based language which we believe are too
%verbose and tedious to program with.

Peng, Lui and Chen showed that
the major concerns associated with SOA governance involve retaining the high reliability of services, recording how many services
are available on the platform to serve, and making sure all the available services are operating within an acceptable service
level~\cite{4730489}. EAGER attempts to satisfy similar requirements for modern RESTful web APIs deployed in cloud environments. EAGER's policy
validation, dependency management and API change management features are geared at ensuring that deployed services
are highly reliable. Its Metadata Manager and ADP record and keep track of all deployed APIs in a simple but comprehensive manner.
%Our future work on run-time governance will focus on ensuring that deployed services are operating under well-defined SLAs 
%while providing satisfactory quality of service features
