PaaS clouds have been experiencing a rapid growth in 
popularity in the recent years~\cite{paas-growth,paas-growth2}. They
typically host web-accessible (HTTP/S) applications, while providing
sandboxed execution, high scalability, and high availability.
PaaS clouds are complex distributed systems that
provide scalability by automatically allocating resources for
applications on the fly (auto scaling), and availability through the
execution of multiple instances of each application.  

PaaS applications rely on a set of managed, scalable services 
offered by the underlying cloud platform. We refer to these services as 
PaaS kernel services. PaaS clouds like Google
App Engine~\cite{gae} and Microsoft Azure~\cite{azure} export
the kernel services via a well-defined set of APIs, that are collectively
referred to as the PaaS ``software development kit'' (SDK). The application
servers provide the linkage between application code and the PaaS kernel
services. A set of front-end servers expose web application entry points, and
provide load-balancing for HTTP/S clients invoking the applications.

By providing most of the functionality that applications require via
kernel services, the PaaS model significantly reduces the amount of code that applications developers
must write.  PaaS clouds also relieve developers of the burden of configuration, 
deployment, and scaling through platform automation.  In combination, the PaaS model
significantly increases programmer productivity.
However, a downside of this approach is that these features also hide
the performance details of PaaS applications. 
Since the applications spend most of their time executing kernel 
services~\cite{Jayathilaka:2015:RTS:2806777.2806842},
it is challenging for the developers to diagnose performance issues given the
opacity of the cloud platform's internal implementation. 

One way to circumvent this 
problem is to instrument application code~\cite{newrelic,datadog,dynatrace}, 
and continuously monitor the time taken by various
parts of the application. But such application-level instrumentation is tedious, and
error prone thereby misleading those attempting to diagnose problems.
Moreover, the instrumentation code may slow down or alter the application's
performance. 
In contrast, implementing data collection and analysis as a kernel service 
built into the PaaS cloud allows 
performance diagnosis to be a ``curated'' service that is 
reliably managed by the cloud platform.

%Client programs interact PaaS-hosted applications via web APIs.  That is, clients
%send HTTP/S requests and receive machine readable responses (e.g. HTML, JSON,
%XML, Protocol Buffers~\cite{protobuff}) in return. Web applications are typically 
%interactive and have clients with strict expectations on the application
%response time~\cite{latency-matters}.
%Additionally, the PaaS cloud on which an application is running may impose 
%additional constraints on the
%application response time for scalability
%reasons~\cite{azure-limits,gae-limits}.  
%We observe that PaaS-hosted web applications spend a majority of their 
%execution time in the PaaS kernel services~\cite{Jayathilaka:2015:RTS:2806777.2806842}.


%moved to architecture since its really not background but instead the model we assume for roots
%\subsection{Performance Anomalies}
%Our model is one in which the clients of a web application have engaged in a
%``service-level agreement'' (SLA)~\cite{Keller:2003:WFS:635430.635442}
%with the ``owner'' or operator of the application that is hosted in a PaaS.  The SLA
%stipulates a response-time ``service-level objective'' (SLO) that, if violated, constitutes a breech of the
%agreement.
%If the performance of an application deteriorates to the
%point that at least one of its SLOs is violated, we treat it as an \textit{anomaly}. Moreover, the process
%of diagnosing the reason for 
%an anomaly is referred to as \textit{root cause analysis}.
%For a given anomaly, the root cause could be a change in the application workload or
%a \textit{bottleneck} in the application runtime. Bottlenecks may occur in the application code, 
%or in the PaaS services that the application rely on.

%In order to maintain a satisfactory level of user experience and adhere to any previously
%agreed upon performance SLOs, application developers and cloud administrators wish
%to detect performance anomalies as soon as they occur. When detected, they
%must perform root cause analysis to identify the cause of the anomaly, and take some
%corrective and/or preventive action. 
%This diagnosis usually occurs as a two step process. First, one must determine
%whether the anomaly was caused by a change in the workload (e.g. a sudden 
%increase in the number of client requests). If that is the case,
%the resolution typically involves allocating more resources to the application or spawning
%more instances of the application for load balancing purposes. If the anomaly cannot be 
%attributed to a workload change, one must go another step to find the bottleneck component
%that has given rise to the issue at hand.

%Detecting performance anomalies
%requires continuous monitoring of application performance which could be tedious with
%cloud platforms in use today. It is even more challenging to perform root cause analysis
%due to the complexity and the blackbox nature of the cloud platforms. 

%Note that there are several
%third party cloud monitoring solutions available today which provide some performance
%anomaly detection support~\cite{newrelic,datadog,dynatrace}. 
%However, they require additional configuration and code instrumentation, are expensive
%and cannot support root cause analysis across the entire cloud stack since they do not
%have visibility into all components of the cloud platform.
