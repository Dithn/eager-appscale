% This is "sig-alternate.tex" V2.1 April 2013
% This file should be compiled with V2.5 of "sig-alternate.cls" May 2012
%
% ----------------------------------------------------------------------------------------------------------------
% This .tex file (and associated .cls V2.5) produces:
%       1) The Permission Statement
%       2) The Conference (location) Info information
%       3) The Copyright Line with ACM data
%       4) NO page numbers
%
% as against the acm_proc_article-sp.cls file which
% DOES NOT produce 1) thru' 3) above.
%
% Using 'sig-alternate.cls' you have control, however, from within
% the source .tex file, over both the CopyrightYear
% (defaulted to 200X) and the ACM Copyright Data
% (defaulted to X-XXXXX-XX-X/XX/XX).
% e.g.
% \CopyrightYear{2007} will cause 2007 to appear in the copyright line.
% \crdata{0-12345-67-8/90/12} will cause 0-12345-67-8/90/12 to appear in the copyright line.
%
% ---------------------------------------------------------------------------------------------------------------
% This .tex source is an example which *does* use
% the .bib file (from which the .bbl file % is produced).
% REMEMBER HOWEVER: After having produced the .bbl file,
% and prior to final submission, you *NEED* to 'insert'
% your .bbl file into your source .tex file so as to provide
% ONE 'self-contained' source file.
%
% For tracking purposes - this is V2.0 - May 2012

\documentclass[10pt,journal,compsoc]{IEEEtran}

% IEEE Computer Society needs nocompress option                                                                                                                               
% requires cite.sty v4.0 or later (November 2003)
\usepackage[nocompress]{cite}

\usepackage{multirow}
\usepackage{graphicx}
\DeclareGraphicsExtensions{.pdf,.png}
\graphicspath{{./figures/}}

\usepackage{amssymb}
\usepackage{amsmath}
\usepackage{color}

\begin{document}


% DOI
%\doi{10.475/123_4}

% ISBN
%\isbn{123-4567-24-567/08/06}

%Conference
%\conferenceinfo{PLDI '13}{June 16--19, 2013, Seattle, WA, USA}

%\acmPrice{\$15.00}

%
% --- Author Metadata here ---
%\conferenceinfo{WOODSTOCK}{'97 El Paso, Texas USA}
%\CopyrightYear{2007} % Allows default copyright year (20XX) to be over-ridden - IF NEED BE.
%\crdata{0-12345-67-8/90/01}  % Allows default copyright data (0-89791-88-6/97/05) to be over-ridden - IF NEED BE.
% --- End of Author Metadata ---

%\title{\LARGE Bottleneck Identification in Cloud-Hosted Web Applications}
\title{\LARGE Detecting Performance Anomalies in Cloud Platform 
Applications}
\author{Hiranya~Jayathilaka,
        Chandra~Krintz,
        and~Rich~Wolski\\
Computer Science Department, Univ. of California, Santa Barbara}

\markboth{IEEE Transactions in Cloud Computing}%                                                                                                         
{Shell \MakeLowercase{\textit{et al.}}: Bare Demo of IEEEtran.cls for Computer Society Journals}

% for Computer Society papers, we must declare the abstract and index terms                                                                                                     
% PRIOR to the title within the \IEEEtitleabstractindextext IEEEtran                                                                                                            
% command as these need to go into the title area created by \maketitle.                                                                                                        
% As a general rule, do not put math, special symbols or citations                                                                                                              
% in the abstract or keywords.                                                                                                                                                  
\IEEEtitleabstractindextext{%                                                                                                                                                   
\begin{abstract}
%
%  Abstract
%

\begin{abstract}
\addcontentsline{toc}{chapter}{Abstract}

Abstract goes here

%\abstractsignature
\end{abstract}

\end{abstract}

% Note that keywords are not normally used for peerreview papers.                                                                                                               
\begin{IEEEkeywords}
Performance anomaly detection, Root cause analysis, Cloud computing
\end{IEEEkeywords}}

% make the title area                                                                                                                                                           
\maketitle

\IEEEdisplaynontitleabstractindextext

\IEEEpeerreviewmaketitle

% We no longer use \terms command
%\terms{Theory}

%\keywords{ACM proceedings; \LaTeX; text tagging}

\IEEEraisesectionheading{\section{Introduction}\label{sec:introduction}}
Cloud computing has revolutionized the way programmers develop and deploy applications.
By running computations and storing data on large, managed infrastructures and 
programming platforms, cloud computing ensures scalability and high-availability of 
user applications. Moreover, by offering such compute infrastructures and programming 
platforms as services, cloud computing is able to facilitate a cost-effective, on-demand
resource provisioning model that greatly enhances user productivity.

Over the last decade cloud computing technologies have enjoyed explosive growth 
and near universal adoption due to their many benefits and promises. Core cloud 
functionalities are available from a large and growing number of service providers. 
Some of these service providers offer hosted cloud solutions that can be used
via the web to deploy applications without installing any physical hardware 
(e.g. Amazon AWS, Google App Engine, Microsoft Azure). Others
provide cloud technologies as downloadable software, which users can install
on their computers or data centers to set up their own private clouds 
(e.g. Eucalyptus, AppScale, OpenShift). 

Both the IT industry and the academia have responded very positively to this new 
model of application 
development. Many business organizations are currently in the process of migrating
their IT infrastructure to the cloud. A large number of organizations
run their entire business as a cloud-based operation (e.g. Netflix, Snapchat). For startups
and academic researchers who don't have a large IT budget or a staff, the cost-effective 
on-demand resource provisioning model of the cloud has proved to be indispensable.
The growing number of academic conferences and journals dedicated to discussing
cloud computing is further evidence that cloud is an essential branch in the field
of computer science.

Despite its many benefits, cloud computing has also given rise to a number of application
development and maintenance issues that have gone unaddressed for many years. 
First and foremost, cloud platforms do not enforce any of the developer best practices
the software engineering community has established over the last half-century. This
includes code reuse, proper versioning of software artifacts, dependency management
between application components and backward compatible code updates. In
some cases organizations require certain conventions/standards to be enforced on
all production software. Cloud platforms do not provide any facilities for reaching
such administrative conformance on cloud-deployed applications. Instead, cloud platforms
are making it extremely simple and quick to deploy new applications or update existing
applications. The resulting speed-up of the development cycles combined with the lack of 
oversight and enforcement, makes it extremely difficult for 
IT personnel to manage large volumes of cloud-deployed applications.

This rapidly proliferating model of application development and IT operation
is designed to scale, both in the number of applications, 
and in the number of web services that must be
hosted and curated.  Each web service exports one or more Application Programming
Interfaces (APIs) that must be accessible by users, applications, and/or
other services.
Because applications encode their logic in terms of
remote ``calls'' to web services,
these APIs define functional boundaries that must be incorporated into the
application architecture.  

From an IT management perspective, the vast collection of web APIs must be
governed as critical infrastructure components or else ``client'' applications
will encounter problems when the APIs they depend on change or fail.  
Further, web services often make API calls to each other 
creating dependencies between APIs that must be carefully managed.
Moreover, the APIs have a software life cycle that is
independent and longer than the life cycle of the services themselves. 
As technological improvements make better service implementations possible, IT
management must upgrade these services while keeping the APIs unchanged.
Similarly, as new API versions emerge to increase functionality, backward
compatibility must be maintained in a way that takes into
account the strategic objectives of the organization.  

Today's cloud platforms severely lag behind in 
API governance. Typical cloud environments impose restrictions
on applications and web services in order to guarantee scalability
and high availability (e.g. Google App Engine~\cite{gae} prevents applications from accessing
the file system). 
However, they do not prevent the developers from violating
software development and maintenance policies. 
Consequently, developers often violate naming and versioning conventions
when naming digital assets, take dependencies
on incorrect or deprecated APIs, and very often end up re-implementing program logic
from the scratch instead of reusing extant APIs that provide the required functionality.
This lack of API governance also leads to many security vulnerabilities (e.g. DoS attacks
by malicious or poorly coded clients), violations of IP rights and licensing terms, and in
some cases, even financial losses. 

Web APIs also play a key role in managing and enforcing service level agreements (SLAs).
Ideally, any public API should be exposed with a well-defined
SLA detailing its performance and quality of service (QoS) characteristics. Without proper means
of API governance, developers that host services in cloud settings have no way of enforcing
competitive SLAs on their APIs, or even getting a thorough idea of what kind of SLAs
their services can uphold. Today, developers have to perform a lot of offline
and online testing to learn the performance characteristics of their APIs.
Furthermore, developers often have to be content with basic SLA monitoring 
as opposed to comprehensive SLA enforcement. 

{\bf My research focuses on designing and implementing systems for defining and
implementing 
API governance as a cloud-native feature.}  API management is often a
documentary function (e.g. ~\cite{apigee,layer7,wso2am} that does not include the ability to
enforce policies, particularly at scale in a cloud setting. 
My research agenda focuses on making API governance a ``first-class'' cloud
service that is fully integrated
with the core of the cloud and the cloud management tools. 
In a cloud-based IT context, APIs must be governed by
policies that are defined by the organization to ensure predictable
operation of the web services hosted by the cloud. Additionally,
there should be automated mechanisms in place for easily developing new
APIs, discovering existing APIs, analyzing the syntactic and semantic features
of APIs, and porting applications across APIs. My research will make 
it trivial for developers to consume, combine and leverage APIs to create powerful
applications, while adhering to tried and tested software engineering 
best practices. They will be able to quickly learn
the QoS properties of APIs and enforce competitive SLAs through the cloud
with clear and accurate levels of certainty. My plan is to
\begin{itemize}
%\item develop new {\bf policy specification languages for web
%services} that
%enables governance enforcement during service development, 
%and at runtime,
\item develop a distributed {\bf cloud-integrated service for enforcing IT 
management policies} governing
web APIs when operating at scale in cloud settings,
\item develop automated mechanisms for discovering, analyzing, comparing and reasoning
about web APIs, in order to {\bf simplify the API-based development model}, and,
\item develop methods for {\bf automated QoS and performance analysis} of web services
to formulate and enforce competitive SLAs for APIs hosted in clouds.
\end{itemize}

My research spans a number of prominent research areas in computer
science including programming languages, 
distributed systems and software engineering. The proposed policy enforcement
platform motivates new research on innovative new ways of using existing
distributed consensus, scalability and high availability techniques. It also calls for
more research in novel policy specification languages that
are developer-friendly, and can be verified and executed efficiently in cloud settings.
Automated performance analysis requires more research in using
static analysis methods (e.g. abstract interpretation, WCET analysis, etc.), system modeling,
and simulations for analyzing a wide range of services deployed in modern cloud platforms.
Above all, since the primary target of this
work are cloud platforms, there needs to be a strong focus towards making all these 
proposed mechanisms scale to handle thousands of APIs, policies, client applications and users.

\subsection{Outcomes and Assessment}

The outcome of this research plan,
I believe, will be new advances in services computing, IT management, cloud
computing, API policy enforcement, and SLA management.
If successful, the research will be transformative since APIs are rapidly
becoming the most valuable digital asset hosted in cloud settings
and my work enables scalable management of these assets.

In addition,
as a cloud systems researcher I intend to use open source, on-premise clouds (AppScale~\cite{krintzappscale13}
and Eucalyptus~\cite{eucalyptus09}) to produce research and educational artifacts (data sets, demos and test applications, 
code repositories, etc.) to my colleagues, 
and the wider research community.  
I also plan to make my systems, tools and mechanisms available as persistent software artifacts to
the wider research community and to monitor its uptake.  Because of their design
and their use of open source cloud technologies, results of my research will be readily available
to researchers and educators who wish to develop new artifacts and curricula
for cloud computing. 


\section{Background}
\label{sec:background}
\section{Cloud Computing}

\textbf{Cloud computing is a form of distributed computing that turns compute infrastructure, programming
platforms and software systems into scalable utility services~\cite{hassan2011demystifying,Mell:2011:SND:2206223}.}
By exposing various compute and programming
resources as utility services, cloud computing promotes resource sharing at scale via the Internet.
The cloud model precludes the users from having to set up their own hardware, and in some cases also software. Instead,
the users can simply acquire the resources ``in the cloud'' via the internet, and relinquish them when
the resources are no longer needed. The cloud model also does not require the users to spend any start up
capital. The users only have to pay for the resources they acquired, usually based on a pay-per-use
billing model.

\textbf{Depending on the type of resources
offered as services, cloud computing platforms can be categorized into three main categories~\cite{Mell:2011:SND:2206223}.}
\begin{description}
\item [Infrastructure-as-a-Service clouds (IaaS)]
Offers low-level compute, storage and networking
resources as a service. Compute resources are typically provided in the form of on-demand virtual machines (VMs)
with specific CPU, memory and disk configurations (e.g. Amazon EC2, Google Compute Engine, Eucalyptus). 
The provisioned VMs usually come with a base operating system installed. The users must install all the application software
necessary to use them.
\item [Platform-as-a-Service clouds (PaaS)]
Offers a programming platform as a service, that can be used to develop and deploy applications at scale 
(e.g. Google App Engine, AppScale, Heroku, Amazon Elastic Beanstalk). The programming platform consists
of several scalable services that can be used to obtain certain application features such as data storage, caching
and authentication.
\item [Software-as-a-Service clouds (SaaS)]
Offers a collection of software applications and tools as a service, that can be directly consumed by
application endusers (e.g. Salesforce, Workday, Citrix go2meeting). This can be thought of as a new way 
of delivering software to
endusers. Instead of prompting the users to download and install any software, SaaS enables the users
to consume software via the Internet.  
\end{description}

Due to these benefits associated with cloud computing (scalability, high availability, productivity enhancement etc.),
many developers and organizations use the cloud as their preferred means of developing and deploying
software applications. \textbf{Cloud-hosted applications expose one or more web application programming 
interfaces (web APIs) through which client programs can remotely interact with the applications.} That is, clients
send HTTP/S requests to the API, and receive machine readable responses (e.g. HTML, JSON,
XML, Protocol Buffers~\cite{protobuff}) in return. This type of web-accessible, cloud-hosted applications
tend to be highly interactive, and clients have strict expectations on the application
response time~\cite{latency-matters}.

\textbf{A cloud-hosted application may
also consume web APIs exposed by other cloud-hosted applications.} Thus, cloud-hosted applications
form an intricate graph of inter-dependencies among them, where each application can service a set of client
applications, while being dependent on a set of other applications. However, in general, each cloud-hosted
application directly depends on the core services offered by the underlying cloud platform for compute power, storage,
network connectivity and scalability.

In the next section we take a closer look at a specific type of cloud platforms -- Platform-as-a-Service clouds.
We use PaaS clouds as a case study and a testbed in a number of our explorations.

\section{Platform-as-a-Service Clouds}

\textbf{PaaS clouds, which have been growing in popularity~\cite{paas-growth,paas-growth2}, 
typically host web-accessible (HTTP/S) applications, to which they provide
high levels of scalability, availability, and sandboxed execution.} PaaS clouds
provide scalability by automatically allocating resources for
applications on the fly (auto scaling), and provide availability through the
execution of multiple instances of the application. Applications deployed on
a PaaS cloud depend on a number of scalable services intrinsic to the 
cloud platform. We refer to these services as \textit{kernel services}.

\textbf{PaaS clouds, through their kernel services, provide a high level of
abstraction to the application developer that effectively hides all the
infra\-structure-level details such as physical resource allocation (CPU,
memory, disk etc), operating system, and network configuration.} 
Moreover, PaaS clouds do not require the developers
to set up any utility services their applications might require such as a 
database or a distributed cache. 
Everything an application requires is provisioned and managed by the PaaS cloud.
This enables
application developers to focus solely on the programming aspects of their
applications, without having to be concerned about deployment issues. On
the other hand, the software abstractions provided by PaaS clouds obscure
runtime details of applications making it difficult to reason about application
performance, and diagnose performance issues.

\begin{figure}
\centering
\includegraphics[scale=0.5]{paas_architecture}
\caption{PaaS system organization.}
\label{fig:paas_architecture}
\end{figure}

Figure~\ref{fig:paas_architecture} shows the key layers of a typical PaaS cloud. Arrows indicate
the flow of data and control in response to application requests.
\textbf{At the lowest level of a PaaS cloud is an infrastructure that consists of the necessary compute, storage
and networking resources.} How this infrastructure is set up may vary from a simple cluster of physical 
machines to a comprehensive Infrastructure-as-a-Service (IaaS) cloud. In large scale PaaS clouds,
this layer typically consists of many virtual machines and/or containers with the ability to acquire more
resources on the fly.

\textbf{On top of the infrastructure layer lies the PaaS kernel -- a collection of managed, scalable
services that high-level application developers can compose into their applications.} The provided kernel services
may include database services, caching services, queuing services and more. 
The implementations of the kernel services are highly scalable, highly available (have SLOs associated with them),
and automatically managed by the platform while being completely opaque
to the application developers. Some PaaS clouds
also provide a managed set of programming APIs (a ``software development
kit'' or SDK) for the application developer to access these kernel services. 
In that case all interactions between the applications and the PaaS kernel must take place through
the cloud provider specified SDK (e.g. Google App Engine~\cite{gae}, Microsoft Azure~\cite{azure}). 

\textbf{One level above the PaaS kernel we find the application servers that are used to deploy and run
applications.} Application servers provide the necessary integration (linkage) between application code and the
PaaS kernel services, while sandboxing application code for secure, multi-tenant execution. They also
enable horizontal scaling of applications by running the same application on multiple application server
instances.

\textbf{On top of the application servers layer resides the request
front-end and load balancing layer.} This layer is responsible
for receiving all application requests, filtering them, and routing them to an appropriate application
server instance for further execution. Front-end server is therefore the entry point for PaaS-hosted
applications for all application clients.

\textbf{Each of the above layers can span multiple processes, running over multiple physical or virtual
machines.} Therefore processing a single application request typically involves cooperation
of multiple distributed processes and/or machines.

\begin{figure}
\centering
\includegraphics[scale=0.4]{cloud_app_model2}
\caption{Applications deployed in a PaaS cloud: (a) An external client making requests
to an application via the web API;
(b) A PaaS-hosted application invoking another in the same cloud.
\label{fig:cloud_app_model}
}
\end{figure}

Figure~\ref{fig:cloud_app_model} illustrates the application deployment model of PaaS clouds. 
The cloud platform provides a set of kernel services. 
The PaaS SDK provides well defined interfaces (entry points) for these kernel services.  
\textbf{The developer uses the kernel services, via the SDK, to implement his/her application logic, and packages 
it as a web application.} Developers then
upload their applications to the cloud for deployment.
Once deployed, the applications and any web APIs exported by them can be accessed 
via HTTP/S requests by external or co-located clients.

\textbf{PaaS clouds are specifically built for deploying and running applications
that are directly consumed by end users and other client applications.} As a result all the problems 
outlined in the previous chapter, such as poor development practices, lack of performance SLOs, and lack of performance 
debugging support directly impact PaaS clouds. Therefore PaaS clouds are ideal candidates for implementing the type
of governance systems proposed in this work. 

\textbf{We use PaaS clouds in our research extensively both as case studies
and experimental platforms.} Specifically, we use Google App Engine and AppScale as test environments
to experiment with our new governance systems. App Engine is a highly scalable public PaaS cloud hosted and
managed by Google in their data centers. While it is open for anyone to deploy and run web applications, it is not
open source software, and its internal deployment details are not commonly known. AppScale is open source
software that can be used to set up a private cloud platform on one's own physical or virtual hardware. AppScale
is API compatible with App Engine (i.e. it supports the same cloud SDK), and hence any web application developed
for App Engine can be deployed on AppScale without any code changes. In our experiments, we typically deploy
AppScale over a small cluster of physical machines, or over a set of virtual machines provided by an IaaS cloud
such as Eucalyptus.

\textbf{By experimenting with real world PaaS clouds we demonstrate the practical feasibility and the effectiveness of 
the systems we design and implement.} Furthermore, there are currently over a million applications deployed
in App Engine, with a significant proportion of them being open source applications. Therefore we have access
to a large number of real world PaaS applications to experiment with.

\textbf{PaaS-hosted applications are typically developed and tested outside the cloud (on a developer's workstation), 
and then later uploaded to the cloud.} Therefore PaaS-hosted applications typically undergo three phases 
during their life-cycle:
\begin{description}
\item[Development-time] The application is being developed and tested on a developer's workstation
\item[Deployment-time] The finished application is being uploaded to the PaaS cloud for deployment
\item[Run-time] Application is running, and processing user requests
\end{description}
We explore ways to use these different phases to our advantage in order to minimize the governance
overhead on running applications. 

\section{Governance}
\subsection{IT and SOA Governance}
\textbf{Traditionally, information and technology (IT) governance~\cite{brown2005framing} has been a branch of 
corporate governance, focused on improving performance and managing the risks associated with the use of IT.} 
A number of frameworks, models and even certification systems have emerged over time to help organizations 
implement IT governance~\cite{Ataya:2013:ISR:2523514.2523590,gov-cert}. 
The primary goals of IT governance are three fold.

\begin{itemize}
\item Assure that the use of IT generates business value
\item Oversee performance of IT usage and management
\item Mitigate the risks of using IT
\end{itemize}

\textbf{When the software engineering community started gravitating towards web services and
service-oriented computing (SOC)~\cite{1254461, what-is-soa, Haines:2010:SAM:1787234.1787269}, 
a new type of digital assets rose to prominence within corporate IT 
infrastructures -- ``services''.} A service is a self-contained entity that logically represents a
business activity (a functionality; e.g. user authentication, billing, VM management) 
while hiding its internal implementation details from the consumers~\cite{what-is-soa}. 
Compositions of loosely-coupled, reusable, modular services soon replaced 
large monolithic software installations. 

\textbf{Services required new forms of governance for managing their performance
and risks, and hence the notion of service-oriented architecture (SOA) governance 
came into existence~\cite{gartner-soa-gov,soagov}.} 
Multiple definitions of SOA governance
are in circulation, but most of them agree that the purpose of SOA governance is to exercise control over
services and associated processes (service development, testing, monitoring etc). A commonly used definition
of SOA governance is ensuring and validating that service artifacts within the architecture are operating
as expected, and maintaining a certain level of quality~\cite{gartner-soa-gov}.
Consequently, a number of tools that help organizations implement SOA governance 
have also evolved~\cite{Schepers:2008:LAS:1363686.1363932,4730489,6478236,5577268}.
Since web services are the most widely used form of services in SOA-driven systems, most of these
SOA governance tools have a strong focus on controlling web services~\cite{6094008}. 

\textbf{Policies play a crucial role in all forms of governance.} A policy is a specification of the acceptable behavior
and the life cycle of some entity. The entity could be a department, a software system, a service or a 
human process such as developing
a new application. In SOA governance, policies state how services should be developed, how they are to be
deployed, how to secure them, and what level of quality of service to maintain while a service is in operation.
SOA governance tools enable administrators to specify acceptable service behavior and life cycle as policies, and
a software policy enforcement agent automatically enacts those policies to control various aspects of the 
services~\cite{5976827,4483228,4279691}. 

\subsection{Governance for Cloud-hosted Applications}
Cloud computing can be thought of as a heightened version of service-oriented computing. While classic
SOC strives to offer data and application functionality as services, cloud computing offers a variety
of computing resources
as services, including hardware infrastructure (compute power, storage space and networking) and programming
platforms. Moreover, the applications deployed on cloud platforms typically behave like services with
separate implementation and interface components. 
Much like classic services, each cloud-hosted application 
can be a dependency for another
co-located cloud application, or a client application running elsewhere (e.g. a mobile app). 

Due to this resemblance, we argue that many concepts related to SOA governance are
directly applicable to cloud platforms and cloud-hosted applications. 
We extend the definition of SOA governance, and define governance for cloud-hosted applications
as the process of ensuring that the cloud-hosted applications
operate as expected while maintaining a certain quality of service level.
This is a broad definition that allows room for many potential avenues of research.
In our work we explore three specific features of governance as they apply to cloud-hosted applications.

\begin{description}
\item [Policy enforcement]
Policy enforcement refers to ensuring that all applications deployed in a cloud platform
adhere to a set of policies specified by a platform or organizational administrator.
Some of these policies include specific
dependency management practices, naming and packaging standards for software artifacts, 
software versioning requirements, and practices that enable software artifacts to evolve 
while maintaining backward compatibility.
Others specify run-time constraints, which need to be enforced per application request.

\item [Formulating performance SLOs]
This refers to automatic formulation of statistical bounds on the 
performance of cloud-hosted web applications.
A service level objective (SLO) specifies a system's minimum quality of service (QoS) level in a measurable and
controllable manner~\cite{smj2000}. They may cover various QoS
parameters such as availability, response time (latency), and throughput. A performance SLO
specifies an upper bound on the application's response time, and the likelihood that bound is valid.

\item [Application performance monitoring]
Application performance monitoring (APM) refers to continuously monitoring cloud-hosted applications
to detect violations of performance SLOs and other performance anomalies. 
It also includes diagnosing the root cause of each detected anomaly, thereby expediting
remediation.
\end{description}

None of the above features are implemented satisfactorily in the
cloud technologies available today.
In order to fill the gaps caused by these limitations, many third-party governance solutions that operate as external services
have come into existence. For example, services like 3Scale~\cite{3scale}, Apigee~\cite{apigee} and Layer7~\cite{layer7} provide a wide range
of access control and API management features for web applications served from cloud platforms. Similarly, 
services like New Relic~\cite{newrelic}, Dynatrace~\cite{dynatrace} and Datadog~\cite{datadog} provide monitoring support for cloud-hosted 
applications. But these services are expensive, and require additional programming and/or configuration.
Some of them also require changes to applications in the form of code instrumentation. Moreover,
since these services operate outside the cloud platforms they govern, they have limited visibility and control
over the applications and related components residing in the cloud. A main goal of our research is to facilitate governance
from within the cloud, as an automated, cloud-native feature. We show that such built-in governance capabilities are
more robust, effective and easy to use than external third-party solutions that overlay governance on top of the
cloud.

\subsection{API Governance}
A cloud-hosted application is comprised of two parts -- implementation and interface. The implementation
contains the functionality of the application. It primarily consists of code that implements
various application features. 
The interface, which abstracts and modularizes the implementation details of the application while making
it network-accessible, is often referred to as a \textit{web API} (or API in short). 
The API enables remote users and client applications to interact with the application by sending
HTTP/S requests. The responses generated by an API could be based on HTML (for display on a web
browser), or they could be based on a data format such as XML or JSON (for machine-to-machine 
interaction). Regardless of the technology used to implement an API, it is the part of the application 
that is visible to the remote clients. 

Developers today
increasingly depend on the functionality of already existing web applications in the cloud, which
are accessible through their APIs. Thus, a modern application 
often combines local program logic with calls to remote web APIs. 
This model significantly reduces both the programming and
the maintenance workload associated with applications. In theory, because
the APIs interface to software that is curated by cloud providers, the client
application leverages greater scalability, performance, 
and availability in the implementations it calls upon through these APIs, than
it would if those implementations were local to the client application
(e.g. as locally available software libraries).
Moreover, by accessing shared web applications, developers avoid ``re-inventing the
wheel'' each time they need a commonly available application feature. The scale at
which clouds operate ensures that the APIs can support the large volume
of requests generated by the ever-growing client population.

As a result, web-accessible APIs and the software applications to which
they provide access are rapidly proliferating. At the time of this writing, 
ProgrammableWeb~\cite{pweb}, a popular web API index, lists more than $15,000$
publicly available web APIs, and a nearly 100\% annual growth rate~\cite{pweb_growth}.
These APIs increasingly employ the REST (Representational State Transfer) architectural style~\cite{Fielding:2000:ASD:932295}, and 
many of them target commercial applications (e.g. advertising, shopping, travel, etc.).
However, several non-commercial entities have also recently published web 
APIs, e.g. IEEE~\cite{ieeeapis}, UC Berkeley~\cite{ucbapis}, and the US White
House~\cite{whitehouseapis}. 

This proliferation of web APIs in the cloud demands new techniques that
automate the maintenance and evolution of APIs as a first-class software
resource -- a notion that we refer
to as \textit{API governance}~\cite{6903538}. A poorly implemented API may render a whole application unusable.
A backward incompatible change to an API, can break any downstream applications dependent on it. 
Therefore it is important to be able to configure and enforce governance policies at the granularity of
APIs. Similarly, we need to be able to stipulate performance SLOs for individual APIs, and monitor
them as separate independent entities.
API management in the form of run-time mechanisms to implement
access control is not new, and many good commercial offerings exist today~\cite{3scale,apigee,layer7}.   
However, API governance -- consistent, generalized, policy
implementation across multiple APIs in an administrative domain --
is a new area of research made poignant by the emergence of cloud computing.

\section{Roots}
\label{sec:arch}
Roots is a holistic system for application performance monitoring (APM), 
performance anomaly detection, and root cause analysis.
It is operated by the cloud providers as a builtin PaaS service that collects data from
all the cloud components user applications interact with. Data collection, storage
and analysis all take place within the cloud, and the insights gained are communicated
to both the cloud administrators and application developers as needed.
The key intuition behind Roots is that, as an intrinsic PaaS service, Roots
has visibility into all activities of the PaaS cloud, across layers.
Moreover, since the PaaS applications we have observed spend most of their time in 
PaaS kernel services~\cite{Jayathilaka:2015:RTS:2806777.2806842}, we hypothesize
that we can infer application performance from observations of how
 the application uses the platform, i.e. by efficiently monitoring the time spent in 
PaaS kernel services. If we are able to do so, then we can avoid application
instrumentation and its downsides, while detecting performance anomalies and 
identifying their root cause quickly and accurately.

The PaaS model that we assume with Roots is one 
in which the clients of a web application engage in a
``service-level agreement'' (SLA)~\cite{Keller:2003:WFS:635430.635442}
with the ``owner'' or operator of the application that is hosted in a PaaS cloud.  The SLA
stipulates a response-time ``service-level objective'' (SLO) that, if violated, 
constitutes a breech of the agreement.
If the performance of an application deteriorates to the
point that at least one of its SLOs is violated, we treat it 
as an \textit{anomaly}. Moreover, we refer to the process
of diagnosing the reason for 
an anomaly as \textit{root cause analysis}.
For a given anomaly, the root cause could be a change in the application workload or
a \textit{bottleneck} in the application runtime. Bottlenecks may occur in the 
application code, or in the PaaS kernel services that the application relies on.

Roots collects performance data across the cloud platform stack, and aggregates it based on 
request/response.  It uses this data to infer application performance, and to identify
SLO violations (performance anomalies).  Roots can further handle different types of anomalies
in different ways.  We overview each of these functionalities in the remainder of this section.

\subsection{Data Collection and Correlation}

We must address two issues when designing a monitoring framework for
a system as complex as a PaaS cloud.
\begin{enumerate}
\item Collecting data from multiple different layers.
\item Correlating data collected from different layers.
\end{enumerate}

%Any PaaS APM must (i) collect data from all layers of the PaaS software stack, and
%(ii) correlate related events across layers.
Each layer of the cloud platform is only able to collect data regarding the
state changes that are local to it. A layer cannot monitor state changes
in other layers due to the level of encapsulation provided by layers. However,
processing an application request involves cooperation of multiple layers. 
To facilitate system-wide monitoring and
bottleneck identification, we must gather data from all the different layers involved
in processing a request. To combine the information across layers
we correlate the data, and link events related to the same request together.

To enable this, we augment the front-end server of the cloud platform. 
Specifically, we have it tag incoming application requests with unique identifiers.
This request identifier is added to the HTTP request as a header, which is visible to all 
internal components of the PaaS cloud. Next, we configure data collecting agents 
within the platform to record the request identifiers along with any events they capture. 
This way we record the relationship between application requests, and the resulting
local state changes in different layers of the cloud, without breaking the existing level
of abstraction in the cloud architecture. This approach is also scalable, since the events are
recorded in a distributed manner without having to maintain any state at the data collecting agents. 
Roots aggregates the recorded events by request 
identifier to efficiently group the related events as required during analysis.

\begin{figure}
\centering
\includegraphics[scale=0.5]{apm_architecture}
\caption{Roots APM architecture.}
\label{fig:apm_architecture}
\end{figure}

Figure~\ref{fig:apm_architecture} illustrates the high-level architecture of Roots, and how 
it fits into the PaaS stack. APM components are shown in grey. 
The small grey boxes attached to the PaaS components represent the
agents used to instrument the cloud platform. 
In the diagram, a user request is tagged with the identifier value
$R$ at the front-end server. This identifier is passed down to the lower layers of the cloud
along with the request. Events that occur in the lower layers while processing this request
are recorded with the request identifier $R$, so Roots can correlate them later. For example, in the 
data analysis component we can run a filter query to select all the events related to a particular
request (as shown in the pseudo query in the diagram). Similarly, Roots can run a ``group by'' 
query to select all events, and aggregate them by the request identifier.

The figure also depicts Roots data collection across the
PaaS stack (i.e. its full stack monitoring). 
From the front-end server, Roots collects
information related to incoming application
requests. It does so by scraping HTTP server access logs, which are
exported by most web servers (e.g. Apache HTTPD or Nginx). 

At the application server level, Roots collects logs and 
metrics related to the application runtime from 
the application servers and operating system.
Roots also employs a set of per-application benchmarking 
processes that periodically probes 
different applications
to measure their performance. These are lightweight, stateless processes 
managed by the Roots framework.
These processes send their measurements to
the data storage component for analysis.

Roots collects information about all kernel invocations
made by the applications by intercepting kernel invocations at 
service interface entrypoints.  For each PaaS kernel invocation, 
we capture the following parameters.
\begin{itemize}
\item Source application making the kernel invocation
\item Timestamp
\item A sequence number indicating the order of PaaS kernel invocations within an application request
\item Target kernel service and operation
\item Execution time of the invocation
\item Request size, hash, and other parameters
\end{itemize}
These PaaS kernel invocation details enable Roots
to trace the execution of application 
requests through the PaaS without instrumenting the application itself.

Finally, at the lowest level Roots collects information 
related to virtual machines, containers
and their resource usage. We gather metrics on network usage 
by individual components which
is useful for traffic engineering use cases. 
We also scrape
hypervisor and container manager logs to track when
resources are allocated and released.

To avoid introducing delays to the application 
request processing flow, we implement
Roots data collecting agents as asynchronous tasks. 
%That is, none of them 
%suspend application request processing to report data to the data storage components.
Agents buffer data locally and periodically write to
data storage components using separate background tasks 
and batch communication
operations. These persistence operations must run 
with sufficient frequency so as to not impede the analysis
that Roots employs to 
detect anomalies soon after they occur.
%We also isolate the activities in the cloud platform from potential
%failures in the Roots data collection or storage components.

\subsection{Data Storage and Analysis}

Roots stores all collected data in a database capable of 
efficient persistent storage
and querying. We facilitate this via indexing
data by application ID and timestamp.
Roots also performs periodic garbage collection on data that is no longer
pertinent to analyses.

The data analysis components consist of two extensible
abstractions: \textit{anomaly detectors} 
and \textit{anomaly handlers}.
Anomaly detectors are processes that periodically 
analyze the data for
each deployed application. 
Roots supports multiple detector implementations, 
each of which is a statistical method for detecting 
performance anomalies. Detectors are configured
on a per-application basis, making it possible for different applications to use different anomaly 
detectors. Roots also supports multiple concurrent anomaly detectors for 
the same application, which can be used
to compare the efficacy of different detection strategies concurrently.
Each anomaly detector has configurable parameters for execution schedule 
and sliding window duration.
We use a period 60 seconds for the former and the previous
hour for the latter, in our prototype and evaluation.
Window size impacts the time range of events processed
by the detector when invoked.
We employ a fixed-size window to bound Roots memory use.

When an anomaly detector detects an anomaly 
in application performance, it sends an event
to a collection of anomaly handlers. 
The event encapsulates a unique anomaly identifier, 
timestamp, application identifier and the source detector's sliding window that correspond to the
anomaly. Anomaly handlers are configured globally (i.e. each handler
receives events from all detectors), but each handler filters events
of interest.
Handlers can also trigger events, which are delivered to
all the listening anomaly handlers. Similar to detectors, 
Roots supports multiple anomaly handler
implementations, e.g., one for logging anomalies, one for sending alert emails, one
for updating a dashboard, etc. 
Additionally, Roots provides two special anomaly handlers:
a workload change analyzer and a bottleneck identifier.
Communication between detectors and handlers 
is performed via shared memory.

The ability of anomaly handlers to filter the events they 
process and to trigger events directly
facilitates construction of 
elaborate event flows with sophisticated logic. For example, the workload
change analyzer can run some analysis upon receiving an anomaly event
from any anomaly detector. If an anomaly cannot be associated 
with a workload
change, it can trigger a different type of event. 
The bottleneck identifier, can
be configured to execute only when such an event occurs.
Using this mechanism, Roots performs workload change analysis first
and systemwide bottleneck identification only when necessary.

%Both the anomaly detectors and anomaly handlers 
%work with fixed-sized sliding windows.
%Therefore, the amount of state these entities must keep in memory has
%a strict upper bound. 
%The extensibility of Roots is primarily achieved through the abstractions of anomaly
%detectors and handlers. Roots makes it simple to implement new detectors and handlers,
%and plug them into the system. Both the detectors and the handlers are executed
%as lightweight processes that do not interfere with the rest of the processes in
%the cloud platform. 

\subsection{Roots Process Management}
\label{sec:process_mgt}

\begin{figure}
\centering
\includegraphics[scale=0.45]{roots_pod}
\caption{Anatomy of a Roots pod. The diagram shows 2 application benchmarking processes (B), 
3 anomaly detectors (D), and 2 handlers (H). Processes communicate via a shared
memory communication bus local to the pod.}
\label{fig:roots_pod}
\end{figure}
Most data collection activities in Roots can be treated as passive -- i.e. they
happen automatically as the applications receive and process requests in the cloud
platform. They do not require explicit scheduling or management. In contrast,
application benchmarking and data analysis are active processes that require
explicit scheduling and management.  This is achieved by grouping benchmarking
and data analysis processes into units called Roots pods. 

Each Roots pod is responsible for starting and maintaining a preconfigured set of
benchmarkers and data analysis processes (i.e. anomaly detectors and handlers). 
These processes are light enough, so as to pack a large number of them
into a single pod. Pods are self-contained entities, and there is no inter-communication
between pods. Processes in a pod can efficiently communicate with each other 
using shared memory, and call out to the central Roots data storage to retrieve 
collected performance data for analysis. 
%This enables starting and stopping 
%Roots pods with minimal impact on the overall monitoring system. 
Furthermore, pods
can be replicated for high availability, and application load can be distributed
among multiple pods for scalability.

Figure~\ref{fig:roots_pod} illustrates a Roots pod monitoring two applications.
It consists of two benchmarking processes, three anomaly detectors and 
two anomaly handlers. The anomaly detectors and handlers are shown communicating
via an internal shared memory communication bus. 

%To automate the process of managing pods, they can be tied into the core
%process management framework of the PaaS cloud. That way whenever the cloud
%platform initializes, a collection of pods can be started automatically.
%Application deployment process of the PaaS cloud can be augmented
%to register each new application with one of the available pods, so that the
%benchmarkers and anomaly detectors can start running on the application.
%Moreover, pods can be moved around or restarted as needed in response
%to errors and autoscaling events that occur in the cloud platform.


\section{Prototype Implementation}
\label{sec:impl}
We implemented a Roots prototype for the AppScale cloud platform. AppScale is an open
source PaaS cloud, API compatible with the popular Google App Engine (GAE) cloud platform.
Due to the API compatibility, any application developed for GAE can be deployed
and executed on AppScale with no code changes. Since AppScale is open source software, we
were able to modify parts of its implementation to integrate the Roots APM into it. AppScale also
turned out to be a great test environment since we could deploy it locally, and run a number of
existing App Engine applications on it.

We use ElasticSearch as the data storage component of our prototype. ElasticSearch is ideal 
for storing large volumes of structured and semi-structured data. It supports scalability and 
high availability via sharding and replication.
ElasticSearch continuously organizes and indexes data, making the information available 
for fast retrieval and efficient querying. Additionally it also provides
powerful data filtering and aggregation features, which greatly simplify the implementations of high-level
data analysis algorithms.

We configure AppScale's front-end server (based on Nginx) to tag all incoming application requests
with a unique identifier. This identifier is attached to the request as a custom HTTP header.
All data collecting agents in the cloud extract this identifier, and include it as an attribute
in all the events reported to ElasticSearch. This enables our prototype to aggregate events originating
from the same application.

We implement a number of data collecting agents in AppScale to gather runtime information
from all major components. These agents buffer data locally, and store them in ElasticSearch
in batches. For scraping server logs and storing the extracted entries in ElasticSearch,
we use the Logstash tool. Logstash supports scraping a wide range of standard log formats (e.g. 
Apache HTTPD access logs), and other custom log formats can be supported via a simple configuration.
It also integrates naturally with ElasticSearch.
To capture the PaaS kernel invocation data, we augment AppScale's PaaS SDK implementation,
which is derived from the GAE PaaS SDK. More specifically we implement an agent that records
all PaaS SDK calls, and reports them to ElasticSearch asynchronously. 

Roots pods are implemented as standalone Java server processes. Threads are used to run benchmarkers,
anomaly detectors and handlers concurrently within each pod. Pods communicate with ElasticSearch via
REST calls, and many of the data analysis tasks such as filtering and aggregation are performed
in ElasticSearch itself. By doing so a lot of the heavy computations are offloaded to the 
ElasticSearch cluster, which is specifically designed for high-performance query processing
and analytics. Some of the more sophisticated statistical analysis tasks (e.g. change point detection, 
linear regression) are implemented in R language,
and the Java-based Roots pods integrate with R using the Rserve protocol.

\subsection{SLO-based Anomaly Detector}
We implement a performance SLO checker as the primary anomaly detector in Roots. This anomaly detector
allows application developers to specify simple performance SLOs for deployed applications. A
performance SLO consists of an upper bound on the application response time ($T$), and the probability ($p$)
that the application response time falls under the specified upper bound. Therefore, a general performance 
SLO can be stated as: ``application responds under $T$ milliseconds $p$\% of the time''.

When activated for a given application this anomaly detector starts a benchmarking process
that periodically measures the response time of the target application. The detector then periodically
analyzes the collected response time measurements to check if the application meets the specified performance
SLO. Whenever it detects that the application has failed to meet the SLO, it triggers an anomaly event. It would also
purge any past response time measurements (up to the anomaly) from the memory 
when this happens.

The SLO-based anomaly detector supports following configuration parameters:
\begin{itemize}
\item Performance SLO: Response time upper bound ($T$), and the probability ($p$).
\item Sampling rate: Rate at which the target application is benchmarked.
\item Analysis rate: Rate at which the anomaly detector checks whether the application has failed to meet the SLO.
\item Minimum samples: Minimum number of samples to collect before checking for SLO violations.
\item Window size: Length of the sliding window (in time) to consider when checking for SLO violations. This
acts as a limit on the number of samples to keep in memory.
\end{itemize}

Since the detector purges past measurements when an anomaly is detected, it is not able to
check for further SLO violations until the minimum number of samples is collected. Therefore,
each anomaly is followed by a ``warm up'' period. With a sampling rate of 15 seconds, and a minimum
samples count of 100, the warm up period can last up to 25 minutes. The detector cannot find new
anomalies during this period. However, it prevents the same anomaly from being needlessly
reported multiple times.

\subsection{Path Distribution Analyzer}
Path distribution analyzer is another special anomaly detector we implement in Roots. This
anomaly detector periodically analyzes the PaaS kernel invocations made by the applications.
By aggregating the PaaS kernel invocations by application request identifiers, and then sorting them by
their sequence numbers, this anomaly detector is able to identify the sequence of
PaaS kernel invocations made by each application request. 
Each identified invocation sequence corresponds to a path of
execution through the application code (i.e. a path through the control flow graph of the application). 
Then the anomaly detector evaluates the number of requests
that invoked the same PaaS kernel invocation sequence. From that the anomaly detector
computes the distribution of different execution paths of an application.

A path distribution is comprised of the set of execution paths available in an application, along
with the proportion of requests that executed each path.
It is an indicator of the type of request workload handled by an application.
For example consider a data management application that has a read-only execution path, and a read-write 
execution path. If 90\% of the requests execute the read-only path, and the remaining 10\% of the requests
execute the read-write path, we may characterize the request workload as mostly read-only. 
Roots path distribution analyzer facilitates computing the path distribution for each application
with no static analysis or runtime instrumentation on the application code.

Roots path distribution analyzer periodically computes the path distribution for a given application.
If it detects that the latest path distribution is significantly different from the distributions seen in the 
past, it triggers an anomaly. This is done by computing the mean request proportion for each path
(over a sliding window of historical data),
and then comparing the latest request proportion values against the means. If the latest proportion
is off by more than $n$ standard deviations from its mean, the detector considers it to be an
anomaly. The sensitivity of the detector can be configured by changing the value of $n$, which
defaults to 2. 

This anomaly detector enables developers to know when the nature of their application request
workload changes. For example in the previous data management application, if suddenly 90\%
of the requests start executing the read-write path, the Roots path distribution analyzer will
detect the change as an anomaly. Similarly it is also able to detect when new paths of execution
are being invoked by requests (a form of novelty detection).

\subsection{Workload Change Analyzer}
Performance anomalies can arise due to bottlenecks in the cloud platform or changes in the application
workload.
When Roots detects a performance anomaly (e.g. an application failing to meet its performance SLO),
we need to be able to determine which of the above two causes may be behind it.
To check if the workload of an application has changed recently, Roots uses a workload change analyzer.
This is implemented as an anomaly handler, which gets executed every time an anomaly detector
notifies of a performance anomaly. Note that this is different from the path distribution analyzer,
which is implemented as an anomaly detector. While the path distribution analyzer looks for changes in the
\textit{type} of the workload, the workload change analyzer looks for changes in the workload \textit{size}.
In other words, it determines if the target application has received more requests than usual, which
may have caused a performance degradation.

Workload change analyzer uses change point detection algorithms to analyze the historical trend of 
the application workload. We use the ``number of requests
per unit time'' as the metric of workload size. This information can be obtained from the Roots
data storage as a time series. Our implementation of Roots supports a number of well known change point
detection algorithms (PELT, binary segmentation and CL method), any of which can be used to detect level shifts in the
workload size. These algorithms favor long lasting shifts (plateaus) in the workload trend, over momentary spikes.
We expect momentary spikes to be fairly common in workload data. But it's the plateaus that cause
request buffers to fill up, and hog server-side resources for extended periods of time thus
causing noticeable performance anomalies.

\subsection{Bottleneck Identification}
This is the primary root cause identification mechanism in our implementation of Roots. Like the workload
change analyzer, this is also implemented
as an anomaly handler. When a performance anomaly is detected, this mechanism attempts to find the
most likely component in the PaaS kernel that may have caused the anomaly. AppScale PaaS kernel
consists of the same core services present in the Google App Engine public cloud (datastore, memcache,
urlfetch, blobstore, user management etc.). The purpose of bottleneck identification is to find, out of all
the PaaS kernel services used in an application, the one service that is most likely to have caused 
application performance to deteriorate. We use a fine grained approach so that when an application
invokes the same service multiple times, our implementation can pinpoint the exact service call that
contributed towards the performance anomaly. If none of the service calls cannot be associated with
the anomaly, we attribute the anomaly to the rest of the application code executed
directly in the application server.

Suppose an application makes $n$ PaaS kernel invocations ($X_1, X_2, ... X_n$). For each user request processed
by the application Roots captures the time spent on each kernel invocation ($T_{X1}, T_{X2}, ... T_{Xn}$), and the 
total response time ($T_{total}$) of the request. These time values are related by the formula
$T_{total} = T_{X1} + T_{X2} + ... + T_{Xn} + r$, where $r$ is all the time spent not invoking any PaaS kernel
services. $R$ is mainly the time spent in the resident application server executing user code, and it is not
directly measured in Roots. In our previous
work we have shown that PaaS-hosted web applications spend most of their time invoking PaaS kernel services.
Therefore we have $r \ll T_{X1} + T_{X2} + ... + T_{Xn}$.

Roots bottleneck identification mechanism uses above parameters to compute four metrics. These four metrics
are then further evaluated by a weighted algorithm to determine the bottleneck in the cloud platform. We
first describe the four metrics computed by Roots.

\subsubsection{Relative Importance of PaaS Kernel Invocations} 
The purpose of this metric is to find the service call that is contributing mostly towards the variance in the total
response time. We start by selecting a window $W$ in time which includes a sufficient number of application requests,
and ending at the point when the performance anomaly was detected. Note that for each application request
in $W$, we can find the total response time ($T_{total}$), and the time spent on individual PaaS kernel
services ($T_{Xn}$).

Then we take all the $T_{total}$ values
and the corresponding $T_{Xn}$ values in $W$, and fit them to the linear regression model
$T_{total} = T_{X1} + T_{X2} + ... + T_{Xn}$. Here we leave $r$ out deliberately, since it is typically small. To prevent
any bad data from skewing the model, we also filter out requests where the $r$ value is too high. This
is done by computing the mean ($\mu_r$) and standard deviation ($\sigma_r$) of $r$ over the selected window, and removing 
any requests where $r > \mu_r + 1.65\sigma_r$.

Once the regression model has been computed, we run a relative importance algorithm to rank each of the
regressors (i.e. $T_{Xn}$ values) based on their contribution to the variance of $T_{total}$. 
We use the LMG method which is resistant to multicolinearity, and provides a break down of the $R^2$ value of
the regression according to how strongly each regressor is influencing the variance of the dependent variable.
The kernel invocation associated with the highest ranked regressor (i.e. the one that contributes mostly 
towards the variance) is a very strong candidate
for the bottleneck that we are looking for. Statistically, this is the kernel invocation that causes the application
response time to vary the most.

\subsubsection{Historical Trend of Relative Importance}
This is a simple extension of the previous metric. We divide the time window $W$ into equal-sized continuous segments,
and compute the relative importance metrics for regressors within each segment. This way we can
obtain a time series of relative importance values for each regressor. These time series
represent how the relative importance of each variable has changed over time.

We subject each time series to change point analysis to detect if the relative importance of any particular
regressor has increased recently. If such a regressor can be found, then the PaaS kernel invocation
associated with that variable is also a potential candidate for the bottleneck. 

\section{Results}
\label{sec:results}
\begin{table}
\begin{center}
\begin{tabular}{|c|p{1cm}|p{1cm}|p{1cm}|}
\hline
Faulty Service & $L_1$ (30ms) & $L_2$ (35ms) & $L_3$ (45ms) \\ \hline
datastore & 18 & 11 & 10 \\ \hline
user management & 19 & 15 & 10 \\ \hline
\end{tabular}
\end{center}
\caption{Number of anomalies detected in guestbook app under different SLOs 
($L_1$, $L_2$ and $L_3$) when injecting faults into two different PaaS kernel services.
\label{tab:anomaly_counts}
}
\end{table}

We first evaluate the effectiveness of Roots as an anomaly detection mechanism. We experiment with
the SLO-based anomaly detector, using a simple HTML-producing web application called ``guestbook''.
This application allows users to login, and post comments. It uses the datastore service to save
the posted comments, and the user management service to handle authentication. Each request processed
by guestbook results in two PaaS kernel invocations -- one to check if the user is logged in, and 
another to retrieve the existing comments from the datastore. We conduct all
our experiments on a single node AppScale cloud except where specified. The node itself is an Ubuntu
14.04 VM with 4 virtual CPU cores (clocked at 2.4GHz) and 4GB of memory.

We run the SLO-based anomaly detector on guestbook with a sampling rate of 15 seconds, an analysis
rate of 60 seconds, and a window size of 1 hour. We set the minimum sample count to 100, and
run a series of experiments with different SLOs on the guestbook application. Specifically, we fix
the SLO probability at 95\%, and set the response time upper bound to $\mu_g + n\sigma_g$. 
$\mu_g$ and $\sigma_g$ represent mean and standard deviation of the
guestbook's response time. We learn these two parameters apriory by benchmarking
the application. Then we obtain three different upper bound values for the guestbook's
response time by setting 
$n$ to 2, 3 and 5. We denote the resulting three SLOs $L_1$, $L_2$ and $L_3$ respectively.

We also inject performance faults into AppScale by modifying its code. We run a set of experiments
by injecting faults into the datastore service. Our fault injection logic activates once every hour, and
slows down all datastore invocations by 45ms within a period of 3 minutes. 45ms is equal 
to $\mu_g + 5\sigma_g$. Therefore this delay is sufficient to violate all three SLOs used in our experiments. 
We run a similar set of experiments where we inject faults into the user management service of
AppScale. Each experiment is run for a period of 10 hours.

Table~\ref{tab:anomaly_counts} shows how the number of anomalies detected by 
Roots in a 10 hour period varies when the SLO is changed. The number of anomalies
drops noticeably when the response time upper bound is increased. When the $L_3$
SLO (45ms) is used, the only anomalies detected are the ones
caused by our hourly fault injection mechanism. As the SLO is tightened by lowering the upper bound,
Roots detects additional anomalies. These additional anomalies
result from a combination of injected faults, and other naturally occurring faults
in the system. That is, Roots detected some naturally occurring
faults (temporary spikes in application latency), when a number of injected faults
were still in the sliding window of the anomaly detector. Together these two types of
faults caused SLO violations, usually several minutes after the fault injection period
has expired.

Next we analyze how fast and often Roots can detect anomalies in an application. We
first consider the performance of guestbook under the $L_1$ SLO while 
injecting faults into the datastore service. Figure~\ref{fig:time_line_guestbook_2s} shows
anomalies detected by Roots as events on a time line. The horizontal axis represents 
passage of time. The red markers indicate the time windows in which we injected faults into
AppScale. Each window is 3 minutes wide, and therefore appears thin in the full 10 hour scale
of the plot. The tall blue lines indicate the Roots anomaly detection events.
Note that every fault injection window is immediately followed by an anomaly
detection event, implying near real time reaction from Roots. The only exception is the fault
injection window at 20:00 hours which is not immediately followed by an anomaly 
detection event. Roots detected another naturally occurring anomaly at 19:52 hours
which caused the anomaly detector to go into the warm up mode. Therefore Roots
did not immediately react to the faults injected at 20:00 hours. But as soon as the detector became
active again at 20:17, it detected the anomaly.

\begin{figure}
\centering
\includegraphics[scale=0.55]{time_line_guestbook_2s}
\caption{Anomaly detection in guestbook application (timeline).}
\label{fig:time_line_guestbook_2s}
\end{figure}

\begin{figure}
\centering
\includegraphics[scale=0.55]{time_line_guestbook_2s_user}
\caption{Anomaly detection in guestbook application (timeline).}
\label{fig:time_line_guestbook_2s_user}
\end{figure}

Figure~\ref{fig:time_line_guestbook_2s_user} shows the anomaly detection time line for the 
same application and SLO, while faults are being injected into the user management service.
Here too we see that Roots detects anomalies immediately following each fault injection window.

\begin{figure}
\centering
\includegraphics[scale=0.55]{time_line_crud}
\caption{Anomaly detection in key-value store application (timeline).}
\label{fig:time_line_crud}
\end{figure}

\begin{figure}
\centering
\includegraphics[scale=0.55]{time_line_caching}
\caption{Anomaly detection in caching application (timeline).}
\label{fig:time_line_caching}
\end{figure}

Next we evaluate the effectiveness and accuracy of the path distribution analyzer. For this we 
employ two applications.
\begin{description}
\item[key-value store] This application provides the functionality of an online key-value store.  It allows 
users to store data objects in the cloud where each object is given a unique key. The objects can then be 
retrieved, updated or deleted using their keys. Different operations
(create, retrieve, update and delete) of the application are implemented as separate paths of
execution in the application source code.
\item[cached key-value store] This is a simple extension of the regular key-value store, which adds
caching to the read operation using the AppScale's memcache service. The application contains
separate paths of execution for cache hits and cache misses.
\end{description}

We first deploy the key-value store on AppScale, and populate it with a number of data objects. Then we
run a test client against it which generates a read-heavy workload. On average this workload
consists of 90\% read requests and 10\% write requests (create and update). The test client
is also programmed to randomly send bursts of write-heavy workloads. These bursts consist
of 90\% write requests on average, and each burst lasts up to 2 minutes. Figure~\ref{fig:time_line_crud}
shows the write-heavy bursts as events on a time line (indicated by red arrows). Note that almost every burst is
immediately followed by an anomaly detection event in Roots (indicated by blue arrows). The path
distribution analyzer detects every time the distribution of requests among different paths of execution
changes significantly. The only time we do not see an anomaly detection event is when multiple
bursts are clustered together in time (e.g. three bursts between 17:04 and 17:24 hours). In this
case Roots detects the very first burst, and then goes into the warm up mode to collect more data. Therefore
the bursts that immediately follow do not raise an alarm. Between 20:30 and 21:00 hours we also
had two instances where the read request proportion dropped from 90\% to 80\% due to random
chance. This is because our test client randomizes the read request proportion around the 90\% mark. 
In these two instances the read proportion was deemed too far off from 90\%, and Roots correctly notified us by 
detecting them as anomalies.

We conduct a similar experiment using the cached key-value store. Here, we run a test client that generates a workload
that is mostly served from the cache. This is done by repeatedly executing read requests on a small
selected set of object keys. However, the client randomly sends bursts of traffic requesting keys that
are not likely to be in the application cache, thus resulting in many cache misses. Each burst
lasts up to 2 minutes. As shown in 
figure~\ref{fig:time_line_caching}, Roots path distribution analyzer correctly detects the change 
in the workload (from many cache hits to many cache misses), nearly every time the test client injects a 
burst of traffic that triggers the cache miss path of the application. The only exception is when
multiple bursts are clumped together, in which case only the first first raises an alarm in Roots.

\begin{figure}
\centering
\includegraphics[scale=0.5]{workload_change_trace}
\caption{Workload size over time for the key-value store application. The test client randomly sends
large bursts of traffic causing the spikes in the plot. Roots anomaly detection events are shown
in red dashed lines.}
\label{fig:workload_change}
\end{figure}

Next we evaluate the Roots workload change analyzer. In this experiment we run a varying workload
against the key-value store application for 10 hours. The load generating client is programmed
to maintain a mean workload level of 500 requests per minute. However, the client
is also programmed to randomly send large bursts of traffic at times of its choosing. During these bursts 
the client may send more than 1000 requests a minute, thus impacting the performance of
the application server that hosts the key-value store. Figure~\ref{fig:workload_change} shows how
the application workload has changed over time. The workload generator has produced 6 large bursts of traffic during the 
period of the experiment, which appear as tall spikes in the plot.
Note that each burst is immediately followed by a Roots anomaly detection event (shown by red dashed lines). 
In each of these 6 cases, the increase in workload caused a violation of the application performance SLO.
Roots detected the corresponding anomalies, and determined them to be caused by changes in the workload size.
Even though the bursts of traffic appear to be momentary
spikes, each burst lasts for 4 to 5 minutes thereby causing a lasting impact on the application performance.
The PELT change point detection method used in this experimental set up is ideally suited for detecting
such lasting changes in the workload level.

Now we evaluate the bottleneck identification capability of Roots. We first discuss the results obtained using
the guestbook application. Later we will show some results obtained using a more complex application.
In the experimental run illustrated in 
figure~\ref{fig:time_line_guestbook_2s}, Roots determined that all the detected anomalies except for one were 
caused by the AppScale datastore service. This is consistent with our expectations since in this experiment we 
artificially inject faults into the datastore.
The only anomaly that is not traced back to the datastore service is the one that was detected at 14:32 hours.
This is indicated by the blue arrow with a small square marker at the top. For this anomaly, Roots concluded that
the bottleneck is the local execution at the application server ($r$). We have verified
this result by manually inspecting the AppScale logs and traces of data collected by Roots. As it turns out,
between 14:19 and
14:22 the application server hosting the guestbook application experienced some problems, which caused
request latency to increase significantly. Therefore we can conclude that Roots has correctly identified 
the root causes of all 18 anomalies in this experimental run.

Similarly, in the experiment shown in figure~\ref{fig:time_line_guestbook_2s_user}, Roots determined
that all the anomalies are caused by the user management service, except in one instance. This is again
inline with our expectations since in this experiment we inject faults into the user management service. For the
anomaly detected at 04:30 hours, Roots determined that local execution time is the primary bottleneck.
In this case too the server hosting the guestbook application became slow
during the 04:23 - 04:25 time window, and Roots correctly identified the bottleneck as the local
application server.

\begin{figure}
\centering
\includegraphics[scale=0.55]{time_line_stocks_1}
\caption{Anomaly detection in stock-trader application (timeline).}
\label{fig:time_line_stocks_1}
\end{figure}

\begin{figure}
\centering
\includegraphics[scale=0.55]{time_line_stocks_2}
\caption{Anomaly detection in stock-trader application (timeline).}
\label{fig:time_line_stocks_2}
\end{figure}

In order to evaluate how the bottleneck identification performs when an application makes more than 2
PaaS kernel invocations, we conduct another experiment using an application called ``stock-trader''.
This application allows setting up organizations, and simulating trading of stocks between the
organizations. The two main operations in this application are \textit{buy} and \textit{sell}. Each of
these operations makes 8 calls to the AppScale datastore. 
According to our previous work~\cite{Jayathilaka:2015:RTS:2806777.2806842}, 8 kernel invocations in the
same path of execution is very rare in web applications developed for a PaaS cloud. The probability
of finding an execution path with more than 5 kernel invocations in a sample of PaaS-hosted
applications is less than 1\%. Therefore the stock-trader application is a good extreme case
example to test the Roots bottleneck identification support.
We execute a number of experimental runs using this application,
and here we present the results from two of them. In all experiments we configure the anomaly
detector to check for the response time SLO of 177ms with 95\% success probability.

In one of our experimental runs we inject faults into the first datastore query executed by the buy operation
of stock-trader. The fault injection logic runs every two hours, and lasts for 3 minutes. The duration of
the full experiment is 10 hours. 
Figure~\ref{fig:time_line_stocks_1} shows the resulting event sequence. Note that every fault injection
event is immediately followed by a Roots anomaly detection event. There are also four additional
anomalies in the time line which were SLO violations caused by a combination of injected faults, and
naturally occurring faults in the system. For all the anomalies detected
in this test, Roots correctly selected the first datastore call in the application code as the bottleneck. 
The additional four anomalies occurred because a large number of injected faults were in the sliding window
of the detector. Therefore, it is accurate to attribute those anomalies also to the first datastore query
of the application.

Figure~\ref{fig:time_line_stocks_2} shows the results from a similar experiment where we inject
faults into the second datastore query executed by the operation. Here also Roots detects all the
artificially induced anomalies along with a few extras. All the anomalies, except for one, 
are determined to be caused by the second
datastore query of the buy operation. The anomaly detected at 08:56 (marked with a square on top of the blue arrow) 
is attributed to the fourth datastore query executed by the application. We have manually verified this
conclusion to be accurate. Since 08:27 (when the previous anomaly was detected), the fourth datastore
query has frequently taken a long time to execute, which resulted in an SLO violation at 08:56 hours.

\begin{figure}
\centering
\includegraphics[scale=0.5]{bottleneck_scores}
\caption{Frequency of different bottleneck scores.}
\label{fig:bottleneck_scores}
\end{figure}

Recall that the bottleneck identification algorithm in Roots
selects up to four candidate components for each performance anomaly detected, and then ranks them
by assigning scores to identify the most likely bottleneck. Figure~\ref{fig:bottleneck_scores} shows the breakdown of 407 anomalies
detected over a period of 3 weeks. X-axis represents the different scores given to candidate components
by our algorithm. Y-axis shows the number of times a particular score was the highest. 
According to this result, on 13 occasions Roots determined the bottleneck based on the highest score
of 4 (score given to the component identified by the relative importance metric). 
This happens when the algorithm chooses four different candidates
for the bottleneck. However, this constitutes only 3.2\% of all the anomalies. In 96.8\% of the time Roots saw
at least two of the four candidates to be the same (score values 6 or higher). This implies that most of the time Roots is able to
identify bottlenecks with a high level of confidence since two or more candidate detection methods
agree on their results.

Next we evaluate the performance overhead incurred by Roots on the applications deployed in the 
cloud platform. We are particularly interested in understanding the overhead of recording the PaaS kernel
invocations made by each application, since this feature requires some changes to the PaaS SDK. 
We deploy a number of applications on a vanilla
AppScale cloud (with no Roots), and measure their request latencies. We use
the popular Apache Bench tool to measure the request latency under a
varying number of concurrent clients. We then take the same measurements
on an AppScale cloud with Roots, and compare the results against the ones obtained
from the vanilla AppScale cloud. In both environments we disable the auto-scaling
support of AppScale, so that all client requests are served from a single application
server instance. The PaaS SDK libraries we have instrumented for Roots reside within
the application server. Therefore, the kernel invocation events get buffered in
the application server before they are sent to the Roots data storage. We wish to
explore how this feature performs when the application server is under heavy load.

\begin{table}
\begin{center}
\begin{tabular}{|c|p{0.8cm}|p{0.8cm}|p{0.8cm}|p{0.8cm}|}
\hline &
      \multicolumn{2}{c|}{Without Roots} &
      \multicolumn{2}{c|}{With Roots} \\ \hline
    App./Concurrency & Mean (ms) & SD & Mean (ms) & SD\\

\hline
guestbook/1 & 12 & 3.9 & 12 & 3.7 \\ \hline
guestbook/50 & 375 & 51.4 & 374 & 53 \\ \hline
stock-trader/1 & 151 & 13 & 145 & 13.7 \\ \hline
stock-trader/50 & 3631 & 690.8 & 3552 & 667.7 \\ \hline
kv store/1 & 7 & 1.5 & 8 & 2.2 \\ \hline
kv store/50 & 169 & 26.7  & 150 & 25.4  \\ \hline
cached kv store/1 & 3 & 2.8 & 2 & 3.3 \\ \hline
cached kv store/50 & 101 & 24.8 & 97 & 35.1  \\ \hline
\end{tabular}
\end{center}
\caption{Latency comparison of applications when running on
a vanilla AppScale cloud vs when running on a Roots-enabled
AppScale cloud.
\label{tab:perf_overhead}
}
\end{table}

Table~\ref{tab:perf_overhead} shows the comparison of request 
latencies. We discover that Roots does not add a significant overhead
to the request latency in any of the scenarios considered. In all the cases,
the mean request latency when Roots is in use, is within one standard deviation
from the mean request latency when Roots is not in use. Also in a number
of cases the mean request latency is slightly lower when Roots is in use.
The request latency increases when the number of concurrent clients is
increased from 1 to 50 (since all requests are handled by a single
application server), but still there is no sign of any detrimental overhead
from Roots even under load. Since we buffer PaaS kernel invocation
events in memory, and report them to ElasticSearch asynchronously, and
out of the request processing flow, there is no measurable impact on
the request latency from Roots.

\begin{figure}
\centering
\includegraphics[scale=0.45]{pod_performance}
\caption{Resource utilization of a Roots pod.}
\label{fig:pod_performance}
\end{figure}

Finally, to demonstrate how lightweight and scalable Roots is, we deploy
a Roots pod on a virtual machine with 4 CPU cores and 4GB memory.
To simulate monitoring multiple applications, we run multiple concurrent anomaly
detectors in the pod. Each detector is configured with a 1 hour sliding window.
We also enable workload change analysis, and bottleneck identification 
support in the pod. We vary the number of concurrent anomaly
detectors between 100 and 10000, and run each configuration for
2 hours. We track the memory usage and the percentage CPU usage of the
pod during each of these runs. 

Figure~\ref{fig:pod_performance}
illustrates the resource utilization of the Roots pod for different counts of
concurrent anomaly detectors. We see that even with 10000 concurrent
detectors, the average CPU usage is at 40\%, and the average memory usage is
at 429 MB. The highest memory usage observed in these experiments is
785 MB. Since each detector operates with a fixed-sized window, and they
bring additional data into memory only when required, the overall memory
usage of the Roots pod stays low. This implies that we can monitor thousands 
of applications using a single pod, thereby scaling up to a very large number
of applications using only a handful of pods.

\section{Related Work}
SLA management on service-oriented systems and cloud systems has been 
studied in some depth previously.
Much of this existing work has focused on issues 
such as SLA monitoring~\cite{Michlmayr:2009:CQM:1657755.1657756,Tripathy:2011:MMS:1980822.1980832,Raimondi:2008:EOM:1453101.1453125,Bertolino:2007:SUS:1294904.1294914} and SLA modeling~\cite{Chau:2008:ASM:1463788.1463802,Stamou:2013:SGM:2516588.2516592,Skene:2004:PSL:998675.999422}. 
In our work, we automatically identify the SLAs that can be defined and
maintained 
for a given web API by using a combination of
static analysis, cloud platform monitoring and time series analysis.  
%In~\cite{cerebro-soccsub15}, we introduce Cerebro and its prediction methodology, and evaluate the
%accuracy and tightness of its response time predictions.

In PROSDIN~\cite{Mahbub:2011:PSN:2061042.2062022}, a proactive service discovery and negotiation
framework, the SLA negotiation occurs during the service discovery phase. This is similar to how
Cerebro establishes an initial SLA with an API consumer, when the consumer subscribes to an API. PROSDIN also
establishes a fixed SLA validity period upon negotiation, and triggers an SLA renegotiation when this time period has 
elapsed. Cerebro on the other hand continuously monitors the cloud platform,
and periodically re-evaluates the response time SLAs of web APIs 
to determine when a re-negotiation is needed.
%-- an idea that draws inspiration from existing SLA
%monitoring techniques. 
%Upon detecting an SLA violation, we prompt the API consumer, and renegotiate the SLA.
Similarly, researchers have investigated the notions of SLA brokering~\cite{6546098}, and the automatic SLA negotiation
between intelligent agents~\cite{Yaqub:2014:ONS:2680847.2681496}, ideas that can complement the
simple SLA negotiation model of Cerebro to make it more powerful and flexible.

Meryn~\cite{Dib:2013:MOS:2465823.2465825} is an SLA-driven PaaS system that attempts to maximize cloud
provider profit, while providing the best possible quality of service to the cloud users. It supports
SLA negotiation at application deployment, and SLA monitoring to detect
violations. However, it does not automatically determine what SLAs are
feasible or address SLA renegotiation, 
and employs a policy-based mechanism coupled
with a penalty cost charged against the cloud provider to
handle SLA violations. Also, Meryn formulates SLAs in terms of the computing resources (CPU, memory,
storage etc.) allocated to applications. It assumes a batch processing environment where the
execution time of an application is approximated based on a detailed description of the application provided
by the developer. In contrast, Cerebro handles SLAs for interactive web applications. It predicts
the response time of applications using static analysis, without any input from the application developer. 
Cerebro also supports automatic SLA renegotiation, with possible room for economic incentives.

There has also been prior work in the area of predicting 
SLA violations~\cite{Leitner10,6976585,Duan:2006:PIP:1142473.1142582}. 
These systems take an existing SLA and historical performance data of a service, and predict when the 
service might violate the given SLA in the future. 
Cerebro's notion of prediction validity period has some commonalities with this concept. In fact, Cerebro
can make use of such a method to determine the frequency at which it should re-evaluate the predicted
SLAs.


\section{Conclusions and Future Work}
\begin{itemize}
\item We tested Cerebro for prediction accuracy using a wide range of web applications. In these tests Cerebro was used to
predict the 95th percentile of the API execution time. In all cases, Cerebro succeeded in achieving a percentage
accuracy level close to or higher than 95\%. That is, the actual API execution times measured from the 
applications were below the predicted 95th percentile values 95\% or more of the time.
\item For most of the test APIs, Cerebro managed to make tight predictions (i.e the predictions are not too far
from the actual values). In situations where the web APIs have highly variant performance characteristics with many high outliers,
we saw that Cerebro trades off tightness for accuracy, by making conservative predictions.
\item We observed that Cerebro takes some time to learn from the gathered time series data before it can make highly reliable
predictions. In our tests this learning period took somewhere from 150 to 200 minutes. The percentage accuracy undergoes
many large fluctuations during this period, but once Cerebro's prediction algorithm has converged, it produces highly
accurate and reliable results consistently. Only very minor fluctuations in percentage accuracy can be observed after
convergence.
\item We introduced a theoretical model for determining when to treat a predicted SLA as invalid, and used that model
to compute the prediction validity periods for Cerebro. We noticed that on average the prediction validity period
can fall somewhere between 24 and 72 hours. Further, we observed that 95\% of the time the validity period is at least 1.41 hours
on App Engine, and 1.95 hours on AppScale.
\item When comparing the App Engine results to AppScale results we saw that Cerebro produces tighter and
more long lasting SLA predictions on AppScale. We attributed this behavior to the fact that AppScale is a small and controlled
environment that is much more stable over time. %In contrast Google App Engine is much larger, shared among many users
%and provides no control over the underlying hardware resources.
\end{itemize}

%
% The following two commands are all you need in the
% initial runs of your .tex file to
% produce the bibliography for the citations in your paper.
\bibliographystyle{IEEEtran} 
\bibliography{references}  % sigproc.bib is the name of the Bibliography in this case
% You must have a proper ".bib" file
%  and remember to run:
% latex bibtex latex latex
% to resolve all references

\begin{IEEEbiography}[{\includegraphics[width=1in,height=1.25in,clip,keepaspectratio]{hiranya2}}]{Hiranya Jayathilaka}
Hiranya Jayathilaka is a fifth year PhD candidate at the computer science department of UC Santa Barbara. His research focuses on automated governance
for cloud platforms, which addresses a wide range of issues including policy enforcement, performance guarantees, monitoring and troubleshooting. He received a
B.Sc. in engineering from University of Moratuwa, Sri Lanka.
\end{IEEEbiography}

\begin{IEEEbiography}[{\includegraphics[width=1in,height=1.25in,clip,keepaspectratio]{ckrintz}}]{Chandra Krintz}
Chandra Krintz is a Professor of Computer Science (CS) at UC Santa Barbara and Chief Scientist at AppScale Systems Inc. Chandra holds M.S./Ph.D. degrees in CS from UC San Diego.  Chandra's research interests include programming systems, cloud and big data computing, and the Internet of Things (IoT).  Chandra has supervised and mentored over 60 students and has led several educational and outreach programs that introduce young people to computer science.
\end{IEEEbiography}

\begin{IEEEbiography}[{\includegraphics[width=1in,height=1.25in,clip,keepaspectratio]{rich}}]{Rich Wolski}
Dr. Rich Wolski is a Professor of Computer Science at the UC Santa Barbara, and co-founder of Eucalyptus Systems Inc.  Having received his M.S. and Ph.D.  degrees from UC Davis (while a research scientist at Lawrence Livermore National Laboratory) he has also held positions at the UC San Diego, and the University of Tennessee, the San Diego Supercomputer Center and Lawrence Berkeley National Laboratory.  Rich has led several national scale research efforts in the area of distributed systems, and is the progenitor of the Eucalyptus open source cloud project.
\end{IEEEbiography}

\end{document}
