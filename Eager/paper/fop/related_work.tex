Our research builds upon advances in the areas of SOA governance and
service management. 
Guan et al introduced FASWSM~\cite{1607141} a web service management
framework for application servers. Wu et al introduced DART-Man~\cite{1504267} a web
service management system based on semantic web concepts. 
Our work is different 
from these past approaches in that EAGER targets policy \textit{enforcement} 
and we focus on doing so by extending extant
cloud platforms (e.g. PaaS) to provide an integrated and scalable governance
solution.

Lin et al proposed a service management system for clouds that monitors all
service interactions via special ``hooks'' that are connected to the
cloud-hosted services~\cite{5616981}. However, this system only supports run-time
service management and provides no support for deployment-time governance. 
Kikuchi and Aoki~\cite{6525502} proposed a technique
based on model checking to evaluate the operational vulnerabilities and fault
propagation patterns in cloud services. But this system provides no
active monitoring or enforcement functionality.

Other researchers have shown that policies can be
used to perform a wide range of governance tasks for SOA such as access
control~\cite{4279630}, fault diagnosis~\cite{6154236},
and management~\cite{Suleiman:2009:IUM:1564601.1564730}. We build
upon the foundation of these past efforts and use policies to govern
RESTful web APIs deployed in cloud settings. 

Peng, Lui and Chen showed that
the major concerns associated with SOA governance 
involve retaining the high reliability of services, recording how many services
are available on the platform to serve, and making sure all the available 
services are operating within an acceptable service
level~\cite{4730489}. EAGER attempts to satisfy similar requirements for 
modern RESTful web APIs deployed in cloud environments. 
However, EAGER's Metadata Manager and ADP record and keep track of all deployed APIs 
in a comprehensive manner.  Moreover, EAGER's governance features 
``fail fast'' to detect violations immediately.

API management has been a popular topic in the industry over the last few years, resulting
in many API management solutions~\cite{wso2am,apigee}. These products facilitate
API lifecycle management, traffic shaping, access control, monitoring and a variety of other
important API-related functionality. However, these tools do not support deep integration with
cloud environments in which many web applications and APIs are deployed today. 

%
%These API
%management products either run as stand-alone entities or are layered on top of existing clouds
%thus leading to many maintenance, reliability and integration issues. Also, the support provided by these
%tools to specify and enforce policies in a flexible manner is very limited. EAGER is complementary
%to such tools, providing very deep integration with PaaS clouds thereby facilitating API governance
%as a core cloud-native feature. Indeed, EAGER can embed and enhance some of these tools to
%obtain the required API governance support in the cloud. In fact, the EAGER prototype we have
%implemented makes extensive use of an existing open source API management product~\cite{wso2am}.
