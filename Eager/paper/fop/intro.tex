The growth of the World Wide Web (WWW), web services, and cloud computing have
significantly influenced the way developers implement software applications.
Instead of implementing all the functionality from the scratch, developers
increasingly offload as much application functionality as possible to remote,
web-accessible application programming interfaces (web APIs) hosted ``in the
cloud''. As a result, web APIs are rapidly proliferating.
At the time of this writing, 
ProgrammableWeb~\cite{pweb}, a popular web API index, lists over $11,000$
web APIs and a nearly 100\% annual growth rate. 

This proliferation of web APIs demands new techniques that
control and govern the evolution of APIs as a first-class software
resource. A lack of API governance can lead to 
security breaches, denial of service (DoS)
attacks, poor code reuse and violation of service-level agreements (SLAs). 
Unfortunately, most existing cloud platforms
within which web APIs are hosted provide only minimal governance support.
Therefore, organizations that use cloud platforms to deploy their web APIs, have to
do it without proper governance support thus leading to many software maintenance
and evolution issues.

API governance for cloud platforms can be broken down into two varieties:
\begin{itemize}
\item deployment-time governance in which governance checks are performed 
when the APIs are being deployed into the cloud, and
\item run-time governance in which governance checks are carried out when the APIs
are being invoked by the clients.
\end{itemize}
Of these two, 
deployment-time enforcement (heretofore unexplored) is attractive for several
reasons.  First, if run-time only API governance is implemented, 
policy violations will go undetected until the offending APIs are used (at which
point it is too late and expensive to take corrective measures).  
By enforcing governance at deployment-time,
cloud platforms can support ``fail fast'' in which violations are detected
immediately, before they even become available for use. Further, APIs served from a cloud
platform are invoked way more times than they are deployed and redeployed. 
Therefore, deployment-time API governance helps reduce the overall API governance
overhead by eliminating many checks that would otherwise be repeated unnecessarily
if executed as run-time checks.

In order to explore the feasibility and the performance traits of deployment-time
API governance in cloud platforms,
we propose EAGER ({\bf E}nforced {\bf A}PI {\bf G}overnance
{\bf E}ngine for {\bf R}EST), a model and an architecture that augments existing
Platform as a Service (PaaS) clouds in order to facilitate API governance as a 
cloud-native feature. EAGER enforces proper versioning of APIs and supports dependency 
management and comprehensive policy enforcement, all at API deployment-time.

EAGER further enhances software maintainability by guaranteeing that 
developers reuse existing APIs when possible to create new software artifacts. 
Concurrently, it
tracks changes made by developers to deployed web APIs to prevent
any backwards-incompatible API changes from being put into production.
EAGER also includes a language for specifying 
API governance policies.  It incorporates a developer-friendly 
Python programming language syntax for 
specifying complex policy statements in a simple and 
intuitive manner. Moreover, we ensure that specifying the required policies 
is the only additional activity that API providers should perform in
order to benefit from EAGER. All other API governance related verification and 
enforcement work is carried out by the cloud platform automatically.

To evaluate the usefulness and the performance of deployment-time API governance, 
we implement EAGER as an extension to AppScale~\cite{appscale13}, 
an open source cloud platform that emulates Google App Engine. Through this
prototype we show that the EAGER 
architecture and hence deployment-time API governance can be easily implemented 
in extant clouds with minimal changes to the underlying platform technology. Further,
our performance test results show that EAGER scales well to handle thousands of APIs,
policies and API inter-dependencies, thus bringing the cloud computing community several
steps closer to supporting comprehensive low-overhead deployment-time API governance
as a cloud-native feature.

In the sections that follow, we present the design and implementation of
EAGER. We then empirically evaluate EAGER using a wide range of
experiments, and conclude.
