Cloud computing has revolutionized the way programmers develop and deploy applications.
By running computations and storing data on large, managed infrastructures and 
programming platforms, cloud computing ensures scalability and high-availability of 
user applications. Moreover, by offering such compute infrastructures and programming 
platforms as services, cloud computing is able to facilitate a cost-effective, on-demand
resource provisioning model that greatly enhances user productivity.

Over the last decade cloud computing technologies have enjoyed explosive growth 
and near universal adoption due to their many benefits and promises. Core cloud 
functionalities are available from a large and growing number of service providers. 
Some of these service providers offer hosted cloud solutions that can be used
via the web to deploy applications without installing any physical hardware 
(e.g. Amazon AWS, Google App Engine, Microsoft Azure). Others
provide cloud technologies as downloadable software, which users can install
on their computers or data centers to set up their own private clouds 
(e.g. Eucalyptus, AppScale, OpenShift). 

Both the IT industry and the academia have responded very positively to this new 
model of application 
development. Many business organizations are currently in the process of migrating
their IT infrastructure to the cloud. A large number of organizations
run their entire business as a cloud-based operation (e.g. Netflix, Snapchat). For startups
and academic researchers who don't have a large IT budget or a staff, the cost-effective 
on-demand resource provisioning model of the cloud has proved to be indispensable.
The growing number of academic conferences and journals dedicated to discussing
cloud computing is further evidence that cloud is an essential branch in the field
of computer science.

Despite its many benefits, cloud computing has also given rise to a number of application
development and maintenance issues that have gone unaddressed for many years. 
First and foremost, cloud platforms do not enforce any of the developer best practices
the software engineering community has established over the last half-century. This
includes code reuse, proper versioning of software artifacts, dependency management
between application components and backward compatible code updates. In
some cases organizations require certain conventions/standards to be enforced on
all production software. Cloud platforms do not provide any facilities for reaching
such administrative conformance on cloud-hosted applications. Instead, cloud platforms
are making it extremely simple and quick to deploy new applications or update existing
applications. The resulting speed-up of the development cycles combined with the lack of 
oversight and enforcement, makes it extremely difficult for 
IT personnel to manage large volumes of cloud-hosted applications.

Secondly, today's cloud platforms do not provide any support for reasoning about the 
performance of the deployed applications. When an application is implemented for
a given cloud platform, one must subject it to extensive performance testing in order
to realize its performance limits; a process that is both 
tedious and time consuming. The difficulty in understanding the performance 
traits of cloud-hosted applications is primarily due to the very high level of 
abstraction provided by the cloud platforms. These abstractions shield many details 
concerning the application runtime, and without visibility into such low level application 
execution details it is impossible
to build a robust performance model for a cloud-hosted application. Because of the same
limitation, it is also not possible to formulate performance service-level agreements (SLAs)
for cloud-hosted applications. Consequently, existing cloud platforms only offer
availability SLAs.  However, performance SLAs are an absolute necessity for
mission critical applications, interactive applications and most web and mobile applications.

Thirdly, cloud platforms provide very poor support for performance debugging
user applications. Most cloud platforms only provide the simplest monitoring and logging features,
and do not provide any mechanisms for detecting performance anomalies or identifying
bottlenecks in the application code or the underlying cloud platform. This limitation has given rise
to a new class of third party service providers that specialize in monitoring cloud applications
(e.g. New Relic, Datadog). But these third party solutions require additional configuration 
and are expensive.
Furthermore, they only have a restricted view of the user application and the underlying
cloud platform, which limits their capabilities in terms of the type of analyses they can carry out.
Today's cloud platforms are also very large and complex with many interacting components.
An external monitoring service cannot see all this complexity, and hence cannot pinpoint
the component that might be responsible for an observed application performance anomaly.

In order to make cloud computing more reliable and dependable for the users as well
as the cloud service providers, the above limitations need to be addressed satisfactorily.
Doing so will also greatly simplify the tasks of developing cloud applications, and maintaining 
them in the long run. As a means of reaching this goal, I would like to propose and explore the
following thesis question.

{\bf Can we efficiently enforce governance on cloud-based web applications to achieve 
administrative conformance, developer best practices, and performance objectives through 
automated analysis and diagnostics?}  

Web APIs also play a key role in managing and enforcing service level agreements (SLAs).
Ideally, any public API should be exposed with a well-defined
SLA detailing its performance and quality of service (QoS) characteristics. Without proper means
of API governance, developers that host services in cloud settings have no way of enforcing
competitive SLAs on their APIs, or even getting a thorough idea of what kind of SLAs
their services can uphold. Today, developers have to perform a lot of offline
and online testing to learn the performance characteristics of their APIs.
Furthermore, developers often have to be content with basic SLA monitoring 
as opposed to comprehensive SLA enforcement. 

{\bf My research focuses on designing and implementing systems for defining and
implementing 
API governance as a cloud-native feature.}  API management is often a
documentary function (e.g. ~\cite{apigee,layer7,wso2am} that does not include the ability to
enforce policies, particularly at scale in a cloud setting. 
My research agenda focuses on making API governance a ``first-class'' cloud
service that is fully integrated
with the core of the cloud and the cloud management tools. 
In a cloud-based IT context, APIs must be governed by
policies that are defined by the organization to ensure predictable
operation of the web services hosted by the cloud. Additionally,
there should be automated mechanisms in place for easily developing new
APIs, discovering existing APIs, analyzing the syntactic and semantic features
of APIs, and porting applications across APIs. My research will make 
it trivial for developers to consume, combine and leverage APIs to create powerful
applications, while adhering to tried and tested software engineering 
best practices. They will be able to quickly learn
the QoS properties of APIs and enforce competitive SLAs through the cloud
with clear and accurate levels of certainty. My plan is to
\begin{itemize}
%\item develop new {\bf policy specification languages for web
%services} that
%enables governance enforcement during service development, 
%and at runtime,
\item develop a distributed {\bf cloud-integrated service for enforcing IT 
management policies} governing
web APIs when operating at scale in cloud settings,
\item develop automated mechanisms for discovering, analyzing, comparing and reasoning
about web APIs, in order to {\bf simplify the API-based development model}, and,
\item develop methods for {\bf automated QoS and performance analysis} of web services
to formulate and enforce competitive SLAs for APIs hosted in clouds.
\end{itemize}

My research spans a number of prominent research areas in computer
science including programming languages, 
distributed systems and software engineering. The proposed policy enforcement
platform motivates new research on innovative new ways of using existing
distributed consensus, scalability and high availability techniques. It also calls for
more research in novel policy specification languages that
are developer-friendly, and can be verified and executed efficiently in cloud settings.
Automated performance analysis requires more research in using
static analysis methods (e.g. abstract interpretation, WCET analysis, etc.), system modeling,
and simulations for analyzing a wide range of services deployed in modern cloud platforms.
Above all, since the primary target of this
work are cloud platforms, there needs to be a strong focus towards making all these 
proposed mechanisms scale to handle thousands of APIs, policies, client applications and users.

\subsection{Outcomes and Assessment}

The outcome of this research plan,
I believe, will be new advances in services computing, IT management, cloud
computing, API policy enforcement, and SLA management.
If successful, the research will be transformative since APIs are rapidly
becoming the most valuable digital asset hosted in cloud settings
and my work enables scalable management of these assets.

In addition,
as a cloud systems researcher I intend to use open source, on-premise clouds (AppScale~\cite{krintzappscale13}
and Eucalyptus~\cite{eucalyptus09}) to produce research and educational artifacts (data sets, demos and test applications, 
code repositories, etc.) to my colleagues, 
and the wider research community.  
I also plan to make my systems, tools and mechanisms available as persistent software artifacts to
the wider research community and to monitor its uptake.  Because of their design
and their use of open source cloud technologies, results of my research will be readily available
to researchers and educators who wish to develop new artifacts and curricula
for cloud computing. 
