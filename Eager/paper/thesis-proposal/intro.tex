Cloud computing has revolutionized the way programmers develop and deploy applications.
Cloud computing turns compute infrastructures, programming platforms and software systems
into online utility services that can be easily shared with many users.
By running computations and storing data on large, managed infrastructures and 
programming platforms, cloud computing ensures scalability and high-availability of 
user applications. Moreover, by offering such resources
as utility services, cloud computing is able to facilitate a cost-effective, on-demand
resource provisioning model that greatly enhances user productivity.

Over the last decade cloud computing technologies have enjoyed explosive growth 
and near universal adoption due to their many benefits and promises. Core cloud 
functionalities are available from a large and growing number of service providers. 
Some of these service providers offer hosted cloud solutions that can be used
via the web to deploy applications without installing any physical hardware 
(e.g. Amazon AWS, Google App Engine, Microsoft Azure). Others
provide cloud technologies as downloadable software, which users can install
on their computers or data centers to set up their own private clouds 
(e.g. Eucalyptus, AppScale, OpenShift). 

Both the IT industry and the academia have responded very positively to this new 
model of application 
development. Many business organizations are currently in the process of migrating
their IT infrastructure to the cloud. A large number of organizations
run their entire business as a cloud-based operation (e.g. Netflix, Snapchat). For startups
and academic researchers who don't have a large IT budget or a staff, the cost-effective 
on-demand resource provisioning model of the cloud has proved to be indispensable.
The growing number of academic conferences and journals dedicated to discussing
cloud computing is further evidence that cloud is an essential branch in the field
of computer science.

Despite its many benefits, cloud computing has also given rise to a number of application
development and maintenance issues that have gone unaddressed for many years. 
First and foremost, cloud platforms do not enforce any of the developer best practices
the software engineering community has established over the last half-century. This
includes code reuse, proper versioning of software artifacts, dependency management
between application components and backward compatible code updates. In
some cases organizations require certain conventions/standards to be enforced on
all production software. Cloud platforms do not provide any facilities for reaching
such administrative conformance on cloud-hosted applications. Instead, cloud platforms
are making it extremely simple and quick to deploy new applications or update existing
applications. The resulting speed-up of the development cycles combined with the lack of 
oversight and enforcement, makes it extremely difficult for 
IT personnel to manage large volumes of cloud-hosted applications.

Secondly, today's cloud platforms do not provide any support for reasoning about the 
performance of the deployed applications. When an application is implemented for
a given cloud platform, one must subject it to extensive performance testing in order
to realize its performance limits; a process that is both 
tedious and time consuming. The difficulty in understanding the performance 
traits of cloud-hosted applications is primarily due to the very high level of 
abstraction provided by the cloud platforms. These abstractions shield many details 
concerning the application runtime, and without visibility into such low level application 
execution details it is impossible
to build a robust performance model for a cloud-hosted application. Because of the same
limitation, it is also not possible to formulate performance service-level agreements (SLAs)
for cloud-hosted applications. 

An SLA is a contract between a service provider and a consumer, that dictates the minimum
service levels that the service provider is obligated to maintain, along with the 
penalties for failing to maintain those service levels. The minimum service levels are sometimes
referred to as service-level objectives (SLOs), and they may be related to various quality of service
parameters such as availability, response time (latency), and throughput. A performance SLA for 
a cloud-hosted application
would specify upper bounds on the application response time, and the likelihood those limits
are valid. Such contracts are an absolute necessity for
mission critical applications, interactive applications and most web and mobile applications.
However, without a systematic mechanism to determine the performance limits of the cloud-hosted
applications, it is impossible to formulate performance SLOs or SLAs.
Consequently, existing cloud platforms only offer SLAs regarding service availability (uptime). 

Thirdly, cloud platforms provide very poor support for performance debugging
user applications. Most cloud platforms only provide the simplest monitoring and logging features,
and do not provide any mechanisms for detecting performance anomalies or identifying
bottlenecks in the application code or the underlying cloud platform. This limitation has given rise
to a new class of third party service providers that specialize in monitoring cloud applications
(e.g. New Relic, Datadog). But these third party solutions require additional configuration 
and are expensive.
Furthermore, they only have a restricted view of the user application and the underlying
cloud platform, which limits their capabilities in terms of the type of analyses they can carry out.
Today's cloud platforms are also very large and complex with many interacting components.
An external monitoring service cannot see all this complexity, and hence cannot pinpoint
the component that might be responsible for an observed application performance anomaly.

In order to make cloud computing more reliable and dependable for the users as well
as the cloud service providers, the above limitations need to be addressed satisfactorily.
Doing so will also greatly simplify the tasks of developing cloud applications, and maintaining 
them in the long run. As a means of reaching this goal, I would like to propose and explore the
following thesis question.

{\bf Can we efficiently enforce governance on cloud-based web applications to achieve 
administrative conformance, developer best practices, and performance objectives through 
automated analysis and diagnostics?} 

Governance in this context can be defined as the mechanism by which the acceptable 
operational parameters are specified and maintained in a software system. This involves 
multiple steps:
\begin{itemize}
\item Specifying the acceptable operational parameters
\item Enforcing the specified parameters
\item Monitoring the system to detect deviations from the acceptable behavior
\item Taking preventive or corrective action when necessary
\end{itemize}

The concept of software system governance as defined here is very similar to the
notion of governing a country. In a country too we should specify the acceptable
parameters, which establishes the general laws and the rights of the citizens. Such 
parameters are then
enforced by various authorities. When some person or a group deviates from those parameters,
it is detected and corrective action is taken by bringing the responsible party to the justice.

Automated governance protocols have been designed and implemented for software systems in
the past, specially in classic web services and SOA applications. Such systems enable specifying
acceptable behavior via machine readable policies, which are then automatically enforced by
a policy enforcement agent. Monitoring agents watch the system to detect any deviations from
the acceptable behavior (i.e. policy violations), and alert users or follow predefined corrective
procedures. We can envision similar facilities being implemented in a cloud platform to 
enforce good programming practices and various performance objectives. The operational
parameters in this case may include coding and packaging conventions for the cloud-hosted
applications, and their performance limits.

In order for governance to be
useful and meaningful within the scope of the cloud, we must be able to meet two additional
goals. First, the governance system should be efficient in the sense that it should not introduce
a significant runtime overhead to the cloud-hosted applications, and it should scale up to
handle a large number of applications and policies. Secondly, the governance system should be
fully automated. Since cloud platforms are comprised of thousands of applications and components,
it is not practical for a human administrator to be involved in the governance process at any level.

Extending on the above thesis question, and the vision for efficient, automated governance for
cloud-hosted applications, I propose the following research plan:

\begin{itemize}
\item Design and implement a scalable, low-overhead governance framework for cloud platforms,
complete with a policy specification language and a policy engine. The governance framework should be
built into the cloud platforms, and must
strive to keep the runtime overhead of the user applications to a minimum while enforcing
common developer best practices and other organizational conventions.
\item Design and implement a methodology for formulating performance SLAs for cloud-hosted 
 web applications, without
 subjecting them to extensive performance testing or instrumentation. The formulated
 performance SLAs must be correct, tight and durable in the face of changing
 conditions of the cloud.
 \item Design and implement a scalable cloud application monitoring framework for detecting
application performance anomalies, SLA violations and diagnosing potential root causes. 
The framework should support collecting
 a wide range of runtime and usage data from the user applications and the underlying cloud platform
 without instrumenting the user code, and without introducing significant runtime overheads.
 It should further facilitate employing a number of new and existing statistical methods
 and algorithms to analyze the gathered data in near realtime.
\end{itemize}

Following section provides the background details regarding cloud computing and governance.
Next three sections further elaborate on the above three research goals respectively. 
Each section details our designs, assumptions and related work. Implementation details,
preliminary results and future work are also discussed where appropriate. Final section
summarizes and concludes this proposal.
