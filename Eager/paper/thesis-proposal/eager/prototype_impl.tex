We implemented a prototype of EAGER by extending AppScale~\cite{appscale13}, 
an open source PaaS cloud that is 
functionally equivalent to Google App Engine (GAE).  AppScale supports
web applications written in Python, Java, Go and PHP. Our prototype
implements governance for all applications and APIs hosted in an AppScale cloud. 

As described in subsection~\ref{sec:policy-lang}, 
our prototype policy specification
language is based on Python.
Using Python allows EAGER to leverage
existing programming tools to edit and debug policy files. 
It also allows the deployment control module (also written in Python) to
execute the policies (using a modified Python interpreter to implement the
restrictions previously discussed) directly.

%Using a real
%programming
%language to specify policies, enables administrators to implement governance
%policies with
%arbitrary complexity and logic with ease.

The prototype relies on a separate tool chain ({\em i.e.} one not hosted as a
service in the cloud) to automatically generate
API specifications and other metadata ({\em c.f.} Section~\ref{sec:adc}), which
currently supports only the Java language.  
Developers must document the APIs manually
for web services implemented in languages other than Java.

Like most PaaS technologies, AppScale includes an application deployment
service that distributes, launches and exports an application
as a web-accessible service.  EAGER controls this deployment
process according to the policies that the platform administrator specifies.

\subsubsection{Autogeneration of API Specifications}
To autogenerate API specifications, the build process for an application must
include an analysis phase that generates specifications from the code.
The prototype includes two stand-alone tools for implementing this
``build-and-analyze'' function.
\begin{enumerate}
\vspace{0.05in}
\item An Apache Maven archetype that is used to initialize a Java
web application project, and 
\vspace{0.05in}
\item A Java doclet that is used to auto-generate API specifications from web APIs implemented in Java
\vspace{0.05in}
\end{enumerate}

Developers can invoke the Maven archetype from the command-line to initialize
a new Java web application project. Our archetype sets up projects with the
required AppScale (GAE) libraries, Java JAX-RS~\cite{jaxrs} (Java API for RESTful Web
Services) libraries and a build configuration.

Once a developer creates a new project using the archetype s/he can develop
web APIs using the popular JAX-RS library. Once code is developed, it can be built
using our auto-generated Maven build configuration, which introspects the
project source code to generate specifications for all enclosed web APIs using
the Swagger~\cite{swagger} API description language. 
It then packages the compiled
code, required libraries, generated API specifications, and the dependency
declaration file into a single, deployable artifact.

Finally, the developer submits the generated artifact for deployment to the
cloud platform (which in our prototype is done via AppScale developer tools). 
To enable this, we modified the tools so that they
send the application deployment request to the EAGER ADC and
delegate the application deployment process to EAGER. This change required
just under 50 additional lines of code in AppScale.

\subsubsection{Implementing the Prototype}

\begin{table}[t]
\begin{center}
\begin{tabular}{| p{6cm} | p{7cm} |}
\hline
EAGER Component & Implementation Technology\\ \hline
Metadata Manager & MySQL\\
API Deployment Coordinator & Native Python implementation\\
API Discovery Portal & WSO2 API Manager~\cite{wso2apimgr}\\
API Gateway & WSO2 API Manager\\
\hline
\end{tabular}
\end{center}
\caption{Implementation Technologies used to Implement the EAGER Prototype}
\label{tab:imp-tech}
\end{table}
%\vspace{-0.2in}

Table~\ref{tab:imp-tech} lists the key technologies that we use to implement 
various EAGER functionalities described in
Section~\ref{sec:eager} as services within AppScale itself.  For example, AppScale
controls the lifecycle of the MySQL database as it would any of its other
constituent services.  EAGER incorporates the WSO2 API Manager~\cite{wso2am} for
use as an API discovery mechanism and to implement any run-time policy 
enforcement.  In the prototype, the API Gateway does not share policies
expressed in the policy language with the ADC (although this integration can
be implemented in the future).

Also, according to the architecture of EAGER, Metadata Manager is the most suitable
location for storing all policy files. The ADC may retrieve the policies from the Metadata
Manager through its web service interface. However, for simplicity, our current prototype stores
the policy files in a file system, that the ADC can directly read from. In a more sophisticated
future implementation of EAGER, we will move all policy files to the Metadata Manager
where they can be better managed, while providing easy access to other distributed
components of the cloud.

%Our default implementation of the EAGER ADC does not allow any third 
%party library
%calls. Also it disallows using all built-in Python modules except for the ``regex'' module used to
%evaluate regular expressions. As explained earlier these restrictions are managed through a
%whitelist which can be modified to allow more libraries if necessary.
