Cloud computing is a way of delivering infrastructure resources, programming platforms, and software
applications as shared utility services. Enterprises and developers increasingly deploy applications 
on cloud platforms due to the scalability, high availability and many other
productivity enhancing features the cloud is able to offer. Cloud-hosted applications 
depend on the core services provided
by the cloud platform for resources. In some cases they use the services provided by the cloud to implement most of
the application functionality as well (e.g. PaaS-hosted applications). 
Cloud-hosted applications are typically
accessed over the Internet, via the web APIs exposed by the applications.

As the number of applications hosted in cloud platforms continues to increase, the need for enforcing
governance on them becomes accentuated. We define governance as the mechanism by which the 
acceptable operational parameters are specified and maintained for a cloud-hosted application.
Governance enables specifying the acceptable
development standards and runtime parameters (performance, availability, security requirements etc.) 
for cloud-hosted applications as policies. Such
policies can then be enforced automatically at various stages of the application life-cycle. 
Governance further entails
monitoring cloud-hosted applications to ensure that they operate at a certain level of quality,
and taking corrective action when problems are detected. Through the steps of specification,
enforcement, monitoring and correction, governance can help resolve a number of prevalent issues in
today's cloud platforms. These issues include lack of good software engineering practices (code reuse,
dependency management, versioning etc), lack of performance SLAs for cloud-hosted applications,
and lack of performance debugging support. 

This proposal is aimed at exploring the feasibility of efficiently enforcing governance on cloud-hosted
applications, and evaluating the effectiveness of governance as a means of achieving administrative
conformance, developer best practices and performance objectives. Considering the scale of
today's cloud platforms we wish to automate much of the governance tasks in the cloud through
automated analysis and diagnostics. To achieve efficiency, we put more emphasis on deployment-time
policy enforcement and non-invasive passive observation of cloud platforms, thereby keeping the 
governance overhead
to a minimum. We avoid run-time enforcement and invasive instrumentation of cloud applications 
as much as possible. We also focus on building governance systems that are deeply integrated with
the cloud platforms. This enables using the existing scalability and high availability features of the cloud
to provide an efficient governance solution that can control all application events in a very fine-grained
manner. Furthermore, such integrated solutions relieve the users from having to maintain and pay
for an additional, external governance solution.