As more and more applications are deployed in cloud platforms, the need
for comprehensive performance monitoring and root cause analysis 
in cloud environments continues to grow in significance. However,
cloud platforms are large, complex systems with high levels of
programming abstractions. This makes it challenging to collect,
correlate and analyze data in a way that is amenable to detecting
performance anomalies and diagnosing bottlenecks. The existing
monitoring frameworks for cloud platforms are not integrated
well into the cloud platforms, which makes them incapable of
performing full stack monitoring, and diagnosing problems
outside user code in the core components of the cloud platform.

We propose Roots, a monitoring framework built into PaaS clouds
that can monitor all aspects of the cloud platform from the top
level front-end servers to PaaS kernel services of the lowest level.
Moreover, Roots supports correlating events that occur at different
levels of the cloud platform in response to user requests. This
in turns enables tracing the flow of user requests through the
cloud stack, thereby facilitating root cause analysis in a very
fine-grained manner. Roots is also highly extensible, and
encourages configuration at the granularity of individual
user applications. 

Roots collects data efficiently using a 
collection of passive sensors/agents, and avoids instrumenting
user code. The data analysis layer of Roots is organized as 
self-contained pods of lightweight processes. Pods can be
deployed in a distributed manner (with sharding), and is
amenable to replication thereby providing high levels of
scalability and availability. 

We are currently in the process of implementing a Roots
prototype for the AppScale open source PaaS cloud. We
use ElasticSearch as the primary data storage component in
our implementation. We plan to evaluate the effectiveness
of Roots using a wide range anomaly detection and root
cause analysis mechanisms. We also plan on testing the
overhaed and scalability of Roots using real world applications
and workloads.
