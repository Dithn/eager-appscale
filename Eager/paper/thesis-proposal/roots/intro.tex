Over the last decade cloud computing has become a popular approach for deploying
applications at scale. Many organizations, academic institutions, and hobbyists make use of public
and/or private clouds to deploy their applications.
This rapid growth in cloud technology has intensified the need 
for new techniques to
monitor applications deployed in cloud platforms. Application developers and users wish
to monitor the availability of the deployed applications, track application performance, and detect 
application and system anomalies as they occur. To obtain this level of deep operational insight into
cloud-hosted applications, the cloud platforms must be equipped with powerful instrumentation,
data gathering and analysis capabilities that span the entire stack of the cloud. 
Moreover, clouds must provide comprehensive
data visualization and notification mechanisms. However, most cloud technologies available
today either do not provide any application monitoring support, or only provide primitive
monitoring features such as application-level logging. Hence, they are not capable of performing
powerful predictive analyses or anomaly detection, which require much more fine-grained, low-level
and systemwide data collection and analytics. 

Further compounding this problem, today's cloud platforms are very large and complex. They are
comprised of many layers, where each layer may consist of a multitude of interacting components.
Therefore when a performance anomaly manifests in a user application, it is rather challenging 
to determine the exact layer or the component of the cloud platform that may be responsible for it. 
Facilitating this level of comprehensive root cause analysis requires
data collection at different layers of the cloud, and mechanisms for correlating events at
different layers and components. Today's cloud platforms do not support such deeply integrated
data collection. The plethora of existing third party cloud monitoring solutions
do not have access to low-level data regarding the cloud thereby rendering them incapable
of performing systemwide root cause analysis.

Moreover, performance monitoring for cloud applications needs to be highly customizable. Different
applications have different monitoring requirements in terms of data gathering frequency (sampling rate), 
length of the history to consider when performing statistical analysis, and the performance 
SLAs to maintain over time and check for violations. It should also be possible to easily extend
cloud application monitoring frameworks with
new statistical analysis methods and algorithms for detecting application performance
anomalies, and performing root cause analysis. Designing such customizable and extensible performance
monitoring frameworks that are built into the cloud platforms is a novel yet challenging undertaking.

To address these needs, we present the design of 
a comprehensive application platform 
monitor (APM) called Roots that can be easily integrated with a wide variety of cloud Platform-as-a-Service 
(PaaS) technologies. The proposed
APM is not an external system that monitors a cloud platform from the outside (as most APM systems today). 
Rather, it integrates with
the PaaS cloud from within thereby extending and augmenting the existing components of the PaaS cloud
to provide comprehensive full stack monitoring and analytics. 
We believe that this design decision is a key differentiator over existing cloud 
application monitoring systems because (i) it is
able to take advantage of the scaling, efficiency, deployment, fault tolerance, security, 
and control features that the underlying cloud offers, 
(ii) while providing low overhead end-to-end monitoring and analysis of cloud applications.

PaaS clouds execute web-accessible (HTTP/S) applications, to which they provide 
high levels of scalability, availability, and execution management. 
PaaS clouds provide scalability by automatically allocating resources 
for applications on the fly (auto scaling), and provide availability through
the execution of multiple instances of the application and/or the PaaS
services they employ for their functionality.
Consequently, viable PaaS technologies as well as
PaaS-enabled applications continue to increase rapidly in number~\cite{paas-growth}.
PaaS clouds provide a high level of abstraction to the application developer that effectively hides
all the infra\-structure-level details such as physical resource allocation (CPU, memory, disk etc), operating
system,
and network configuration. This enables application developers to focus solely on the programming
aspects of their applications, without having to be concerned about deployment issues. But
due to this high level of abstraction, performance monitoring and root cause analysis
is particularly challenging in PaaS clouds. Due to this reason, and the large number of 
PaaS applications available for testing, we design Roots APM to operate within PaaS
clouds.

Roots is highly customizable so that different monitoring policies can be
configured at the application level. It also facilitates a number of statistical analysis
methods for anomaly detection and root cause analysis. New analysis methods
can be easily brought into the framework by building on the high-level abstractions
that Roots provides. This enables us to experiment with different combinations of
statistical methods to determine which analysis works best for a given application or
SLA scenario. Roots collects most of the data it requires by instrumenting 
various components of the cloud platform. However, it takes special care to keep the data
collection overhead to a minimum. To this end it does not instrument any user code (i.e. applications)
deployed in the cloud. On a related note, it also does not impose any restrictions on the user code.
That is, the developer does not have to write code using some Roots-specific API or link their
code with a Roots-specific library. Any application that can run on the cloud platform without Roots, can
be monitored and analyzed by Roots with no code changes necessary. 

Roots uses batch operations and asynchonous 
communication whenever possible to record events in a manner that does not introduce
delays to the application request processing activities carried out by the 
PaaS cloud. 
Additionally, Roots employs a collection of lightweight continuous application benchmarking
processes to collect performance data regarding user applications. Both
the benchmarking processes, and the data analysis processes are executed 
out of the request processing flow of the cloud platform. Such processes can be
grouped together, and managed by a single deployable entity known as a
``Roots Pod''. Pods are specifically designed to keep minimum state
information regarding the applications they monitor and analyze. This enables
a single pod to monitor a large number of applications. Each pod is self-contained,
and therefore scalability and high availability can be achieved by running multiple pods (sharding),
and running multiple replicas of the same pod.

The following subsections detail the architecture of Roots APM, and how it integrates with a typical PaaS
cloud. We describe individual components of the APM, their functions and how they interact with each
other. Where appropriate, we also detail the concrete technologies (tools and products) that we plan to use to implement
various components of the APM, and provide our rationale and intuition behind choosing these technologies.
