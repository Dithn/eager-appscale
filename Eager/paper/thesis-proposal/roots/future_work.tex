We are currently in the process of implementing the proposed Roots APM for the open
source AppScale private cloud. Most of the data collection agents and sensors are already
in place, and we have been able to successfully integrate ElasticSearch into
the AppScale cloud platform. Some minimal changes to the AppScale cloud SDK implementation were needed to
support collecting cloud SDK call data, and reporting them to ElasticSearch efficiently.

AppScale uses Nginx as the frontend server all incoming user requests. We have been
successful at configuring Nginx to tag all user requests with unique identifiers for cross-layer
data correlation. The tags inserted by Nginx are used by other internal sensors and agents 
in the PaaS when reporting events, and we are able to execute queries against ElasticSearch
that aggregate events by request identifiers.
 
The simplest anomaly detector we wish to provide uses response time measurements gathered by periodic
application benchmarkers. These measurements are used to compute the level of SLA satisfaction
with respect to a predefined SLA. For example, assume a predefined application SLA which states
at least 95\% of the application requests should be processed under 50ms. Then for some fixed-length
of the sliding window, Roots can compute the proportion of benchmark results under 50ms. If this
proportion is less than 95\%, the detector can trigger an anomaly event.

Another anomaly detector will periodically compute the correlation between average response time and
the workload (number of requests per unit time) of the applications. Past work has employed
statistical tools such as Pearson's R and dynamic time warping to compute this correlation.
If the correlation drops below a certain threshold, an anomaly has been detected.

We will experiment with other detector implementations, and evaluate their effectiveness
in various application monitoring scenarios. Similarly we plan to support a wide range
of changepoint detection mechanisms in workload change analyzer. The basic Roots implementation
will support binary segmentation, PELT method and Chen and Lui algorithm~\cite{killick2012optimal, cl93}. 
We also wish
to investigate the feasibility of using linear regression to model the relationship 
between the time spent on PaaS SDK calls, and the total response time of application
requests. That way we can use an existing relative importance metric for multiple
linear regression for root cause analysis.

We plan to conduct a wide range of experiments using our Roots implementation for the
AppScale cloud. We will use real world App Engine applications where possible (AppScale
is API compatible with App Engine, and hence supports the same applications). To
further evaluate the effectiveness of our design choices, we will also test with
artificial applications that are specifically programmed to exhibit performance
anomalies in a deterministic way. This way we can verify whether Roots is
capable of detecting performance anomalies.
We will also inject faults into the underlying cloud platform (the PaaS kernel),
and test whether Roots can detect the resulting anomalies, and trace them to
exact fault injection point.

We will also evaluate the overhead of Roots, and report on how well Roots can
scale in the presence of very large number of applications, and request loads.
