We implemented a prototype of EAGER by extending AppScale~\cite{appscale13}, 
an open source PaaS cloud that is 
functionally equivalent to Google App Engine (GAE).  AppScale supports
web applications written in Python, Java, Go and PHP. Our prototype
implements governance for all applications and APIs hosted in an AppScale cloud. 
Our current prototype of the EAGER developer tools that automatically generate
API specifications and other metadata (c.f. Section~\ref{sec:adc}),
currently support only the Java language.  Developers document the APIs manually
for web services implemented in other languages.

Like most PaaS technologies, AppScale includes an application deployment
service that distributes, launches and exports an application
as a web-accessible service.  EAGER controls this deployment
process according to the policies that the platform administrator specifies.

\subsection{Developer Tools for EAGER}
EAGER provides two developer tools for implementing applications and APIs for an
EAGER-enabled cloud platform.
\begin{enumerate}
\vspace{0.05in}
\item An Apache Maven~\cite{maven} archetype that is used to initialize a Java
web application project, and 
\vspace{0.05in}
\item A Java doclet that is used to auto-generate API specifications from web APIs implemented in Java
\vspace{0.05in}
\end{enumerate}

Developers can invoke the Maven archetype from the command-line to initialize
a new Java web application project. Our archetype sets up projects with the
required AppScale (GAE) libraries, Java JAX-RS~\cite{jaxrs} (Java API for RESTful Web
Services) libraries and a build configuration.

Once a developer creates a new project using the archetype s/he can develop
web APIs using the popular JAX-RS library. Once code is developed, it can be built
using our auto-generated Maven build configuration, which introspects the
project source code to generate specifications for all enclosed web APIs using
the Swagger~\cite{swagger} API description language. 
It then packages the compiled
code, required libraries, generated API specifications, and the dependency
declaration file into a single, deployable artifact.

Finally, the developer submits the generated artifact for deployment to the
cloud platform (which in our prototype is done via AppScale developer tools). 
To enable this, we modified the tools so that they
send the application deployment request to the EAGER ADC and
delegate the application deployment process to EAGER. This change requires
just under 50 additional lines of code in AppScale.


\begin{table}[t]
\begin{center}
\begin{tabular}{| p{3cm} | p{3.5cm} |}
\hline
EAGER Component & Implementation Technology\\ \hline
Metadata Manager & MySQL~\cite{mysql}\\
API Deployment Coordinator & Native Python implementation\\
API Discovery Portal & WSO2 API Manager~\cite{wso2am}\\
API Gateway & WSO2 API Manager\\
\hline
\end{tabular}
\end{center}
\caption{Implementation Technologies used to Implement the EAGER Prototype}
\label{tab:imp-tech}
\end{table}
%\vspace{-0.2in}

Table~\ref{tab:imp-tech} lists the key technologies that we use to implement 
various EAGER functionalities described in
Section~\ref{sec:eager} as services within AppScale itself.  For example, AppScale
controls the lifecycle of the MySQL database as it would any of its other
constituent services.
