We implemented a prototype of EAGER for AppScale~\cite{appscale13}, 
an open source PaaS cloud
functionally equivalent to Google App Engine (GAE).  AppScale supports
web applications written in Python, Java, Go and PHP. Our prototype service
implements governance for any application hosted in an AppScale cloud. 
However,
currently we have prototyped the auto-generation governance tools for the
developer (as opposed to the
governance service component) for Java-based web services.
That is, our current set of development tools can only auto-generate API
specifications for web APIs implemented in Java. This does not mean that only
Java applications can export web APIs in our system. For web services not
written in Java, a developer can
provide the API specifications manually.

\subsection{Developer Tools for EAGER}
We provide two main tools for implementing applications/APIs for EAGER-enabled AppScale.
\begin{enumerate}
\item An Apache Maven~\cite{maven} archetype that is used to initialize a Java web application project
\item A Java doclet that is used to auto-generate API specifications from web APIs implemented in Java
\end{enumerate}

Developers can invoke the Maven archetype from the command-line to initialize
a new Java web application project. Our archetype sets up projects with the
required AppScale (GAE) libraries, Java JAX-RS~\cite{jaxrs} (Java API for RESTful Web
Services) libraries and a build configuration.

Once a developer creates a new project using the archetype he/she can develop
web APIs using JAX-RS. After the code has been finalized, it can be built
using the auto-generated Maven build configuration, which will introspect the
project source code to generate specifications for all enclosed web APIs using
the Swagger~\cite{swagger} API description language. 
It will then package all the compiled
code, required libraries, generated API specifications and the dependency
declaration file into a single deployable artifact.

Finally, the developer can submit the generated artifact for deployment in the
cloud, using the AppScale application deployment tools. We have made some
minor modifications to the original AppScale application deployment tools, so
that it sends an application deployment request to the EAGER ADC, and
delegates the application deployment process to EAGER. This change required
changes to less than 50 lines of AppScale code.

\subsection{Metadata Manager} 
We implemented the Metadata Manager using MySQL~\cite{mysql},
a relational database engine.  The database server was registered with the
built-in process coordination mechanism of AppScale, which automatically
cleans up and restarts crashed or faulty processes. 
%This is a good example of
%using the existing features in the cloud to implement governance in a reliable
%manner.

\subsection{API Deployment Coordinator}

We implemented the ADC as a Python server process in our prototype. This
process is also monitored and managed by the process coordination mechanism of
AppScale. The ADC implementation exports a secure web service interface which
is used by the AppScale application deployment tools to submit new deployment
requests. 
%The entire ADC component has been implemented in about 975 lines of

%Python code. 
%It loads the policy files from a specific directory in the server file
%system. In our experiments so far, we deployed new policies into the system
%by manually copying the policy files into this directory in the cloud. We are
%currently in the process of creating a toolkit for deploying and managing
%policies remotely, without accessing the ADC's file system directly.

\subsection{API Discovery Portal and API Gateway}

We used WSO2 API Manager~\cite{wso2am} to implement the ADP and API Gateway.
WSO2 API Manager is an open source API management solution that allows
creating online portals for publishing APIs, and facilitates API key
provisioning, API call authentication and rate limiting.  Our prototype starts
an instance of the WSO2 API Manager when the AppScale cloud starts up. The
product exposes several administrative web services, that we call from the ADC
to publish APIs to the ADP and API Gateway. 
