Cloud computing delivers IT infrastructure resources, programming platforms, and software
applications as shared utility services. Enterprises and developers increasingly deploy applications 
on cloud platforms due to their scalability, high availability and many other
productivity enhancing features. Cloud-hosted applications depend on the core services provided
by the cloud platform for compute, storage and network resources. 
In some cases they use the services provided by the cloud to implement most of
the application functionality as well (e.g. PaaS-hosted applications). 
Cloud-hosted applications are typically
accessed over the Internet, via the web APIs exposed by the applications.

As the applications hosted in cloud platforms continue to increase in number, the need for enforcing
governance on them becomes accentuated. We define governance as the mechanism by which the 
acceptable operational parameters are specified and maintained for a cloud-hosted application.
Governance enables specifying the acceptable
development standards and runtime parameters (performance, availability, security requirements etc.) 
for cloud-hosted applications as policies. Such
policies can then be enforced automatically at various stages of the application life-cycle. 
Governance also entails
monitoring cloud-hosted applications to ensure that they operate at a certain level of quality,
and taking corrective action when deviations are detected. Through the steps of specification,
enforcement, monitoring and correction, governance facilitates resolving a number of prevalent issues in
today's cloud platforms. These issues include lack of good software engineering practices (code reuse,
dependency management, versioning etc), lack of performance SLOs for cloud-hosted applications,
and lack of performance debugging support. 

We explore the feasibility of efficiently enforcing governance on cloud-hosted
applications, and evaluate the effectiveness of governance as a means of achieving administrative
conformance, developer best practices and performance SLOs in the cloud. Considering the scale of
today's cloud platforms in terms of the number of users and the applications, 
we strive to automate much of the governance tasks through
automated analysis and diagnostics. To achieve efficiency, we put more emphasis on deployment-time
policy enforcement, static analysis of performance bounds, and non-invasive passive monitoring of 
cloud platforms, thereby keeping the governance overhead
to a minimum. We avoid run-time enforcement and invasive instrumentation of cloud applications 
as much as possible. We also focus on building governance systems that are deeply integrated with
the cloud platforms themselves. This enables using the existing scalability and high availability features of the cloud
to provide an efficient governance solution that can control all application events in a fine-grained
manner. Furthermore, such integrated solutions relieve the users from having to maintain and pay
for additional, external governance and monitoring solutions.

In order to explore the feasibility of implementing efficient, automated governance systems in 
cloud environments, and evaluate the efficacy of such systems, we follow a three-step
research plan.
\begin{enumerate}
\item Design and implement a scalable, low-overhead policy enforcement framework for cloud platforms.
\item Design and implement a methodology for formulating performance SLOs for cloud-hosted applications.
 \item Design and implement a scalable application performance monitoring framework for 
 detecting and diagnosing performance anomalies in cloud platforms.
\end{enumerate}

We design and implement EAGER~\cite{6903538, eager-fop15} -- a lightweight governance policy
enforcement framework built into PaaS clouds. It supports defining policies using a simple syntax
based on the popular Python programming language. EAGER promotes deployment-time
policy enforcement, where policies are enforced on user applications (and APIs) every time
an application is uploaded to the cloud. By carrying out policy validations at
application deployment-time, and refusing to deploy applications that violate policies,
 we provide fail-fast semantics, which ensure that deployed applications are fully policy compliant. 
EAGER architecture also provides the necessary provisions for facilitating run-time policy
enforcement (through an API gateway proxy) when necessary. This is required, since not all
policy requirements are enforceable at deployment-time; e.g. a policy that prevents an
application from making connections to a specific network address. Our experimental results show
that EAGER validation and policy enforcement overhead is negligibly small, and it scales well to
handle thousands of user applications and policies. Overall, we show that integrated governance
for cloud-hosted applications is not only feasible, but also can be implemented with very
little overhead and effort.

To facilitate formulating performance SLOs, we design and implement 
Cerebro~\cite{Jayathilaka:2015:RTS:2806777.2806842} --
a system that predicts bounds on the response time of web applications developed for PaaS clouds.
Cerebro is able to analyze a given web application, and determine a bound on its response time without
subjecting the application to any testing or runtime instrumentation. This is achieved by a mechanism
that combines static analysis of application source code with runtime monitoring of the underlying
cloud platform (PaaS SDK to be specific). Our approach is limited to interactive web applications
developed using a PaaS SDK. We show that such applications have very few branches and loops, 
and they spend most of their execution time invoking PaaS SDK operations. These properties
make the applications amenable to both static analysis, and statistical treatment of their 
performance limits.

Cerebro is fast, can be invoked at the deployment-time 
of an application, and does not require any human input or intervention. 
The bounds predicted by Cerebro can be used as statistical guarantees (with well defined correctness
probabilities) to form performance SLOs. These SLOs in turns can be used in SLAs that are negotiated
with the users of the web applications. Cerebro's SLO prediction capability, coupled with a policy
enforcement framework such as EAGER, can facilitate specification and enforcement of performance-related
policies for cloud-hosted applications. We implement Cerebro for Google App Engine public cloud
and AppScale private cloud. Our experiments with real world PaaS applications show that Cerebro
is able to determine accurate performance SLOs that closely reflect the actual response time
of the applications. Furthermore, we show that Cerebro-predicted SLOs are not easily affected by
the dynamic nature of the cloud platform, and they remain valid for long durations. More specifically, 
Cerebro predictions remain correct for more than 12 days on average~\cite{7396174}. 

Finally, we design and implement Roots -- a performance anomaly detection and 
bottleneck identification system built into PaaS clouds. It collects data from all the different layers of the
PaaS stack; from load balancers to low level PaaS kernel service implementations. However,
it does so efficiently, without instrumenting user code, and without introducing a significant
overhead to the application request processing flow. 
Roots uses the metadata (request identifiers) injected by the load balancers to correlate the
events observed in different layers, 
thereby enabling tracing of application requests through the PaaS stack.
Roots is also extensible in the sense that 
any number of statistical analysis methods can be incorporated into Roots for performance
anomaly detection and diagnosis. Furthermore, it facilitates configuring monitoring requirements
at the granularity of user applications, which allows different applications to be monitored
and analyzed differently. 

Roots detects performance anomalies by monitoring applications for SLO violations. 
When an anomaly (i.e. an SLO violation) is detected, Roots is able to determine if
the anomaly was caused by a change in the application workload or by a performance 
bottleneck in one of the underlying PaaS kernel services. To this end we present a mechanism
that uses a combination of linear regression, change point detection and quantile analysis to
perform root cause analysis. We show that our combined approach makes correct diagnoses nearly
100\% of the time. Finally, we also present a path distribution analyzer that can identify different
paths of execution in an application via the run-time data gathered from the cloud platform.
We show that this mechanism is capable of detecting characteristic changes in application
workload as a special type of anomalies. Our design of Roots also consists of a component
that can detect quantitative changes in application workload.

Above results demonstrate that efficient and automated governance in cloud environments
is not only feasible, but also highly effective. We did not have to implement a cloud platform from the 
scratch to implement the governance systems designed as a part of this work. Rather,
we were able to implement the proposed governance systems in existing cloud platforms
like Google App Engine and AppScale; often with minimal changes to the cloud platform
software. Our implementations are integrated with the cloud platform (i.e. they operate from
within the cloud platform), and hence preclude 
the cloud platform users from having to set up or
implement their own external governance solutions that provide API management or
application monitoring functionality. Our governance systems are also efficient, in the 
sense they do not add a significant overhead to the applications deployed in the cloud
platform, and they scale well to handle a very large number of applications and governance
policies. In conclusion, our governance systems enable achieving developer best practices, 
administrative conformance and performance SLOs for cloud-hosted applications in ways
that were not possible before.