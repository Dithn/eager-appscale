Monitoring is a fundamental requirement for any large scale software installation.
Over the decades many monitoring frameworks have been designed and implemented to
support gathering and analyzing data to draw insights about system performance,
availability and faults. We studied a number of such frameworks including Nagios~\cite{nagios},
OpenNMS~\cite{opennms}, Shinken~\cite{shinken} and Zabbix~\cite{zabbix}. 
While all of them support data collection, storage,
analysis and visualization to varying degrees, none of them are designed
to operate as part of a cloud platform. Their data storage mechanisms (schema and query system),
APIs and configuration model are targetted at monitoring servers or 
applications as individual entities. They do not provide any support for
end-to-end tracing of request flows in a larger system. Further, they are not easily extensible,
supports only basic metric calculations, and provides no support for correlation
analysis or root cause analysis. We design Roots APM to be implemented as an integral
part of a cloud platform. Therefore it leverages the scalability and resource
management capabilities built into the cloud. It collects data from all
possible levels of the cloud platform (full stack monitoring), and facilitates
event correlation and root cause analysis. However, we do acknowledge the
overall architecture of some of the classic monitoring frameworks, and
adopt some of their design elements into Roots. Consequently, Roots also
has clearly separate components responsible for data collection, storage, analysis
and visualization.

Cloud platform monitoring is a rapidly evolving area with a number of 
competitive third party solutions already in the market. Platforms such
as New Relic~\cite{newrelic}, Datadog~\cite{datadog} and Dynatrace~\cite{dynatrace} 
provide hosted solutions with
agents for widely used cloud platforms such as Amazon AWS and Azure.
They provide comprehensive support for monitoring software
systems deployed on IaaS clouds, by planting agents on virtual
machines that execute the user code.
They are particularly strong in their ability to monitor transaction
processing systems, and some of them even facilitate (via code instrumentation) 
tracing observed events all the way down to code fragments and 
SQL queries in user applications.
But since these are third party monitoring frameworks, they cannot
monitor any parameters that are not exposed by the
cloud providers. This inhibits their use in PaaS clouds, that hide
many details of the cloud platform and the application runtime.  
Also these systems require additional configuration effort, and are expensive
which might preclude their use in some scenarios. Roots operates as a part
of the PaaS cloud, does not require additional configuration effort or code
instrumentation.

Anwar et al studied the monitoring facilities currently available in
open source cloud platforms like OpenStack~\cite{Anwar:2015:ACM:2797022.2797039}. 
They showed that these
frameworks use globally configured sampling rates for collecting data,
and provide poor support for policy-based monitoring. Roots attempts
to address these limitations by making it possible to configure
benchmarkers and anomaly detectors at application level. Each benchmarker 
and detector can have its own sampling rate and monitoring policy.
This allows fine tuning the monitoring support according to the
needs of the individual user applications.
Roots can also support changing sampling rates and monitoring
policies for applications at runtime if needed.

Dautov, Paraskasis and Stannett showed that a cloud platform monitor
can be organized as a sensor network~\cite{Dautov2014}. 
They argue that cloud platforms are 
dynamic (continuous change and evolution), distributed, have a high-volume
of applications and data, and heterogenous. To handle this complexity
they propose instrumenting different components of the cloud with
data collecting sensors. Sensors route data through a series of routing 
nodes into a central component that is responsible for storage
and analysis. We follow a similar approach in Roots, where we
instrument different layers of the PaaS cloud with sensors that
report data to a central storage. Independent pods of anomaly
detectors and anomaly handlers analyze the stored data in near
real-time.

Corradi et al designed and implemented a integrated monitoring
framework for the CloudFoundry PaaS~\cite{6912627}. This solution is organized
into two modules -- an availability monitor, and a performance
monitor. The availability monitor uses periodic heartbeats to
track the continuous operation of deployed applications. The
performance monitor uses some predefined application and
database benchmarks to periodically evaluate application performance.
While this solution is able to detect service outages and significant
performance anomalies, it provides no support for workload change
detection or root cause analysis.

Magalhaes et al have designed a series of systems for detecting
performance anomalies in web applications~\cite{5598229}. Some of their work
also addresses root cause analysis to some extent~\cite{Magalhaes:2011:RAP:1982185.1982234}. 
They use
various statistical methods (correlation analysis, dynamic time
warping etc) to detect anomalies in observed application
performance. Then they try to look for any correlations between
detected performance anomalies and workload level. If not
they attempt to perform root cause analysis. However, their
solution requires instrumenting application code, so that
backend API calls (e.g. database calls) can be intercepted and timed. To further
enable this they also require implementing the applications 
using an aspect-oriented programming (AOP) framework. Provided that
the application meets these requirements, their system is able to
pinpoint the bottleneck backend API calls using a linear regression
model. We incorporate some of their statistical methods into Roots,
but unlike their system Roots does not instrument application code, nor
it requires the application to be developed using a specific framework
such as AOP. Their root cause analysis method also assumes that
backend API call performance is independent, and shows no correlations
(i.e. no multicollinearity).
While this might be a reasonable assumption for classic web application
deployments, in the cloud where the underlying platform is shared
this assumption may not always hold. We therefore improve on their
root cause analysis technique by using regression models that are
resistant to multicollinearity~\cite{JSSv017i01}.

Changepoint analysis plays a big role when it comes to detecting
sudden changes in application performance and workload. This is a well
understood area in statistics, and we use a number of wellknown 
change point detection methods in the Roots implementation~\cite{killick2012optimal,cl93,bereznay2006did}. 
In general, we strive to support multiple statistical methods for
both anomaly detection, workload analysis and root cause analysis
thereby making it possible to compare and contrast different techniques.

Performance anomaly detection and bottleneck identification (PADBI) systems
have been studied by a number of researchers in the past~\cite{Ibidunmoye:2015:PAD:2808687.2791120}. 
However,
the community agrees that the scale, multi-tenancy, complexity,
dynamic behavior and the autonomy of cloud platforms make PADBI a
difficult problem to solve in the cloud. Roots is one step towards this direction,
and by making monitoring an integrated part of the cloud we believe
that we can deal with all the above challenges satisfactorily. In 
particular, an integrated solution like Roots has the advantage of
having full visibility into all the layers, components and interactions
in the cloud platform. This enables Roots to trace application-level
events and anomalies all the way down to the core services provided
by the cloud platform.
