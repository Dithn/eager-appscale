In this paper we highlighted the need for thorough API governance in the cloud, and 
presented the high-level design of our solution -- EAGER. We also demonstrated its
deployment-time API governance capabilities and scalability features via prototyping and 
empirical results. In the future we intend to further enhance EAGER and develop it into
a comprehensive end-to-end API governance facilitator for clouds by implementing the
run-time API governance support. This would allow cloud administrators and developers
to keep their APIs and dependent applications in check from  the moment they are rolled out into
the cloud until they are eventually retired. The current EAGER architecture has been carefully
designed to enable this by having all API traffic go through the API gateway and the API discovery
portal tracking the lifecycle of APIs.

However, run-time API governance imposes a number of new scalability and reliability challenges
that were not present with deployment-time API governance. More specifically, EAGER must ensure
that the API gateway can scale up to handle large volumes of API traffic and it will be always available
to receive and mediate API calls. By having API gateway integrated with the existing scalability and
availability mechanics of the cloud, we believe these requirements can be fulfilled to the expected
level (hence the need for having API governance as a native feature of the cloud).

We also intend to explore the ways our policy language can be further improved and simplified.
While the current state of the language is very flexible, too much flexibility can result in policy errors
that are hard to track and debug. Therefore we plan to further study the ways in which the policy language
can be further restricted and less error prone for the EAGER administrators. Also, work is currently underway
to develop tools for better managing the policy store of EAGER (i.e. adding, editing and removing policies),
with powerful atomic update and rollback support.

Another future research direction is the integration of policy language and run-time API governance. In
other words we wish to explore the possibility of using the same Python-based policy language for
specifying policies that are enforced on APIs at run-time (i.e. on individual API calls). Since API calls
are way more frequent than API deployment events, we need to pay attention to the performance aspects
of the policy engine to make this integration practically useful.

