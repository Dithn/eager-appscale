The popularization of network computing and the World Wide Web (WWW) 
has led to the development and adoption of web services~\cite{6094008} as
the technology of choice for implementing modern service-oriented 
architectures (SOA~\cite{Haines:2010:SAM:1787234.1787269}).
The interface portion of a web service, which abstracts and modularizes
its service implementation
details while making the service network-accessible, is commonly referred to
as a {\em web API}. As far as the users and applications that consume a 
web service
are concerned, the web API is the only point of contact and source of
functionality for the underlying service implementation.

Software engineering best practices separate the service implementation
from API, both during development and maintenance.
The service implementation and API are integrated via 
a ``web service stack'' that implements functionality common to all web
services (message routing, request authentication, etc.).
Because the API is visible to external parties ({\em i.e.} clients of the
services), any changes to the API
impacts users and applications not under the immediate administrative control
of the API provider.  For this reason, API features 
usually undergo long
periods of ``deprecation'' so that independent clients of the services can have
ample time to ``get ready'' for an API change.  At the same time,
technological innovations often prompt service reimplementation and/or 
upgrade to
achieve greater cost efficiencies, performance levels, etc.
Thus, APIs typically have a more
slowly evolving and longer lasting lifecycle than the service
implementations
to which they interface. 

Cloud computing is based on the idea of exposing some digital asset or a
capability ({\em e.g.} compute power, database, etc.) 
as a highly scalable web service.  Mobile
devices, due to their limited hardware resources often offload much of their
processing and storage needs to remote services running in a ``cloud''
connected to the Internet.  Web APIs
play a crucial role in both these paradigms. 

As a result, modern computing clouds, especially clouds implementing some form
of Platform-as-a-Service (PaaS)~\cite{4548165}, have accelerated the
proliferation of
web APIs and their use.  Most PaaS
clouds~\cite{appscale13,cloudfoundry,openshift} include
features designed to
ease the development and hosting of web APIs for scalable use over the Internet. 
This phenomenon is making API governance an absolute necessity in the cloud
environments.

In particular, API governance promotes code reuse among developers
since each API must be treated as a tracked and controlled software entity.
It also ensures that software users benefit from change control since the APIs
they use
change in a controlled and non-disruptive manner.  From a maintenance
perspective, API governance 
makes it possible to enforce best-practice coding procedures, 
naming conventions, and deployment procedures uniformly.
API governance is also critical to API lifecycle
management --  the management of deployed APIs in response to new feature
requests, bug fixes, and organizational priorities. 
API ``churn'' that results from lifecycle management
is a common phenomenon and a growing
problem for web-based applications~\cite{6930607}.
Without proper governance systems to manage the constant evolution of APIs,
API providers run the risk of making their APIs unreliable while potentially
breaking downstream applications that depend on the APIs.

Unfortunately, most web frameworks used to develop and host web APIs do not 
provide API governance facilities. This missing functionality is
especially glaring
for cloud platforms that are focused on rapid
deployment of APIs at scale.   Commercial pressures frequently prioritize
deployment speed and scale over longer-term maintenance considerations only to
generate unanticipated future costs.

As a partial countermeasure, developers of cloud-based web services are 
frequently given
additional tasks associated with 
implementing custom {\em ad hoc} governance solutions using either locally
developed mechanisms or loosely integrated
third-party API management services. 
These add-on governance
approaches often fall short in terms of their consistency and enforcement
capabilities since
by definition they have to operate outside the
cloud (either external to it or as a cloud-hosted application). 
As such, they do not have the end-to-end 
access to all the metadata and cloud-internal control mechanisms
that are necessary to implement strong governance at scale. 
