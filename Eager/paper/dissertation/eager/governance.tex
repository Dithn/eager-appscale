% enforcement problem: what and when
% requirements for a solution
In a cloud setting, enforcement of governance policies on APIs is a tradeoff between
what can be enforced, and when they are enforced. Performing policy enforcement
at application run-time provides full control over what can be enforced, since the policy 
engine can intercept and control all operations and instructions executed by the application.
However, this approach is highly intrusive, which introduces complexity and performance
overhead. Alternatively, attempting to enforce policies prior to application's execution is
attractive in terms of performance, but it necessarily limits what can 
be enforced. For example, ensuring that an
application does not connect to a specific network address and/or port
requires run-time traffic interception, typically by a firewall that is
interposed between the application and the offending network. Enforcing such a policy
can only be performed during run-time.

For policy implementation, often the additional complexities and overhead
introduced by run-time enforcement outweigh its benefits.  For example, in an
application that consists of API calls to services that, in turn, make calls
to other services, run-time policy enforcement can make violations difficult
to resolve, especially when the interaction between services is
non-deterministic.  When a specific violation occurs, it may be ``buried'' in a
lattice of API invocations that is complex to traverse, especially if the
application itself is designed to handle large-scale request traffic loads.

Ideally, then, enforcement takes place as non-intrusively as possible before
the application begins executing.  In this way, a violation can be detected and
resolved \textit{before the API is used}, thereby avoiding possible
degradations in user-experience that run-time checks and violations may
introduce.  The drawback of attempting to enforce all governance before the
application begin executing is that policies that express restrictions only
resolvable at run time cannot be implemented.  
Thus, for scalable applications that use API calls internally in a cloud
setting, an API governance approach should attempt to implement as much as
possible no later than deployment time, but must also include some form of
run-time enforcement.  

Note that the most effective approach to implementing a specific policy may not
always be clear.  For example, user authentication is usually implemented as a
run-time policy check for web services since users enter and leave the system dynamically.
However it is possible to check statically, at deployment time, whether the
application is consulting with a specific identity management service (accessed
by a versioned API) thereby enabling some deployment-time enforcement.

Thus, any efficient API governance solution for clouds must include the
following functionalities.
\begin{itemize}
\item {\bf Policy Specification Language} -- The system must include a way to
specify policies that can be enforced either at deployment-time (or sooner)
or, ultimately at run-time.  
\item {\bf API Specification Language} -- Policies must be able to refer to
API functionalities to be able to express governance edicts for specific APIs
or classes of APIs.
\item {\bf Deployment-time Control} --  The system must be able to check policies
no later than the time that an application is deployed.
\item {\bf Run-time control} -- For policies that cannot be enforced before
runtime, the system must be able to intervene dynamically.
\end{itemize}
In addition, a good solution should automate 
as much of the implementation
of API governance as possible.  Automation in a cloud context serves two
purposes.  First, it enables scale by allowing potentially complex
optimizations to be implemented reliably by the system, and not by manual
intervention.  Secondly, automation improves repeatability and auditability
thereby ensuring greater system integrity.