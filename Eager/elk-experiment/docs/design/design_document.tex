\documentclass[11pt]{report}
\begin{document}

\title{Architecture of the Proposed Cloud Application Platform Monitor}
\author{Chandra Krintz, Rich Wolski, Hiranya Jayathilaka, Wei-Tsung Lin}
\maketitle

\section{Introduction}
Over the last decade Platform-as-a-Service (PaaS) clouds have become a popular means for deploying
applications. Many organizations, academic institutes and hobbyists make use of public
and/or private PaaS clouds to deploy their applications.
PaaS clouds provide a high level of abstraction to the application developer that effectively hides
all the infrastructure-level details such as physical resource allocation (CPU, memory, disk etc),
and network configuration. This enables application developers to focus solely on the programming
aspects of their applications, without having to be concerned about deployment issues. PaaS
clouds can also ensure high levels of scalability and availability for the deployed applications. 
Scalability is typically provided by automatically allocating resources for applications
on the fly (auto scaling), and availability is ensured by running multiple instances of the application.
Consequently the viable PaaS technologies, and the PaaS-deployed applications
continue to increase in number.

This rapid growth in PaaS technology has intensified the need for powerful new techniques to
monitor applications deployed in a PaaS cloud. Application developers and users need
to monitor the availability of the deployed applications, track application performance and detect 
application and system anomalies as they occur. 
\end{document}