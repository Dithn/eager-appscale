\documentclass[11pt]{article}
\usepackage{graphicx}
\DeclareGraphicsExtensions{.png,.jpg}
\graphicspath{{./figures/}}

\begin{document}

\title{Architecture of the Proposed Cloud Application Platform Monitor}
\author{Chandra Krintz, Rich Wolski, Hiranya Jayathilaka, Wei-Tsung Lin}
\maketitle

\section{Introduction}
Over the last decade Platform-as-a-Service (PaaS) has become a popular approach for deploying
applications in the cloud. Many organizations, academic institutes and hobbyists make use of public
and/or private PaaS clouds to deploy their applications.
PaaS clouds provide a high level of abstraction to the application developer that effectively hides
all the infrastructure-level details such as physical resource allocation (CPU, memory, disk etc), operating
system configuration, 
and network set up. This enables application developers to focus solely on the programming
aspects of their applications, without having to be concerned about deployment issues. PaaS
clouds can also ensure high levels of scalability and availability for the deployed applications. 
Scalability is typically provided by automatically allocating resources for applications
on the fly (auto scaling), and availability is ensured by running multiple instances of the application.
Consequently the viable PaaS technologies, and the PaaS-deployed applications
continue to increase in number.

This rapid growth in PaaS technology has intensified the need for new techniques to
monitor applications deployed in a PaaS cloud. Application developers and users wish
to monitor the availability of the deployed applications, track application performance and detect 
application and system anomalies as they occur. To obtain this level of deep operational insight into
PaaS-deployed applications, the PaaS clouds need to be equipped with powerful instrumentation,
data gathering and analysis capabilities that span the entire stack of the PaaS cloud. 
Moreover, PaaS clouds need to provide comprehensive
data visualization and notification mechanisms. However, most PaaS technologies available
today either do not provide any application monitoring support, or only provide primitive
monitoring features such as application-level logging. Hence, they are not capable of performing
powerful predictive analyses or anomaly detection, which require much more fine-grained, low-level
and full stack data collection and analytics. 

To address this limitation we intend to design and implement a comprehensive application platform 
monitor (APM) that can be easily integrated with a wide variety of PaaS technologies. The proposed
APM is not an external system that monitors a PaaS cloud from the outside. Rather, it integrates with
the PaaS cloud from within thereby extending and augmenting the existing components of the PaaS cloud
to provide comprehensive full stack monitoring, analytics and visualization capabilities. In other words,
the APM is built into the PaaS cloud so that it is active as long as the cloud is, and has operational
insight to both core cloud components as well as the deployed applications.

This document details the architecture of the proposed APM, and how it integrates with a typical PaaS
cloud. We describe individual components of the APM, their functions and how they interact with each
other. Where appropriate we also detail the concrete technologies (tools and products) used to implement
various components of the APM, and give rationale for choosing those technologies.

We start by describing the layered system organization typically seen in PaaS clouds. Then we describe
the APM architecture, and show how it fits into the PaaS.

\section{PaaS System Organization}
\begin{figure}
\centering
\includegraphics[scale=0.5]{paas_architecture}
\caption{PaaS system organization.}
\label{fig:paas_architecture}
\end{figure}

Figure~\ref{fig:paas_architecture} shows the key system layers of a typical PaaS cloud. Arrows indicate
the flow of data and control in response to application requests.

At the lowest level of a PaaS cloud is an infrastructure that consists of the necessary compute, storage
and networking resources. How this infrastructure is set up may vary from a simple cluster of physical 
machines to a comprehensive Infrastructure-as-a-Service (IaaS) solution. In large scale PaaS clouds,
this layer typically consists of many virtual machines and/or containers with the ability to acquire more
resources on the fly.

On top of the infrastructure layer lies the PaaS kernel. This is a collection of managed, scalable
services that high-level application developers can compose into their applications. The provided services
may include database services, caching services, queueing services and much more. Some PaaS clouds
provide a managed set of APIs (an SDK) for the application developer to access these fundamental services. 
In that case all interactions between the applications and the PaaS kernel must take place through
the cloud provider specified APIs (e.g. Google App Engine). 

One level above the PaaS kernel we find the application servers that are used to deploy and run
applications. Application servers provide the necessary integration (linkage) between application code and the
underlying PaaS kernel, while sandboxing application code for secure, multi-tenant operation. On top
of the application servers layer resides the fronted and load balancing layer. This layer is responsible
for receiving all application requests, filtering them and routing them to an appropriate application
server instance for further execution. As the fronted server, it is the entry point for PaaS-deployed
applications for all application clients.

\section{Cloud APM Architecture}
\subsection{Key Functions}
\begin{figure}
\centering
\includegraphics[scale=0.5]{apm_functions}
\caption{Key functions of the APM.}
\label{fig:apm_functions}
\end{figure}

\begin{figure}
\centering
\includegraphics[scale=0.5]{apm_layout}
\caption{Deployment view of the APM functions.}
\label{fig:apm_layout}
\end{figure}

Like most system monitoring solutions, the proposed cloud APM needs to serve four major functions: Data
collection, storage, processing (analytics) and visualization. Figure~\ref{fig:apm_functions} shows the
logical organization of these functions in the APM, and various tasks that fall under each of them.
Figure~\ref{fig:apm_layout} shows a physical deployment view of the said functions. Arrows indicate
the flow of information through the APM.

Data collection is performed by various sensors and agents that instrument the applications and the
core components of the PaaS cloud. While sensors are very primitive in their capability to monitor
a given component, an agent may intelligently adapt to changing conditions, making decisions on
what information to capture and how often. Instrumentations should be lightweight and as non-intrusive
as possible so their existence does not put any additional overhead on the applications.

Data storage components should be capable of
dealing with potentially very high volumes of data. The data needs to be organized and indexed
to facilitate efficient retrieval, and replicated to maintain reliability and high availability. 

Data processing components should also be capable of processing large volumes of data in near real-time,
while supporting a wide range of data analytics features such as filters, projections and aggregations. 
They will employ various statistical and perhaps even machine learning methods to understand the
data, detect anomalies and identify bottlenecks in the system.

Data visualization layer mainly consists of graphical interfaces (dashboards) for displaying various
metrics computed by the data processing components. Additionally it may also have APIs to export
the calculated results and trigger alerts. 

\subsection{APM Architecture and Integration with PaaS}
\begin{figure}
\centering
\includegraphics[scale=0.35]{apm_architecture}
\caption{APM architecture.}
\label{fig:apm_architecture}
\end{figure}

Figure~\ref{fig:apm_architecture} illustrates the overall architecture of the proposed APM, and how 
it fits into the PaaS cloud stack. APM components are shown in grey, with their interactions indicated
by the black lines. The small grey boxes attached to the PaaS components represent the sensors and
agents used to instrument the cloud platform for data collection purposes. Note that the APM collects
data from all layers in the PaaS stack (i.e. full stack monitoring).

From the frontend and load balancing layer we gather all information related to incoming application
requests. A big part of this is scraping the HTTP server access logs, which indicate request timestamps,
source and destination addressing information, response time (latency) and other HTTP message
parameters. This information is readily available for harvesting in most technologies used as frontend
servers (e.g. Apache HTTPD, Nginx). Additionally we may also collect information pertaining to active
connections, invalid access attempts and other errors.

From the application server layer we intend to collect basic application logs as well as any other logs and 
metrics that can be easily collected from the application runtime. This may include some process level
metrics indicating the resource usage of the individual application instances. If deeper insight into the 
application execution becomes necessary, more intrusive instrumentation can be introduced to the 
application server (perhaps selectively or adaptively). 

At the PaaS kernel layer we employ instrumentation to record information regarding all kernel invocations
made by the applications. This instrumentation needs to be applied carefully as to not introduce a noticeable
overhead to the application execution. For each PaaS kernel invocation, we can capture the 
following parameters.
\begin{itemize}
\item Source application making the kernel invocation
\item Timestamp
\item Target kernel service and operation
\item Execution time of the invocation
\item Request size, hash and other parameters
\end{itemize}
Collecting this PaaS kernel invocation details enables tracing the execution of application 
requests, without the need for instrumenting application code, which we believe is a feature 
unique to PaaS cloues. 

Finally, at the lowest infrastructure level, we can collect information related to virtual machines, containers
and their resource usage. We can also gather metrics on network usage by individual components which
might be useful in a number of traffic engineering use cases. Where appropriate we can also scrape
hypervisor and container manager logs to get an idea of how resources are allocated and released over
time.

\subsection{Cross-layer Data Correlation}
Previous subsection details how the APM collects useful monitoring data at each layer of the cloud
stack. To make most out of the gathered data, and use them to perform complex analyses, we need
to be able to correlate data records collected at different layers of the PaaS. For example consider
the execution of a single application request. This single event results in following data records at
different layers of the cloud, which will be collected and stored by the APM as separate entities.

\begin{itemize}
\item A frontend server access log entry
\item An application server log entry
\item Zero or more application log entries
\item Zero or more PaaS kernel invocation records
\end{itemize}

We need a mechanism to tie these disparate records together, so the data processing layer can easily
aggregate the related information. For instance, we need to be able to retrieve via an
aggregation query, all PaaS kernel invocations made by a specific application request.

To facilitate this we propose that frontend server tags all incoming application requests with unique identifiers.
This request identifier can be attached to HTTP requests as a header which is visible to all components 
internal to the PaaS cloud. All data collecting agents can then be configured to record the request identifiers
whenever recording an event. At the data processing layer APM can aggregate the data by request identifiers
to efficiently group the related records.
\end{document}